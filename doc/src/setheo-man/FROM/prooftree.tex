\documentstyle[12pt,a4]{article}


\title{Generation of Proof-trees for SETHEO}
\author{Johann Schumann}
\date{22.7.92}

\begin{document}

\maketitle

\begin{abstract}
This document specifies the changes necessary for the output of the proof
tree.

\end{abstract}

\section{Description of Change}
Within this document, necessary changes are described to debug the
generation of the proof-tree during the run of setheo.

\subsection{Solution Provided}

In the current version of setheo (V3.01 or less), the three commands
{\bf disp, dptree}, and {\bf ptree} are used to generate the proof
tree. In the first run (the proof-search mode) where {\tt pmode = 0},
information necessary for the {\em reproof-mode}
is assembled.

After a proof could be found, a second run is started with {\tt pmode} set to
1. This causes the {\bf call} instruction to get the branching address
not from the code-area (ie. from the argument), but rather from the proof-tree
area. Therefore, a {\em deterministic} construction of the tableau is be
obtained. An instruction {\bf disp} at the end of each contrapositive
causes the current clause-head to be printed.

The proof-tree is printed then in a bottom-up\footnote{This means
that the nodes of the tableau are listed in post-order.}
manner in a given format.


\subsection{The Problem}
With the algorithm as described above, the following problem occurrs:

In certain cases, the literals which have been printed
carry not the full instantiation.
This is the case, whenever a variable gets instantiated {\em after\/} the
resp.\ clause or contrapositive has been displayed.

\section{Requirements for new Solution}

\begin{itemize}
\item
all instantiations of the literals printed must be correct,
\item
the overhead wrt.\ time and space within the SAM is to be kept as
minimal as possible,
\item
the generation and especially the display of the proof-tree must be optional,
\item
a (partial) tableau is to be displayed on request whenever the system
is interrupted,
\item
the generation of the proof tree must be flexible in order to allow
for an additional display of all subgoals still to be proven
in case, the proof has not been found yet,
\item
the generation of the proof-tree is needed only, if NO tail-recursion
optimisation is being made within setheo, and
\item
the solution proposed must allow for extensions of setheo and
the dynamic reordering of subgoals (local and global).
\end{itemize}

\section{Description of New Solution}
The solution proposed is based on the fact, that the entire proof-tree (or
tableau) is present (after a proof could be found) in the {\em environments\/}
on the stack. This, of course requires that no tail-recursion optimisation
takes place. The dynamic links between the environments, however, do not
describe the entire proof-tree, they only describe the
{\bf branches} of the tree.

Therefore, additional structures must be generated such that a proof-tree can
be displayed in the correct order {\em without\/} a reproof.

This means, that for {\em each\/} environment there must exist pointers
to the environment of each child (i.e. for each subgoal of the
clause/contrapositive).
This information is kept on the atack and is appended to each environment.
One additional pointer (TP) is added to the environment.

For a more optimised version, the entry {\tt ps\_symb} in the environment
is replaced by a pointer into the newly defined {\bf Literal Control Block}.
This control-block which resides in the code-area and which also contains
the argument vector the contains this information.

\newpage
\section{Data Structures}

\subsection{The new Environment}

From the environment, the entry {\tt ps\_symb} is removed. Instead, it contains
the following entries:

\begin{verbatim}
bp-->+--------------------------+
     |      dyn_link            |
     +--------------------------+
     |      ret_addr            |
     +--------------------------+
     |      gpr                 |
     +--------------------------+
     |      lit_cntrl_blck *--------------------------> Lit_cntrl_block
     +--------------------------+
     |      TP             *------------------------+
     +--------------------------+                   |
     |      TP_start       *--------------------+   |
     +--------------------------+               |   |
     |       variables          |               |   |
                ...                             |   |
                                                |   |
     |                          |               |   |
     +--------------------------+               |   |
     |         etc              |               |   |
                ...                             |   |
     +--------------------------+ <-------------+   |
     |     >for SG1 <           |                   |
     +--------------------------+                   |
               ...               <------------------+
     +--------------------------+
     |     >for SGn<            |
     +--------------------------+

\end{verbatim}

The {\tt Lit\_cntrl\_block} in the code area contains all information
necessary to describe the contrapositive with the given head.
It is placed {\em in front} of the {\bf argument vector} used to describe
its arguments. Henceforth, it has {\em negative\/} offsets wrt.\ the
argument pointer.

It is defined as a struct and has to contain at least the following
items:

\begin{verbatim}
struct list_cntrl_block {
        WORD        ps_symb             // the same as ps_symb in the old
                                        // environment
        WORD        sign;
        WORD        clause_nr;
        WORD        lit_nr;        
        WORD        nr_sgls;
        ....
        };
// the argument vector directly follows:
argvxxx:
        .dw        xxx
\end{verbatim}

Furthermore, this literal control block may contain additional
(static) information like
\begin{itemize}
\item
name of the local variables,
\item
number of distinct and common variables in this contrapositive, and
\item
information for heuristic control.
\end{itemize}

This control block is being generated statically during compile time.
    

In the {\bf choice-point}, the entry {\tt red\_lvl} is being ommitted.[

\section{SAM instructions involved}

\subsection{The alloc-instruction}

The alloc instruction must be modified as follows:

The syntax is changed into 
\centerline{\bf alloc nrvars, argv}
where {\bf argv} points to the argument-vector.

The pseudo-code for alloc will be:

\begin{verbatim}
alloc:
        ...
        // link to the old environment
        if (proof-tree && bp) bp->TP = (environ *)sp;

        // create the environment as usual
        // ... bp = (environ *)sp;
        // ... bp->ret_addr = ra;
        // ... bp->gpr = gp;

        // NEW
        bp->lit_cntrl_block = ARG(2);
        ...
        sp +=...
                // need that block only, if there are subgoals
        if (proof-tree && (ARG(2).nr_sgls ){
                // set TP to one below the ``TP-list''
		// and TP_start to its beginning
                bp->TP = sp -1;      
		bp->TP_start = sp;
                // allocate space for the ``TP-list''
                sp +=  (ARG[2]).nr_sgls;
        else
                bp->TP_start = bp->TP = NULL ;      
        ...
\end{verabatim}

\subsection{The call-instruction}

The call instruction just has to modify the {\tt TP} pointer of its
current environment. This change must be trailed.

\begin{verbatim}
call:
        ...
        if (proof-tree){
                trailvar(&(bp->TP));
                bp->TP ++;
                }
        ...
\end{verbatim}

\subsection{The reduct-instruction}

In this instruction, all references to {\tt chp->red\_lvl} are removed.

\subsection{Other instructions}
Due to the changed data within the environment, each reference to
{\tt bp->ps\_symb} must be modified to {\tt bp->lit\_cntrl\_block->ps\_symb}.

Furthermore, all occurrences of {\tt pmode} are removed.


\subsection{New instruction: disptree}

This new instruction does not take any arguments. It prints the 
proof-tree {\em below\/} the current environment (if {\tt proof-tree} is set)
into the proof-tree file.
This instruction always succeeds.

This instruction calls the function {\tt disp\_proof\_tree(environ *bp)}.

\section{Additional functions within SETHEO}

The additional function {\tt disp\_proof\_tree(environ *bp)} prints
the current proof-tree below the environment given by bp.

\begin{verbatim}
disp_proof_tree(environ *bp)
{
if (!proof-tree) return;

tp = bp->TP_start;

if (tp){
    // for all subgoals (accessible via bp->lit_cntrl_block->nr_sgls))
    //
    while (tp <= bp->TP){
         if (tp <= bp) {
             // this is an reduction step
             // go up the dynamic link-list starting at bp
             // until it is equal *tp. With this code, the reduction
             // level can be obtained
             ... }
         else {		
             // we have an extension step
             //
             disp_proof_tree(*tp);
             tp++;
             }
        }
print-head-info of current subgoal...
}
\end{verbatim}
        



\section{Required Changes in mplop}

The instructions {\bf dptree, disp, ptree} are not generated when the
tree-option is active.

The tree-generation option is skipped

For each literal, the literal-control-block is generated prior to the
generation of the argument vector.

In front of the {\bf stop} instruction ( or the {\bf fail}
instruction, if the {\bf -all} option is activated)
 in the query, the new instruction
{\bf disptree} is placed.

\section{Required Changes in wasm}

\begin{itemize}
\item The {\bf alloc} instruction takes two arguments only

\item The new {\bf disptree} instruction takes 0 arguments

\item The instructions {\bf dptree, disp, ptree} are removed.
\end{itemize}

\section{Proposed Time Schedule}

\begin{center}
\begin{tabular}{|l|r|r|r|}
\hline\hline
Action & mplop & wasm & setheo \\
\hline
Specification & & & \\
Implementation & & & \\
Test & & & \\
Documentation & & & \\
Review & & & \\
\hline
Total & & & \\
\hline\hline
\end{tabular}
\end{center}
\section{History}

\begin{center}
\begin{tabular}{|l||c|c||c|c||c|c||}
\hline\hline
Action & \multicolumn{2}{|c|}{mplop} &
\multicolumn{2}{|c|}{wasm} & \multicolumn{2}{|c|}{setheo}  \\
\hline
& Name & Date & Name & Date & Name & Date \\
\hline\hline
 Proposed:& & & & & & \\
 Specified:& & & & & & \\
 Approval of Specification:& & & & & & \\
 Start of Implementation:& & & & & & \\
 End of Implementation:`& & & & & & \\
 Start of Tests:& & & & & & \\
 End of Tests:& & & & & & \\
 Approval of Code:& & & & & & \\
 Integration into Version: & & & & & & \\
 Start of Tests (of new Version):& & & & & & \\
 End of Tests (of new Version):& & & & & & \\
 Approval of Change:& & & & & & \\
\hline\hline
\end{tabular}
\end{center}

\end{document}
