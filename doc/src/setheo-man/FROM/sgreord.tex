\documentstyle[12pt,a4]{article}


\title{A general Framework for Global Reordering of Subgoals
--- 1st Draft ---}
\author{Johann Schumann}
\date{29.7.92}

\begin{document}

\maketitle

\begin{abstract}
This document specifies the data structures and instructions which
are necessary to allow for a global (dynamic) reordering of subgoals.

\end{abstract}

\section{Description of Change}

\subsection{Solution Provided}

see Carlson 1987

\subsection{The Problem}

\section{Requirements for new Solution}

\section{Description of New Solution}

The basic idea of handling subgoal-reordering within the SAM is similar to
that found in the PROLOG-predicates ``freeze'' and ``delay'' (see e.g.\ \cite{Carl87}).
Each subgoal  which has to be solved is considered to be a ``light-weight'' process which has to be executed eventually.
The order of execution of the subgoals is controlled by a given
subgoal-selection function. This function will have access to the list of
all (still unsolved) subgoals with their attached information. From this list,
the subgoal to be solved next has to be selected.

Note, however, that with such an approach, the memory requirements
for the search very much depend on the order, in which the subgoals are executed.
E.g.\ the requirements are minimal, if the subgoals are executed in a LIFO manner.
If, however the subgoals are to be executed in a FIFO-manner, a breadth first
search is performed. In this search mode, all data have to be kept.

First, let us define what is a subgoal within the SETHEO abstract machine:

A subgoal comes into existence, when the SAM-instruction ``call'' is being
executed.
Therefore, a subgoal cal be seen as the tuple
$( pc,bp,{\rm argv},{\rm resources})$ where {\em pc} is the jump-address,
{\em bp\/} is the pointer to the current environment, and {\em argv\/}
is the second argument of the call instruction, pointing to the
parameters of the subgoal. Furthermore, the {\em resources\/}, i.e.\ the bounds
and counters are necessary. All other registers of the SAM are not
needed to uniquely define a subgoal.


Within the SAM, subgoals are considered as small processes. One is
created whenever a {\bf call} instruction is executed. After that, the SAM
proceeds in the execution of the next instruction instead of calling the
subgoal.
Additionally, a {\bf signal} is set.

Whenever one inference has been finished, the status of the signal is
to be checked and subgoals which are in the waiting list have to be executed.
This is done, whenever a {\bf dealloc} instruction is being executed.

This execution can be accomplished by the following mechanism:

\begin{enumerate}
\item
save the current SAM-status on the stack.
\item
load the information contained in the process (subgoal) to be executed
next.
\item
call this subgoal with a special return-address pointing to the
following instruction which is executed {\em after\/} this subgoal
has been solved:
\item
restore the old SAM-status and proceed solving the old subgoal.
In this step, it must be checked, if there are subgoals still waiting to
be executed.
\end{enumerate}

The steps executed above can be expressed by a number of SAM-instructions:

\begin{verbatim}
    ...
    signal  subgoals,sublabel         // at the beginning of the formula,
    ...                               // activate the signal mechanism
    
sublabel:
    save_cntxt                        // save the SAM-registers
    ...                               // possibly reorder waiting list
    call_cntxt                        // call the new context
                                      // after executing the process,
    rest_cntxt                        // restore the old context and proceed
\end{verbatim}

\section{Data Structures}


We have an additional register {\tt susp\_ptr}, a pointer to the list of waiting
subgoals. This list is kept on the heap (for a first version).
All assignments and changes to this list are being trailed.

Each subgoal-process has the following data-structure:
\begin{verbatim}
typedef struct s_p {
    WORD        *pc;                 // starting address of process
    WORD        *argv;               // pointer to argument vector
    environ     *bp;
    int         depth; 
    ...
    struct s_p  *next;               // pointer to next process
    struct s_p  *prev;               // pointer to previous process
    } SAM_process;
\end{verbatim}

For ease of handling, this list is kept in the {\bf heap} (Version 1).

Whenever a process is activated, a {\tt process-control-block} is being
allocated on the stack (and pointed to by {\tt bp}). All information
necessary to restore the old process environment is kept in this
structure:

\begin{verbatim}
typedef struct {
    WORD        *pc;                 // starting address of process
    WORD        *argv;               // pointer to argument vector
    environ     *bp;
    int         depth; 
    } process_control_block;
\end{verbatim}

\section{SAM instructions involved}
Whenever the {\bf dealloc} instruction is being executed,
it will be checked, if there are any processes to wake up.
If yes, the function {\tt switch\_cntxt} is called.

\section{New SAM instructions}
\subsection{scall}
The {\bf scall} instruction is used in the same context as the {\bf call}
instruction. In contrast to the call instruction which {\em directly\/} calls
a subgoal, it generates a new process denoting this subgoal to be executed,
and proceeds.

This instruction has the following pseudo-code:

\begin{verbatim}
sam_scall(){
SAM_process *wp;
                        // get a new process entry
wp= (SAM_Process *)hp;
hp += sizeof(wp)/sizeof(WORD);

// fill up the new process entry
wp->bp = bp;
wp->depth = depth;
...

// fill in the arguments
wp->pc   = ARG(1);
wp->argv = ARG(2);

// and link the new process into the list pointed to by susp_ptr
wp->next = susp_ptr;
TRAIL_VAR(susp_ptr);
susp_ptr = wp;
return 1;               // we are done
}
\end{verbatim}

\subsection{save\_cntxt}
This SAM-instruction is placed in code to which the interpreter branches
when it detects that a process is waiting to be executed.
This instruction allocates a {\tt process\_control\_block} on the stack
in the way similar to an environment, and fills up this block
with data of the SAM registers.

\begin{verbatim}
sam_save_cntxt(){
process_control_block *pcb;

pcb = sp;
sp += sizeof(process_control_block)/ sizoef(WORD);

// fill up the process control block
pcb->bp = bp;
bp = pcb;

// pc, ra ??
pcb->pc = signal_saved_pc;            // must still be evaluated
...
}
\end{verbatim}

\subsection{call\_cntxt}
This instruction causes the process selected (pointed to by {\tt susp\_ptr})
to be executed and the process marked as ``executed''.
This however, is backtrackable.
From the outside this instruction works like the {\bf call} instruction.
\begin{verbatim}
sam_call_cntxt(){

pc = susp_ptr->pc;
...
}
\end{verbatim}

\subsection{rest\_cntxt}
This instruction is used to restore the old environment after the execution of
the process. The process control block (pointed to be {\tt bp}) is
used to restore the SAM-registers.
Afterwards, this block is deallocated from the stack if possible.

After execution of this instruction, the SAM resumes the execution of the
process which had been processed prior to handling the signal.

\begin{verbatim}
sam_rest_contxt(){
..
}
\end{verbatim}

\section{Required Changes in mplop}

If subgoal reordering is requested, the {\bf call} instructions of those
subgoals must be replaced by {\bf scall} instructions.
Furthermore, the instruction {\bf signal} must be inserted before any
reordering can take place.

The code for handling the signal ({\bf save\_cntxt, call\_cntxt, rest\_cntxt})
must be generated.


\section{Required Changes in wasm}

The new instructions must be understood by the {\bf wasm} and must produce
the correct op-code.

\end{document}

\section{Proposed Time Schedule}

\begin{center}
\begin{tabular}{|l|r|r|r|}
\hline\hline
Action & mplop & wasm & setheo \\
\hline
Specification & & & \\
Implementation & & & \\
Test & & & \\
Documentation & & & \\
Review & & & \\
\hline
Total & & & \\
\hline\hline
\end{tabular}
\end{center}
\section{History}

\begin{center}
\begin{tabular}{|l||c|c||c|c||c|c||}
\hline\hline
Action & \multicolumn{2}{|c|}{mplop} &
\multicolumn{2}{|c|}{wasm} & \multicolumn{2}{|c|}{setheo}  \\
\hline
& Name & Date & Name & Date & Name & Date \\
\hline\hline
 Proposed:& & & & & & \\
 Specified:& & & & & & \\
 Approval of Specification:& & & & & & \\
 Start of Implementation:& & & & & & \\
 End of Implementation:`& & & & & & \\
 Start of Tests:& & & & & & \\
 End of Tests:& & & & & & \\
 Approval of Code:& & & & & & \\
 Integration into Version: & & & & & & \\
 Start of Tests (of new Version):& & & & & & \\
 End of Tests (of new Version):& & & & & & \\
 Approval of Change:& & & & & & \\
\hline\hline
\end{tabular}
\end{center}

\end{document}
