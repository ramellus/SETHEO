\documentstyle[12pt,a4]{article}


\title{The built-in predicate ``functor/3''}
\author{Johann Schumann}
\date{27.3.93}

\begin{document}

\maketitle

\begin{abstract}
This document specifies the behavior and implementation of the built-in
{\bf functor/3}.
Additionally, we describe the built-in {\bf negate/2} which negates
a predicate symbol.

\end{abstract}

\noindent{\bf Keywords:}  built-in, prolog

\noindent{\bf Action: } {\tt EXTENSION}

\section{Solution Provided}

none

\subsection{Problems with Solution Provided}
	
No means for assembly and disassembly of complex terms provided.


\section{Detailed Description}

\subsection{Informal Description}

The behavior of the built-in {\bf functor/3} is similar to that
in PROLOG as described in [Clo81], except, that the SETHEO system
requires all symbols to have a fixed arity.

The built-in is used as follows: {\bf functor(T,F,A)}.

If {\bf T} is instantiated to a term, then {\bf F} is set to the
Functor, and {\bf A} to its arity. A constant always has the arity $0$.
%
If, however, {\bf T} is an unbound variable, a new complex term
with functor {\bf F} and arity {\bf A} is created with new variables
$X_1, \ldots, X_{\bf A}$.


The built-in {\bf negate(I,O)} must be called with a predicate symbol
(= constant) as its first argument. If it is one, {\bf O} is unified
with its negation.

Example: {\tt negate(p,X),write(X)} results in {\tt ~p}.


\section{SAM-Based Specification}

\subsection{Data Structures}

none

\subsection{New SAM instructions}

\subsubsection{The functor-instruction}

A new SAM instruction {\bf functor} is being defined. Its argument
points to a correponding argument vector of length 3.

The pseudo-code of {\bf functor} is as follows:

\begin{verbatim}
functor(){

// if we have a constant: functor = T and arity = 0
if (ISCONST(T)){
	unify(F,T) && unify(0,A);
	return success;
	}
// if we have a complex term (gterm,ngterm,crterm)
if (ISCOMPL(T)){
	funct=GETVAL(*T); // get the index of the functor
	unify(GENOBJ(funct,T_CONST),A) && unify(symbtab[funct].arity,A);
	return success;
	}
// if the first argument is a variable,
// we have to construct a new term
if (ISVAR(T)){
	if (!ISCONST(F)) return failure; // F must be set
	unify(symbtab[GETVAL(F)].arity,A) &&
		// create a new term with variables on the heap,
		// and let the variable T point to it
	unify(T,heapgen(T_CRTERM,GETVAL(F)); 
	}
}
\end{verbatim}


\subsubsection{The negate-instruction}

The pseudo-code for this built-in is as follows. Again, the argument
of this built-in is a pointer to a one-element argument vector.

\begin{verbatim}
negate()
{
int fu;
int unsigned_fu;

if (! ISCONST(ARGPTR(1))){
	return failure;
	}
fu = GETVAL(ARGPTR(1));
// get the unsigned version of the predicate
if (IS_NEG(fu)) fu = CHANGE_SIGN(fu);
else 		unsigned_fu = fu;
if (!symbtab[unsigned_fu].type == predicate) return failure;

// unify the changed predicate with the 2nd argument
unify(GENOBJ(,CHANGE_SIGN(fu),T_CONST),ARGPTR(1)+1) && 
	return success;
}
\end{verbatim}

\subsection{Additional C-functions within SETHEO}
\subsubsection{The function heapgen}

The function {\em heapgen\/} generates a new term with new variables
on the heap, and returns a pointer to it.

\begin{verbatim}
WORD *heapgen(tag,functor)
int tag,functor;
{
WORD *hp_save = hp;
int i;

GENOBJ(hp++,functor,tag);   // create new term-head
for (i=0; i< symbtab[functor].arity;i++){
	GENOBJ(hp++,0,T_FVAR);
	}
return hp_save;
}
\end{verbatim}

The generation of the new variables must be conform to the generation
of constraints.

        


\section{Required Changes}

\subsection{Required Changes in sam}

Files to be modified: conf.c

New files: i\_functor.c, i\_negate.c

\subsection{Required Changes in inwasm}

The instructions {\bf functor, negate} must be added to the set of
built-in predicates.

Furthermore, the user must be allowed to use functors and predicate
symbols (and their negation) additionally as constants. Changes (still
to be specified) must be made in the symbol-table, the scanner and
parser.

\section{Required Changes in wasm}

Must be compiled after makeing setheo.

\section{The Testing}
\subsection{Test-examples}
{\tt Beispiele hier angeben}

\subsection{Results of Tests}
{\tt Resultate: Ausgabe, Zeiten,... angeben}

\section{Proposed Time Schedule}

\begin{center}
\begin{tabular}{|l|r|r|r|}
\hline\hline
Action & inwasm & wasm & sam \\
\hline
Specification & & & \\
Implementation & & & \\
Test & & & \\
Documentation & & & \\
Review & & & \\
\hline
Total & & & \\
\hline\hline
\end{tabular}
\end{center}


\section{Design and Implementation Procedure}

\begin{center}
\begin{tabular}{|l||c|c||c|c||c|c||}
\hline\hline
Action & \multicolumn{2}{|c|}{inwasm} &
\multicolumn{2}{|c|}{wasm} & \multicolumn{2}{|c|}{sam}  \\
\hline
& Name & Date & Name & Date & Name & Date \\
\hline\hline
 Proposed:& & & & & & \\
 Specified:& & & & & & \\
 Approval of Specification:& & & & & & \\
 Start of Implementation:& & & & & & \\
 End of Implementation:`& & & & & & \\
 Start of Tests:& & & & & & \\
 End of Tests:& & & & & & \\
 Approval of Code:& & & & & & \\
 Integration into Version: & & & & & & \\
 Start of Tests (of new Version):& & & & & & \\
 End of Tests (of new Version):& & & & & & \\
 Approval of Change:& & & & & & \\
\hline\hline
\end{tabular}
\end{center}

\end{document}
