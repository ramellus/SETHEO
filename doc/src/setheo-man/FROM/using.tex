%using
\ssection{Overview of the SETHEO System}

The theorem prover SETHEO consists of a number of programs
which can be used as independent stand--alone modules or run together
to get a high performance theorem prover for full first order logic
or to have a fast inference machine to execute LOP.
The programs may be called with inline parameters like UNIX
commands or can be used inside the window--based user interface.
They are:
\begin{description}
\item[plop]
converts a formula in first order predicate logic
with arbitrary quantifiers and connectives into clausal form
which can be processed by the other modules.
\item[preproc]
tries to reduce the formula (or at least its complexity) before
a proof of the formula is attempted. It performs a number of reductions which
hopefully reduces the time, setheo needs to find a proof.
It takes a formula in clausal form and produces another one.
\item[inwasm]
is the (semi--) compiler of the SETHEO--system. It converts
the input formula (LOP program) into assembler code for
the SETHEO Abstract Machine.
\item[wasm]
does the assembly of the code generated by inwasm and
produces huge amounts of ugly hex--digits.
The outcome of wasm can only be understood by SETHEO --- but not
by a human (or android) being.
\item[setheo]
itself does the dirty work: It performs the main proof procesdure
and tries to find a proof
of the formula or executes the LOP--program and eventually
returns a success or a fail -- or nothing -- as it may loop forever.
\item[xtheo]
is the user--friendly version of SETHEO, based
on windows. It runs on SUN workstations only.
So if you don't have a sun on your desk, go out and buy one,
or go down to the beach and watch
the sun--set. {\bf Xtheo} lets forget all you have had to learn
about {\bf inwasm, preproc} etc., just grab your mouse and start
proving theorems...
\end{description}

First we describe how {\bf xtheo} can be used to run setheo and
display the results, after that we dig into the single commands
and their parameters. A manual page can be found in the Appendix
and on-line on your system -- if not, you probably used up
too much disk space and everyone has to fight for each bit.
