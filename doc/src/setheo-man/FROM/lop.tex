%lop
\ssection{Programming in LOP }
SETHEO can also be used for Logic Programming. The programming
language LOP (LOgic Programming) is in some sense similar to PROLOG
[Clo] but has a number of restrictions as well as
powerful and easy to
use enhancements.
How programs can be written in LOP is shown in the following
sections. As a first example a program to calculate the first 10
fibonacci numbers is shown.

\begin{verbatim}
?-nfib(10),write($Fiblist),write("\n").
                /* the \n is a newline  */

nfib(X) :- X > 0, $Fiblist := [], calfib(X,0).
nfib(X) :- write("must_be_greater_zero\007\n").
                /* the \007 is the ASCII-code 7 (Bell)  */

calfib(X,I) :- X <= I, fib(X,F), $Fiblist := [$Fiblist | F],
               I1 is I + 1, calfib(X,I1).
calfib(Y,Z).

fib(0,1).
fib(1,1).
fib(N,F) :- N1 is N-1, N2 is N-2,
            fib(N1,F1), fib(N2,F2), 
            F is F1+F2.
\end{verbatim}

The global variable {\tt\$Fiblist} is used to collect a list of all the
fibonacci numbers. At the end of the calculation, the list is printed.

\ssubsection{Pure LOP}
Pure LOP (i.e. LOP) without built-ins is described in the 
section ''SETHEO as a Theorem Prover`` and ''Syntax of the Input Language``.
 For programming in LOP
all these features can be used as well. Note, however, the use
of procedural built--ins in formulae with clauses which
are fanned and reordered automatically (using the $<$-) is very problematic,
so stay away from this in the beginning.

\ssubsection{Restrictions wrt. PROLOG}
LOP does not support some features of PROLOG which are described in
[Clo].
These are:
\begin{itemize}
\item
The {\bf cut} is not supported.
\item
There is no dynamic entry or deletion of clauses in the internal data--base.
This means that there are no {\bf assert} and {\bf retract} predicates.
However, a similar effect (with a more clear denotational semantics) can
be reached, using global variables.
\item
The I/O predicates of LOP are somewhat restricted. Especially when 
LOP is used to program in NON-Horn formulae, one should be very careful
using I/O predicates, since they have (non--backtrackable) side effects.
\end{itemize}

\ssubsection{Special Data Types}
Two special data types are provided in LOP:
\begin{itemize}
\item Global Variables and
\item Sets.
\end{itemize}

Global variables are denoted like variables but with a starting \$,
e.g.\ \$Xyz is a global variable (\$xyz is garbage only).
They can be used as parameters in subgoals of a clause or in
assignments but never in the head of a clause.
Global variables can be assigned (destructively) a value via
the assignment instruction ({\it :is\/}). This value is valid through
the entire logic program. However, if a backtrack occurs, the old
value, i.e. the value before the last assignment is restored. Thus 
global variables can be used for:
\begin{itemize}
\item
as a type of assert and retract of facts,
\item
global description of the ''world`` in planning problems,
\item
keeping system status etc.\ and
\item
explicit control of clause usage.
\end{itemize}

The following example calculates the minimum and maximum of a list

\begin{verbatim}
?-minmax([1,5,3,0,9,34,23]),write($Min),write($Max).

minmax([]).
minmax([F|X]) :- $Min := F, $Max := F, minsearch(X).

minsearch([]).
minsearch([F|R]) :- setval(F), minsearch(R).

setval(F) :- F > $Max, $Max := F.
setval(F) :- F < $Min, $Min := F.
setval(F).
\end{verbatim}

Another example shows how global variables are backtracked:

\begin{verbatim}
?-exa.

exa :- $X := a, p.

p :- write($X), $X := b, write($X), fail.
p :- write($X).

will produce:
a
b
a
\end{verbatim}


{\it Sets} in SETHEO are a very restricted form of sets:
they may be sets of constants only. This is a severe
restriction, however enabling a very efficient implementation. Sets
have to be initialized before usage. They can be assigned or unified
to variables or global variables like ordinary terms. A number of
built in predicatess allow comparison, selection of members, subset
testing etc. 
Note that the maximal cardinality is restricted to a fixed (but in
the source--code definable)
number.
If a variable is bound to a set and printed with the write statement,
a standard notation with \{ ... \} is used.

\ssubsection{Built--Ins}
In the following a list of all built ins{\footnote{
Note: Not all built-ins have been implemented in Version V2.55 yet.
Those built ins are marked by a $\dag$}} is given. The names
of the built-ins are reserved and must not be used as constants
or simple prediate symbols. Also the negated use of a built--in
is illegal.
\ssubsubsection{Global Assignment {\it :=}}
The global assignment {\it :=\/} is used to assign a term to
a global variable destructively. The left hand side (LHS) of this infix
expression must be a global variable, the RHS may be any term.

{\bf Example:}
\begin{center}
{\tt $X := f(X,g($X))}
\end{center}


\ssubsubsection{Arithmetic Assignment {\it is}}
The infix {\it is\/} is used to assign the evaluated numeric expression
to an unbound variable. If variables are used on the right
hand side of the {\it is\/} they must be bound to a number.
The right hand side is an arithmetic expression containing numbers
and variables with the operators +, -, *, / with the usual preference
and associativity rules. Parenthesis may be used.
Note that there is no unary minus. Use $0 -$ expr instead.

{\bf Example:}
\begin{center}
{\tt X is (3 + X) / Y - 44}
\end{center}

\ssubsubsection{Global arithmetic Assignemt {\it :is}$\dag$}
To assign the result of an arithmetic expression to a global variable,
the {\bf :is} is used. The RHS is exactly as described before.
Note that we have a destructive assignment here.
This is not implemented in 2.55. Use
{\tt X is expr, \$X := X, }
instead.

\ssubsubsection{Relational Arithmetic Operators}
The relational operators for comparing numerical values are
implemented: "$<, \leq\dag$" and "$>, \geq\dag$".
On the RHS and LHS of the operator
an arbitrary numerical expression may be placed which is
evaluated.

\ssubsubsection{Output of an Object {\it write}}
To print an object, only a very primitive built--in predicate is 
provided: {\it write\/} displays its argument on the screen.
The argumant may be an arbitrary term. Variables are handled in
a special way: each variable is given a unique number so that
an unbound variable "Y" may be displayed as {\tt X\_1346}.
Strings which have special characters ,e.g\ newline ,are interpreted.
Special characters start with a $\backslash$.
After that a character or a 3--digit octal number may follow.
The octal number is the ASCII--code of the character to
be printed.
The character which may stand after the $\backslash$ may be one of the following:

\begin{center}
\begin{tabular}{|l|l|}
\hline
$\backslash$b & Backspace\\
$\backslash$t & Tabular\\
$\backslash$n & Newline\\
$\backslash$ddd & ASCII character ddd (octal)\\
$\backslash$x & character x (if not one of the above \\
\hline
\end{tabular}
\end{center}

Note that the '"' may not be used inside a string. (in version V2.55)
The length of a string is limited to 20 characters (V2.55)

%\ssubsubsection{Short Arithmetic Operators {\bf inc, dec}}

\ssubsubsection{Arrays of Global Variables {\it mkglob}$\dag$}
An array of global variables can be allocated during run-time
with a given size. {\it mkglob(n,\$X)\/} allocates space for
$n$ global variables accessible under the name \$X.
$n$ must be a numeric constant. Individual variables can be accessed
using the construct \$X[$i$] to access the $i+1$th variable.
Note that the index starts with $0$ and ends with $n-1$.
Index boundaries are checked and errors reported.

\ssubsubsection{Set Predicates}
A number of built--in predicates deal with sets. Note that
these sets may only contain constants.
In the following list of predicates, some of the arguments
must have special data types, like {\it set\/},{\it number\/}, or
{\it var\/} or must be variables bound to such a data--type.
Otherwise a runtime error is produced and the predicate fails.
The run-time complexity of the predicates is given in parenthesis.

\begin{description}
\item[isempty({\it set\/})]
tests if the set is empty. It fails if the {\it set\/} contains
any elements. (O(1)).
\item[issubset({\it set1\/},{\it set2\/})]
succeeds if {\it set1\/} is a subset of or is equal to {\it set2\/}.
(O(card(MAXSET))).
\item[ismember({\it const\/},{\it set\/})]
This predicate tests, if the constant {\it const\/} is a member
of the set {\it set\/}. Note that the first argument of the
{\it ismember}--predicate MUST be a bound constant. This predicate is not
useable for selecting members from a set. (O(1)).
\item[smember({\it const\_var\/},{\it set\/})]
If the first parameter of {\it ismember\/} is a constant, a test is made,
if this constant is a member of the set {\it set\/}.
If the first argument is a variable, all members of the {\it set\/} are
produced one after the other. {\it Smember\/} is the same as the standard
member definition for lists as shown below. (O(1) -- O(card(set)).
\item[card({\it set\/},{\it numvar\/})]
The cardinality of the set {\it set\/} is returned in {\it numvar\/} 
if it is a variable. If it is a number, the number is checked against
the cardinality and a fail is made if they are not equal.
(O(card({\it set\/}))).
\item[mkempty({\it var\/})]
This predicate always succeeds and generates an empty set which
is bound to the variable {\it var\/}. Normally {\it var\/} will be a
global variable, but normal variables can be used as well.
\item[addtoset({\it const\/},{\it set\/})]
adds an element {\it const\/} to the set {\it set\/}. This predicates
always succeeds. (O(1)).
\item[remfromset({\it const\/},{\it set\/})]
removes the element {\it const\/} from the set {\it set\/}.
It does not test, if the element is in the set {\it set\/}.
(O(1)).
\item[selnth({\it var\/},{\it set\/},{\it number\/})]
is used in combination with the {\it selnext\/} predicate to
select elements contained in a set one after the other.
An example shows how a standard member--predicate may be defined.
The next member of the set {\it set\/} after the index {\it number\/}
is returned in the variable {\it var\/}.
\item[selnext({\it var\/},{\it set\/},{\it number\/},{\it variable\/})]
is also used to define a member--predicate.
This predicate returns the next found member of a set in its first parameter.
Its index is returned in the last parameter.
\end{description}

\begin{verbatim}
% Member definition using the built--in set predicates
% this is equivalent to the smember built--in
member(X,Y) :- isconst(X),ismember(X,Y).
member(X,Y) :- isvar(X),selnext(X,Y,0,I),selmem(X,Y,I).

selmem(X,Y,I) :- selnth(X,Y,I).
selmem(X,Y,I) :- selnext(X,Y,I,I1),selmem(X,Y,I1).

% standard member definition for lists
member(X,[X|R]).
member(X,[Y|R]) :- member(X,R). 
\end{verbatim}

The following example illustrates the usage of the set predicates
{\it smember, selnext,} and {\it selnth}.
A given set \$S is to be printed. This can be done in three ways:
using the {\it write\/} built-in, selecting one element of the set
after the other with the help of backtracking
and print each individual element, or have a recursive loop to
print the elements.

\begin{verbatim}
prtset1 :- write($S).         /* built in printing of set */

                              /* select and fail          */
prtset2 :- smember(X,$S),write(X),write(" "),fail.
prtset2.

prtset3 :- gothru(0).         /* recursive loop           */
gothru(I) :- selnext(X,$S,I,In),selnth(X,$S,In),
             write(X),write(" "),gothru(In).
gothru(_X).
\end{verbatim}

\ssubsubsection{Checking for equality {\it eq, neq}}
These two predicates {\it eq(arg1, arg2)\/}
and {\it neq(arg1, arg2)\/} test if {\it arg1\/} and {\it arg2\/}
are equal or not. This is equivalent to the == and =/= operators
in PROLOG.

\ssubsubsection{Pruning of the search space {\it eqpred\/}}
The predicate {\it eqpred(number)\/} fails, if there exists
a predecessor in the current path in the model elimination tree, 
which is equal (lisp--like {\it eq\/}) to the literal {\it number\/}
in the current clause. This built--in is normally used in theorem--proving
only. Note that this predicate can occur in a clause {\em before\/}
the mentioned literal. 
\ssubsubsection{Artificial Query}
The artificial query {\it query\_\_\/} is added automatically
if the {\it inwasm\/} is invoked with a '-q' option. This is
necessary, if the formula or program contains more than one clause
with negative literals only.
This situation occurs with non-horn clause formulae and when using SETHEO
for theorem--proving.

{\bf Example:}
\begin{verbatim}
FORMULA:                is transformed into:
                        ?-query__
p(a).                   p(a).
...                     ...
<-p(X).                 query__ :- p(X).
<-q(X,a).               query__ :- q(X,a).
\end{verbatim}
\ssubsubsection{Failing: {\it fail\/}}
This predicate causes a fail.
\ssubsubsection{Testing Data Types: {\it isvar, isconst, isnumber\/}}
These predicates test, if their resp. argument is an unbound variable,
a constant or a number. Note that isconst fails with a number given as its parameter.

\ssubsubsection{Calling Predicates: {\it call\/}$\dag$}
The predicate {\it call(sign,expr)\/} starts the execution of a query,
which is described by {\it expr\/}. Sign is the sign of the query.
This predicate can be used only when 
a special option {\it --l\/} has been used when {\it inwasm\/} was invoked.
In the current version (V2.55), this predicate can be used on the assembler--level
only.

\ssubsubsection{Reading Terms: {\it read\/}$\dag$}
Note: This built--in predicate is useable in SETHEO only.\\
The {\it read(var)\/} reads a term from stdin which is terminated
by a '.\@'\ . The read term is put onto the heap and new variable cells are
created for variables in the read term.

\ssubsubsection{Delaying: {\it delay}$\dag$}
The predicate {\it delay(var, predicate)\/}
delays the execution of the predicate {\it predciate\/} until the variable
{\it var\/} has been instantiated. In 2.44 this predicate is not implemented.

\ssubsubsection{Stopping the Setheo: {\it stop\/}}
The {\it stop\/}--predicate stops the SETHEO--machine with
a success.
\ssubsubsection{Entering the Monitor mode: {\it monitor\/}}
NOTE: This predicate can be used with SETHEO only !\\
When a {\it monitor\/}--prodicate is executed, the SETHEO
goes into the monitor mode. A number of commands can be executed
in this mode which is described below.
This predicate always succeedes, if the monitor mode is left with
a continue--command. NOTE: This predicate is normally used for debugging
only.
The following commands are available:
\vspace{5mm}

\begin{tabular}{|l|l|}
\hline
Command & description \\
\hline
\hline
reg  &
displays a number of essential control registers of the SAM.\\
\hline
!  &
escapes to a UNIX shell. On this shell, arbitrary commands can be executed. \\
& With a "exit" a return to the SETHEO monitor mode is accomplished.\\
\hline
stack  &
displays the stack. Beside the absolute address of each cell, \\
&  its value and tag is displayed.\\
\hline
trail  &
displays the trail.\\
\hline
heap  &
displays the heap\\
\hline
cont  &
continues the proof at the current location.\\
\hline
help or ?  &
prints a list of available commands\\
\hline
re  &
starts a reproof to generate a (partial) proof--tree.\\
\hline
temps  &
is used to display the contents of the {\it temp\/} registers.\\
\hline
choice  &
lists in hexadecimal or symbolic form the contents of the current
choice--point.\\
\hline
links  &
The dynamic link--list (list of environments) and the list of choice--points \\
 & (their stack--relative address) is printed.\\
\hline
todo  &
Lists the OR-nodes in the search tree and the clauses which have to be
entered \\
&  when the proof procedure continues.\\
\hline
quit  &
Leaves SETHEO.\\
\hline
\end{tabular}
\vspace{8mm}

Many of the monitor command only work properly, if {\it setheo\/} was
invoked with the -g option. Otherwise only hexadecimal values are displayed.

