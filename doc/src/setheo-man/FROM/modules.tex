%modules
\ssection{Description of the Modules}
\ssubsection{plop}
{\bf Plop} 
takes formulae of first order predicate logic in standard notation
and converts them into LOP clausal sets preserving the property
of unsatisfiability. Obvious tautological clauses and redundant
literals are removed. 
Both, the first-order-language syntax and the LOP syntax is described in
a later section. Plop can be invoked by the following command line or via
the Xtheo interface:
\begin{center}
{\bf plop} [ {\bf --necho} ][ {\bf --nopti} ][
{\bf --sko} ][ {\bf --neg} ][ {\bf --baum]} ][
{\bf --debu} ][ {\bf --defgro} ] {\it file\/}[{\bf .pl1}]
\end{center}
The only parameter which has to be given is the name of the input file.
It must have the extension {\bf .pl1}. The generated output file
has the extension {\bf .lop} which can be handled by preproc or inwasm.

The other parameters are optional and have the following meaning:
\begin{description}
\item[--necho]
suppresses the screen output. 
Otherwise the input- and output formula is displayed.
\item[--nopti]
suppresses the removal of tautologies and redundancies. 
\item[--sko]
yields output clauses not in LOP form but in skolemized
normal form, i.e. clauses containing disjunction
and negation symbols but neither conditional nor conjunction
symbols.
\item[--neg]
negates the input formula before doing the transformation.
(Some examples or programs may be formulated affirmatively.)
\item[--baum]
displays statistics of the internal formula tree.
\item[--debu]
turns on the verbose debugging mode of plop.
\item[--defgro]
displays constraints concerning the size of some elements
of the input formula. If a given example exceeds them, define-statements
in the source code of PLOP are to be modified.
\end{description}

\ssubsection{preproc}
The preprocessing module can be started optionally and tries to reduce
the size and complexity of a given LOP formula, without affecting
its provability. A number of currently 5 reductions are performed in
sequence as long as the formula can be reduced. 
\begin{itemize}
\item Purity--reduction
\item Subsumption--reduction
\item Isolated Literal reduction
\item Sema--Isolated Literal reduction, and
\item Restricted Unit Resolution.
\end{itemize}
For details of the reduction steps see [LSBB89,BLS88].
The command line syntax of preproc is:
\begin{center}
{\bf preproc}
[{\bf --c}]
[{\bf --b}]
[{\bf --d} {\it number}]
{\it file}
\end{center}

The input file {\it file\/} must have the extension {\bf .lop}, the output file,
which is generated has the name {\bf p\_}{\it infile}{\bf .lop}.
This is necessary as input-- and output--files must have different
names.
In some cases, the formula can be proved by the preprocessor already.
In that case, no output file is generated, instead a message is displayed
on the screen, and the program returns an appropriate exit code.

All parameters of {\bf preproc} except the input--file name are optional:
\begin{description}
\item[--d {\it nu}]
A debugging level of {\it nu\/} is set, causing the program to generate
a lot of intermediate output. The default value is 10.
\item[--c]
If a formula, which contains no query (e.g.\ a program without a request)
has to be processed, that option has to be given. In that case, no
purity reduction is performed (which would possibly eliminate all clauses).
\item[--b]
If this option is set, the isolated reductions may generate bigger clauses.
\end{description}

Note, that in the current version, no built--ins and global variables are
allowed.


\ssubsection{inwasm}
The SETHEO compiler {\bf inwasm} takes a formula or program
written in LOP and compiles it into SETHEO assembler code. 
The resulting 
code can be assembled by {\bf wasm} and executed by {\bf setheo}.
The input file must have the extension {\bf .lop\/}, the file genererated
has the extension {\bf .s\/}.
If the {\bf --t} option is selected, an additional file with the extension
{\bf .symb} is generated, holding symbol table information needed by the
{\bf xtheo} graphic interface to display the proof tree.

Also a fast version of the purity--reduction is performed.
The additional run of that reduction (cf. {\bf preproc}) does not increase
the compile--time considerably, but is very convinient esp.\ for logic
programming.
When using SETHEO for logic programming, normally no option is needed
when invoking {\bf inwasm}.
A number of options must be given to control code generation for theorem
proving purposes.
The following line gives a list of all available parameters and a short
description of each. Details will be given in the section: SETHEO as a
Theorem Prover.

\begin{center}
{\bf inwasm}
[ {\bf --v}]
[ {\bf --w}]
[ {\bf --a}]
[ {\bf --f}]
[ {\bf --q}]
[ {\bf --t}]
[ {\bf --T}]
[ {\bf --p}]
[ {\bf --c} [{\it nu\/}]] 
[ {\bf --P} [{\it nu\/}]] \\ \,
[ {\bf --o}]
[ {\bf --R}]
[ {\bf --r}]
[ {\bf --C}]
[ {\bf --i}]
[ {\bf --I}]
[ {\bf --h{\bf [}s{\bf ][}f{\bf ][}y{\bf ]}}]
[ {\bf --O}{\it file\/}]
{\it file\/}
\end{center}

Here we only describe a number of essential options, the entire list of
options can be found in the manual pages.
\begin{description}
\item[--a]
Find all solutions for a given query. If the theorem prover is started
in the DFS mode or the --r mode (iterative deepening), all solutions
are searched, otherwise only the solutions within the given 
limits are searched. If solution substitutions have to be displayed,
{\it write\/} statements must be added manually to the LOP--formula.
\item[--c {\it nu\/}]
A global copy-bound of {\it nu\/} is set to all clauses with more than
one literal, i.e.\ a maximum of {\it nu\/} copies of each clause may be
used with the proof. If, however, the limits must be set to different values
for different clauses, changes in the output file of the inwasm have to
be made.(experts only)
\item[--C] suppress code generation.
\item[--i] generates code to allow for a counting of all inferences
made during the proof. An inference is a successful unification of two
literals or a successful reduction step.
\item[--I] is similar to the {\bf --i} option, but counts the {\em attempted\/}
unifications and successful reductions.
\item[--o]
If this option is selected, the weak unification graph is generated
by an algorithm, {\it without\/} occurs check.
\item[--p]
This option inserts code to perform the equal predecessor fail pruning.
\item[--q] If a LOP formula has more than one query, i.e.\ more than one
clause with negative literals only, or if it is a non--horn formula,
the --q option is needed to add an artificial query, which allows setheo
to work properly. Note that with the --q option one additional inference
and one additional depth level is needed.
Details are described in the section "Built--ins".
\item[--r] Code for performing reduction steps is added.
This option is needed for most non--horn formulae to guarantee
completeness.
\item[--T] Code for generating a proof tree is added to the code.
(This is a default parameter when using the xtheo user interface)
Also the list of literals and additional information is written
into {\it file.symb\/} in order to be read by {\it xtheo\_graphics}.
\item[--v] turns on the verbose mode.
\item[--R] turns off the reordering of the subgoals in non-horn
formulae.
\item[--hs] turns on static reordering or OR-branches consisting of
at least 2 clauses with more than one literal.
\end{description}

{\bf Note:} Please ensure that either the local directory or your home
directory contains a copy of the file  \verb-.inwasmrc- which contains
definitions of predefined built-ins and can be found in the subdirectory
containing the inwasm sources.






\ssubsection{wasm}
The SETHEO assembler {\bf wasm} takes files produced by {\bf inwasm}
with the extension {\bf .s} and converts them into hexadecimally coded
binary files. Also libraries may be included at this time. Symbolic labels
are resolved. If the assembler is invoked with the optimize option {\bf -O}
the connection graph will be stored in a compressed format.
Details of the {\bf wasm} are only interesting for the very experienced
user who wishes to program in SETHEO assembler language. 
The input file must have the extension {\bf .s\/}, the output file
the extension {\bf .hex\/}.

{\bf Wasm} has the following options:
(all except the {\it file\/} are optional)

\begin{center}
{\bf wasm}
[{\bf --v}]
[{\bf --O}]
[{\bf --l}]
[{\bf --lx}]
[{\bf --d}]
[{\bf --o\/}{\it file\/}]
{\it file\/}
\end{center}
\begin{description}
\item[-v]
produce a verbose output (for debugging purposes only)
\item[-O]
invoke the optimizer to compactify the connection graph.
\item[-l [x ]]
give a listing of all labels and their addresses. This option only
works in conjunction with the {\bf -v} option.
\item[-d]
generates debugging information which can be used in the monitor
mode of {\bf setheo}. It allows to display clause and literal numbers
instead of pure addresses.
\item[-o file]
uses the {\bf file} as an output instead of the default {\bf infile.hex}.
i\end{description}

\ssubsection{setheo}
{\bf Setheo} is the kernel of the {\small SETHEO} system. It is realised
as an abstract machine
based on Model Elimination, a specialisation of the Connection Method.
Setheo executes the instructions generated by the inwasm and wasm commands
in a file with the extension {\bf .hex}.
If a proof of the formula or LOP program could be found within given limits,
a success message is displayed, together with the number of
nferences used in the proof. If no proof could be found,
the reason is displayed (reaching a certain boundary or total fail).
Additionally the total number of abstract instructions and the
elapsed time is displayed. Note
that due to the incorrectness of the internal clock (ticks every 60 ms)
the accurracy of that time is not very high. 

If an internal error occurs, the monitor is entered (see description
of the {\bf monitor} built--in predicate.
Hopefully, only a "Stack overflow" has occurred, which can be overcome
by using the {\bf --S} option or setting a depth- or inference--bound.
Setheo has quite a lot of parameters, all except the input file being optional.
If setheo is started without parameters, the abstract machine runs in
a depth--first left--to--right mode with an almost unlimited depth bound
(9999).

When called from the shell, setheo has the following synopsis:
\begin{center}
{\bf setheo}
[{\bf --t}]
[{\bf --g}]
[{\bf --r\/}[{\it number\/}]]
[{\bf --l\/}[{\it number\/}]]
[{\bf --d} {\it number\/}] \\

[{\bf --i} {\it number\/}]
[{\bf --S} {\it number\/}]
[{\bf --H} {\it number\/}]
[{\bf --B} {\it number\/}]
{\it file}
\end{center}

\begin{description}
\item[{\it file\/}]
The input file must have the extension {\bf .hex}.
\item[--t]
If the inwasm was invoked with {\bf --t} and this option is set,
a proof tree will be generated and saved the file {\it file\/}.{\bf tree}.
This tree can be displayed graphically by the xtheo--user interface.
\item[--g] This option loads debugging information, which is used
in the monitor mode of setheo. For details see the description of
the built--in predicate {\bf monitor}.
\item[--d {\it nu\/}]
Sets the depth bound to the number {\it nu\/}. Due to historical reasons,
the number given must be 1 greater than the actual maximal depth.
\item[--i {\it nu\/}]
Sets the inference bound to the number {\it nu\/}. Due to historical reasons,
the number given must be 1 greater than the actual maximal number of inference.
\item[--r [{\it nu\/}]]
With this option, setheo is started in a dfs--iterative deepening mode,
i.e.\ starting with a depth of 3 (or {\it nu\/} if given), setheo
tries to find a proof. If a fail occurs, the depth limit is incremented
by 1.
\item[--l [{\it nu\/}]]
allows the generation and usage of lemmata. In the current version
this is not supported by the inwasm--compiler, so setheo-assembler files
have to be modified. (for expert users only)
\item[--S {\it nu\/}]
set the size of the stack space to {\it nu\/} bytes. (default: 30000)
\item[--H {\it nu\/}]
set the size of the heap space to {\it nu\/} bytes. (default: 30000)
\item[--B {\it nu\/}]
set the size of the trail space to {\it nu\/} bytes. (default: 30000)
\end{description}
