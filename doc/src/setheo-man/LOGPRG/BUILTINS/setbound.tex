%%%%%%%%%%%%%%%%%%%%%%%%%%%%%%%%%%%%%%%%%%%%%%%%%%%
%   SETHEO MANUAL
%	(c) J. Schumann, O. Ibens
%	TU Muenchen 1995
%
%	%W% %G%
%
% BUILT-IN:	
% SCCS:		%W% %G%
% AUTHOR:	J. Schumann
%
%%%%%%%%%%%%%%%%%%%%%%%%%%%%%%%%%%%%%%%%%%%%%%%%%%%
\subsection{\$set{\em bound\/}/1}

This section describes a number of built-ins which allow to destructively 
set the current (or maximal) values of the search bounds.
All assignments to the bounds are backtrackable, i.e., their effect
is undone as soon backtracking occurs over the corresponding built-in.
\begin{description}
\item[Synopsis:]
	{\tt \$set{\em bound}(N)}
\item[Parameter:]\ \\[-0.5cm]
	\begin{description}
	\item[{\tt N}] term which must evaluate to a non-negative number.
	\end{description}
\item[Low Level Name:]
	{\tt set\_depth, set\_inf, set\_maxinf, set\_locinf, set\_maxfvars,
	     set\_maxtc, set\_maxsgs}
\item[Result:]\ \\
These built-ins always succeed.
\end{description}

\vspace*{0.5cm}
\noindent
{\bf Description.}
A built-in of this group sets the current or maximal value
of a search-bound.
\begin{description}
\item[{\tt \$setdepth(N)}]
            sets the bound to the maximum tableau depth
            (allowed for a final tableau) minus current depth.
\item[{\tt \$setinf(N)}]
            sets the number
            of inferences.
\item[{\tt \$setlocinf(N)}]
            sets the number
            of local inferences.
\item[{\tt \$setmaxinf(N)}]
            sets the maximum number
            of inferences (allowed for a final tableau).
\item[{\tt \$setmaxfvars(N)}]
            sets the maximum number
            of free variables (allowed for a final tableau).
\item[{\tt \$setmaxtc(N)}]
            sets the maximum termcomplexity
            (allowed for a final tableau).
\item[{\tt \$setmaxsgs(N)}]
            sets the maximum number
            of open subgoals (allowed for a final tableau).
\end{description}

\vspace*{0.5cm}
\noindent
{\bf Side-effects.}
These built-in effect global registers of the SETHEO Abstract Machine
\SAM.
Therefore, a wrong usage of these built-ins can have undesired effects
and can destroy the completeness of the proof procedure.

\vspace*{0.5cm}
\noindent
{\bf Note.}
These built-ins are used to adapt the iterative deepening, or to implement
Meta-interpreters. For details and examples refer to Section~\ref{chap:How-to}.

\vspace*{0.5cm}
\noindent
{\bf Example.}
\begin{verbatim}
p(X) :- $getdepth(D),D1 is D+1,$setdepth(D1),q(X),$setdepth(D).
\end{verbatim}

When this clause is called, then its subgoal {\tt q} can use  the same
resources (here, the depth), as the clause itsself.
