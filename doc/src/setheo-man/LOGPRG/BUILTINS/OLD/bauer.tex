Sehr geehrter Herr Prof.~Bauer,

leider hat es nun doch l"anger gedauert, bis ich
Ihnen meinen alten Computer f"ur die Sammlung des
Deutschen Museums vorbeibringen konnte.

Hier noch einige Daten zu dem System:

gekauft: Mai 1979 und August 1979 (Expansion Unit)

Die Zentraleinheit (Tastatur) kam mit 12 KB ROM und 4 KB RAM.
Dazu der Monitor und der Kassettenrecorder.
Preis: ca. DM 2100.-

Die Erweiterungseinheit (ohne RAM) kostete damals ca. DM 900.-

Folgende Hardware-"anderungen wurden an dem System durchgef"uhrt:
\begin{description}
\item[Monitor:]
Einbau eines st"arkeren Netztransformators
\item[Zentraleinheit:]
\begin{itemize}
	\item
	Zus"atzlicher Bildschirmspeicher (RAM + EPROM) zum Anzeigen auch von
	Kleinbuchstaben. Dazu ein Kippschalter an der R"uckwand.
	\item
	DSUB-9 Stecker an der R"uckwand. Dieser Stecker ist mit der
	Tastatur verbunden und wurde zum Anschluss eines Joy-Sticks
	verwendet.
	\item
	Aufr"ustung auf 16KB RAM
	\item
	Der Stecker f"ur das Kabel zur Erweiterungseinheit (urspr"unglich
	direkter Stecker) wurde durch einen Pfostenstecker ersetzt, da
	der alte defekt war. (Ebenso auf der Seite der Erweiterungseinheit).
	\item
	Einbau eines gr"o"seren K"uhlk"orpers f"ur den Spannungsregler
	(5V).
\end{itemize}
\item[Zentraleinheit:]
\begin{itemize}
	\item
	Einbau eines Floppy-disk-controllers f"ur doppelte
	Schreibdichte (180KB/Floppy-seite)
	\item
	Aufr"ustung auf 32KB RAM
	\item
	zus"atzliche bidirektionale Schnittstelle zum Anschluss
	eines CPM-Selbstbaurechners. Auch hier wurden die direkten
	Steckverbindungen durch Pfostenstecker ersetzt.
\end{itemize}
\end{description}

Das System ist -- bis auf den Kassettenrecorder -- betriebsbereit.
Ich habe neben der Benutzerdokumentation auch noch das Hardware-manual
(mu"ste extra gekauft werden), sowie noch einige Demo-programme auf
Kassette beigelegt.

Ich hoffe, da"s dieses System Ihre sehr sch"one und aufschlu"sreiche
Sammlung (zumindenstens im Archiv) bereichern kann.


