%%%%%%%%%%%%%%%%%%%%%%%%%%%%%%%%%%%%%%%%%%%%%%%%%%%
%   SETHEO MANUAL
%	(c) J. Schumann, O. Ibens
%	TU Muenchen 1995
%
%	%W% %G%
%
% BUILT-IN:	
% SCCS:		%W% %G%
% AUTHOR:	J. Schumann
%
%%%%%%%%%%%%%%%%%%%%%%%%%%%%%%%%%%%%%%%%%%%%%%%%%%%
\subsection{\$tell/1}

\begin{description}
\item[Synopsis:]
	{\tt \$tell(T)}
\item[Parameters:]\ \\[-0.5cm]
	\begin{description}
	\item[{\tt T}] term or string
	\end{description}
\item[Low Level Name:]
	{\tt tell}
\item[Result:]\ \\
\end{description}

\vspace*{0.5cm}
\noindent
{\bf Description.}
When this built-in is encountered, the \SAM\ tries to open a file
with the given name in the ``append mode'' ("+a"). 
Subsequent outputs (e.g., by
{\tt \$write} or {\tt \$dumplemma}) will be written into that file.
If a named file has been open previously, that file is closed prior
to opening the new one.

In contrast to PROLOG, no nested {\tt \$tell}'s are allowed.
\vspace*{0.5cm}
\noindent
{\bf Side-effects.}

\vspace*{0.5cm}
\noindent
{\bf Notes.}
Since strings and terms are treated likewise, {\tt \$tell(foo)} and
{\tt \$tell("foo")} have the same effect.

\vspace*{0.5cm}
\noindent
{\bf Example.}
\begin{verbatim}
p(X) <- $tell("output"),$write(X),$told.
\end{verbatim}


