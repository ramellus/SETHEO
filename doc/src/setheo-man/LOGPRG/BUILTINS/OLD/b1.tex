\subsection{genulemma/3}
{\tt genulemma(N1,N2,N3)}

\begin{description}
	\item[{\tt N1}]
	This number indicates the number of the literal for which
	a unit-lemma is to be generated. {\tt N1} must be in the
	range between $1$ and the length of the clause (including
	built-in literals).
	\item[{\tt N2}]
	If this number is $\neq 0$, the generation of a unit-lemma
	is considered. Otherwise, this predicate just succeeds.
	\item[{\tt N3}]
	If a unit-lemma is stored in the index, this number will be
	stored together with the lemma. This information can the
	be used when unit-lemmata are being used (e.g., by {\bf uselemma/1}).
	\end{description}
\item[Low Level Name:]
	{\tt genulemma}
\item[Result:]\ \\
	This built-in predicate always succeeds. Memory overflow and
	wronly instantiated parameters will result in a 
	non-fatal run-time error.
\end{description}

%----------------------------------------------------
This built-in generates a unit-lemma {\tt L<-.} for a literal
{\tt ~L} in the current clause. Its number is given as the first argument.
The lemma is generated, if all of the following
conditions hold.
The lemma will be annotated by the value of the third argument
(must be a number).

\begin{itemize}
\item
all arguments must be instantitated to numbers,
otherwise, a run-time error is generated.
\item
the first argument must be instantiated to a number
in the range between $1$ and the length of the clause (including
built-in literals).
\item
the value of the second argument is $\neq 0$,
\item
the current instantiation of the given literal is not subsumed
by any unit-lemma in the lemma store.
\end{itemize}

If the unit-lemma subsumes one or more unit-lemmata in the lemma-store,
these unit-lemmata are deleted.

%----------------------------------------------------
Notes:
\begin{itemize}
\item
the range of the first argument is NOT checked
\item
it is not checked, if the solution of the given literal required
reduction steps above the current node in the tableau.
Therefore, it is easily possible to generate unsound proofs, if
yet unsolved literals are kept as lemmata.
\item
all built-in literals (including this one) count as literals
\end{itemize}


%----------------------------------------------------

\begin{verbatim}
q(X,Y) <- p(X),genulemma(2,1,5),r(X,Y).
\end{verbatim}

generates a lemma {\tt p(a)<-.}, if X is instantiated to {\tt a}.
The lemma carries the number $5$.

%%%%%%%%%%%%%%%%%%%%%%%%%%%%%%%%%%%%%%%%%%%%%%%%%%%%

\subsection{uselemma/1}

{\tt uselemma(N)}


during run-time, only lemmata from the index may be used which carry
a number which is smaller than the given parameter.

%----------------------------------------------------

This built-in always succeeds. If the pushed lemmata and the remaining
alternatives are to be tried, this built-in must be directly followed
by a {\bf fail}.
%----------------------------------------------------
This built-in tries to use lemmata from the lemma-store.
This built-in must be used within a clause of a specific structure
as shown below.
When called, this built-in extracts all lemmata from the index
and marks them as possible choices, which fulfill the following conditions:
\begin{enumerate}
\item
the lemma in the index must not be marked ``deleted''.
\item
the lemma must be unifiable with the head of the current clause.
\item
the number stored with the lemma must be smaller ($<$) than the
given parameter.
\end{description}.

After adding the choices, this built-in succeeds, If the
selected lemmata are to be tried, a fail must be executed
after this built-in.
Then, the lemmata are tried and then
all other or-branches for
the current subgoal which have not yet been touched are tried.


%----------------------------------------------------

The following example illustrates the usage of this built-in.
\begin{verbatim}
...
p(a,b) <-..
p(b,c) <-..

p(_,_) <- uselemma(5),fail.

p(e,f) <-.
..
\end{verbatim}

If the clauses are not reordered, SETHEO first tries the alternatives
{\tt p(a,b)..} and {\tt p(b,c)..}.
Then it extracts all suitable lemmata from the lemma store.

These lemmata are pushed upon the stack as additional alternatives
which are tried (caused by the fail), before {\tt p(e,f)..} are tried.
%%%%%%%%%%%%%%%%%%%%%%%%%%%%%%%%%%%%%%%%%%%%%%%%%%%%


\subsection{getnrlemma/1}
{\tt getnrlemma(N)}

this parameter unifies with the number of lemmata currently in the
lemma index.
%----------------------------------------------------
This built-in fails, if the unification fails.
%----------------------------------------------------
This built-in unifies the given argument with the number
of lemmata which are currently stored (and not deleted) in the
lemma store.

%----------------------------------------------------
\begin{verbatim}
p(X) <- getnrlemmata(Y), X > Y.
\end{verbatim}
%%%%%%%%%%%%%%%%%%%%%%%%%%%%%%%%%%%%%%%%%%%%%%%%%%%%

\subsection{dumplemma/0}
{\tt dumplemma}

%----------------------------------------------------
This built-in always succeeds.
%----------------------------------------------------
This built-in dumps all lemmata in the lemma-store
onto the current output-file (as set by {\bf tell/1}), or on
stdout.
Lemmata are printed in a LOP-like syntax. Therefore, the output can directly
be processed by {\bf inwasm(1)}.
Lemmata which have been marked deleted are preceeded by a {\bf \#}.
This option is present for debugging and testing purposes and may
be removed in future versions.

%----------------------------------------------------
\begin{verbatim}
printlemma <- tell("out"),write("Lemmata:\n"),
              dumplemma,told.
\end{verbatim}

%%%%%%%%%%%%%%%%%%%%%%%%%%%%%%%%%%%%%%%%%%%%%%%%%%%%
\subsection{dlrange/2}

{\tt dlrange(N1,N2)}

N1: lower bound of the range to print
N2: upper bound of the range to print
%----------------------------------------------------
This built-in succeeds, if both parameters are
instantiated to numbers.
%----------------------------------------------------

This built-in dumps all lemmata in the lemma-store
with annotated values between the first and the second parameter
(inclusive; both parameters must be instantiated to numbers)
onto the current output-file (as set by {\bf tell/1}), or on
stdout.
Lemmata are printed in a LOP-like syntax. Therefore, the output can directly
be processed by {\bf inwasm(1)}.

%----------------------------------------------------
Example:
\begin{verbatim}
printlemma <- tell("out"),write("Lemmata:\n"),
              dlrange(3,4),told.
\end{verbatim}
%%%%%%%%%%%%%%%%%%%%%%%%%%%%%%%%%%%%%%%%%%%%%%%%%%%%

\subsection{delrange/2}
{\tt delrange(N1,N2)}
N1: lower bound of the range to print
N2: upper bound of the range to print
%----------------------------------------------------
This built-in succeeds, if both parameters are
instantiated to numbers.
%----------------------------------------------------
This built-in deletes all lemmata in the lemma store
with annotated values between the first and the second parameter
(inclusive).
Both parameters must be instantiated to a number.

