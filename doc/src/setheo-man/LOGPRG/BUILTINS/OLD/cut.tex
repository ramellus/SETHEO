%%%%%%%%%%%%%%%%%%%%%%%%%%%%%%%%%%%%%%%%%%%%%%%%%%%
%   SETHEO MANUAL
%	(c) J. Schumann, O. Ibens
%	TU Muenchen 1995
%
%	%W% %G%
%
% BUILT-IN:	
% SCCS:		%W% %G%
% AUTHOR:	J. Schumann
%
%%%%%%%%%%%%%%%%%%%%%%%%%%%%%%%%%%%%%%%%%%%%%%%%%%%
\subsection{\$cut/0,\$precut/0}

\begin{description}
\item[Synopsis:]
	{\tt \$precut, ..., \$cut}
\item[Parameters:]\ \\[-0.5cm]
\item[Low Level Name:]
	{\tt precut, cut}
\item[Result:]\ \\
\end{description}

\vspace*{0.5cm}
\noindent
{\bf Description.}
These two built-ins implement the PROLOG-cut ``!''.
The Prolog clause
 
\begin{verbatim}
        p(X) :- q(X),
                !,
                r(X).
\end{verbatim}
% 
is operationally equivalent to the LOP clause
% 
\begin{verbatim}
        p(X) :- $precut,
                q(X),
                $cut,
                r(X).
\end{verbatim}
 
The {\tt \$precut} always must be the {\em first\/} tail-literal of a clause.


\vspace*{0.5cm}
\noindent
{\bf Side-effects.}

\vspace*{0.5cm}
\noindent
{\bf Notes.}
The Prolog Clause
 
\begin{verbatim}
        p(X) :- not q(X),
                r(X).
\end{verbatim}
 
             is operationally equivalent to the following sequence of
             LOP clauses
 
\begin{verbatim}
        p(X) :- not_q(X),
                r(X).
 
        not_q(X) :- $precut,q(X),
                 $cut, $fail.
 
        q(X).
\end{verbatim}
 


\vspace*{0.5cm}
\noindent
{\bf Example.}
\begin{verbatim}
xxx
\end{verbatim}


