%%%%%%%%%%%%%%%%%%%%%%%%%%%%%%%%%%%%%%%%%%%%%%%%%%%
%   SETHEO MANUAL
%	(c) J. Schumann, O. Ibens
%	TU Muenchen 1995
%
%	%W% %G%
%
% BUILT-IN:	
% SCCS:		%W% %G%
% AUTHOR:	J. Schumann
%
%%%%%%%%%%%%%%%%%%%%%%%%%%%%%%%%%%%%%%%%%%%%%%%%%%%
\subsection{\$write/1}

\begin{description}
\item[Synopsis:]
	{\tt \$write(T)}
\item[Parameters:]\ \\[-0.5cm]
	\begin{description}
	\item[{\tt T}] term to be printed
	\end{description}
\item[Low Level Name:]
	{\tt out}
\item[Result:]\ \\
This built-in always succeeds
\end{description}

\vspace*{0.5cm}
\noindent
{\bf Description.}
The printable representation of the given term is printed.
Logical variables are printed in the form {\tt X\_}{\em number}, where
{\em number\/} identifies the variable in a unique way.
Please note that these numbers may differ from run to run.

Quoted special characters (e.g., {\verb+ \n+}) in strings are expanded.
\vspace*{0.5cm}
\noindent
{\bf Side-effects.}

\vspace*{0.5cm}
\noindent
{\bf Notes.}

\vspace*{0.5cm}
\noindent
{\bf Example.}
\begin{verbatim}
p(X) <- $write("X="), $write(X), $write("\n\007").
\end{verbatim}

When called with {\tt <- p(f(a))} this clause will print
{\tt X=f(a)}, a new-line, and will ring the bell.

