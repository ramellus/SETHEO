%%%%%%%%%%%%%%%%%%%%%%%%%%%%%%%%%%%%%%%%%%%%%%%%%%%
%   SETHEO MANUAL
%	(c) J. Schumann, O. Ibens
%	TU Muenchen 1995
%
%	%W% %G%
%
% BUILT-IN:	
% SCCS:		%W% %G%
% AUTHOR:	J. Schumann
%
%%%%%%%%%%%%%%%%%%%%%%%%%%%%%%%%%%%%%%%%%%%%%%%%%%%
\subsection{unify/2}

\begin{description}
\item[Synopsis:]
	{\tt unify(T1,T2)}
\item[Parameters:]\ \\[-0.5cm]
	\begin{description}
	\item[{\tt T1,T2}]
two terms to be unified.
	\end{description}
\item[Low Level Name:]
	{\tt unify}
\item[Result:]\ \\
This built-in succeeds, if both terms are unifiable.
\end{description}

\vspace*{0.5cm}
\noindent
{\bf Description.}

\vspace*{0.5cm}
\noindent
{\bf Side-effects.}

As a side-effect, the substiutions of the unification are applied to
the terms {\tt T1, T2}.
 
\vspace*{0.5cm}
\noindent
{\bf Notes.}

If only a check is required, if two terms are unifiable, {\tt isunifiable/2}
should be used.



\vspace*{0.5cm}
\noindent
{\bf Example.}

\begin{verbatim}
p(X) <- unify(X,f(g(a,b),c)).
\end{verbatim}
 
produces the same result as
\begin{verbatim}
p(f(g(a,b),c)<-.
\end{verbatim}
 

