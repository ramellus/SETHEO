%%%%%%%%%%%%%%%%%%%%%%%%%%%%%%%%%%%%%%%%%%%%%%%%%%%
%   SETHEO MANUAL
%	(c) J. Schumann, O. Ibens
%	TU Muenchen 1995
%
%	%W% %G%
%
% BUILT-IN:	genlemma/2
% SCCS:		%W% %G%
% AUTHOR:	J. Schumann
%
%%%%%%%%%%%%%%%%%%%%%%%%%%%%%%%%%%%%%%%%%%%%%%%%%%%
\subsection{genlemma/2}

\begin{description}
\item[Synopsis:]
	{\tt genlemma(N1,N2)}
\item[Parameters:]\ \\[-0.5cm]
	\begin{description}
	\item[{\tt N1}]
	If this number is $\neq 0$, the generation of a unit-lemma
	is considered. Otherwise, this predicate just succeeds.
	\item[{\tt N2}]
	If a unit-lemma is stored in the index, this number will be
	stored together with the lemma. This information can then
	be used when unit-lemmata are being used (e.g., by {\bf uselemma/1}).
	\end{description}
\item[Low Level Name:]
	{\tt genlemma}
\item[Result:]\ \\
	This built-in predicate always succeeds. Memory overflow and
	wronly instantiated parameters will result in a 
	non-fatal run-time error.
\end{description}

\vspace*{0.5cm}
\noindent
{\bf Description.}
This built-in generates a unit-lemma {\tt H<-.} for the
head {\tt H} of the current clause, if all of the following
conditions hold:
\begin{itemize}
\item
both arguments are instantitated to a number,
\item
the value of the first argument is $\neq 0$,
\item
The solution of the head of the current subgoal does not involve
any reduction steps above the current node in the tableau.
(This means, that we really have a unit-lemma).
\item
The current head (with its current instantiation) is not subsumed
by any unit-lemma in the lemma store.
\end{itemize}

The lemma will be annotated by the value of the second argument
(must be a number).

If the unit-lemma subsumes one or more unit-lemmata in the lemma-store,
these unit-lemmata are deleted.

\vspace*{0.5cm}
\noindent
{\bf Side-effects.}
If the {\bf sam(1)} is called with the inline-command flag {\tt -lemmatree},
the proof tree for the current lemma (if it gets stored in the index)
is appended to the proof-tree file.

\vspace*{0.5cm}
\noindent
{\bf Notes.}
The {\bf genlemma/1} built-in must be the last subgoal of a given
clause or contrapositive.
Otherwise, correctness of SETHEO is not ensured, if the lemmata are
used.

\vspace*{0.5cm}
\noindent
{\bf Example~1.}
\begin{verbatim}
q(X,Y) <- p(X),r(X,a),genlemma(1,5).
\end{verbatim}

In this example, a unitlemma {\tt q(X,Y)<-.} --- with the current substitutions of {\tt X} and {\tt Y} --- is generated and labeled with the number $5$.

\vspace*{0.5cm}
\noindent
{\bf Example~2.}
\begin{verbatim}
q(X,Y) <- p(X),r(X,a),getdepth(D),genlemma(1,D).
\end{verbatim}

In this example, the generated unit-lemma is labeled with the current
depth. When lemmata are selected for usage ({\bf uselemma/1}), one
can select only those unit-lemmata which have been generated on a lower
depth.
