%%%%%%%%%%%%%%%%%%%%%%%%%%%%%%%%%%%%%%%%%%%%%%%%%%%
%   SETHEO MANUAL
%	(c) J. Schumann, O. Ibens
%	TU Muenchen 1995
%
%	%W% %G%
%
% BUILT-IN:	
% SCCS:		%W% %G%
% AUTHOR:	J. Schumann
%
%%%%%%%%%%%%%%%%%%%%%%%%%%%%%%%%%%%%%%%%%%%%%%%%%%%
\subsection{is}

\begin{description}
\item[Synopsis:]
	{\tt V is NUMEXPR}
\item[Parameters:]\ \\[-0.5cm]
	\begin{description}
	\item[{\tt V}]
(logical) variable
	\item[{\tt NUMEXPR}]
term or numerical expression which must evaluate to a number
        when this statement is encountered.
 

	\end{description}
\item[Low Level Name:]
	{\tt sto}
\item[Result:]\ \\
This statement succeeds, if {\tt V} is an unbound variable or
if {\tt V} is instantiated to a number which is equal to the value
of the numerical expression {\tt NUMEXPR}.
Otherwise a fail occurrs.
If {\tt NUMEXPR} does not evaluate to a number, a non-fatal run-time
error occurrs.
\end{description}

\vspace*{0.5cm}
\noindent
{\bf Description.}

\vspace*{0.5cm}
\noindent
{\bf Side-effects.}

\vspace*{0.5cm}
\noindent
{\bf Notes.}
In contrast to global variables in some Prolog Systems
         like Sepia or Eclipse, destructive assignments to
         global variables in LOP are backtracked if the proof
         search fails at later subgoals.





\vspace*{0.5cm}
\noindent
{\bf Example.}
\begin{verbatim}
p(X,Y) <- X is Y+1.
\end{verbatim}


