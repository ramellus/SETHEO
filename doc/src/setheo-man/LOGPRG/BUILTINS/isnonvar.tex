%%%%%%%%%%%%%%%%%%%%%%%%%%%%%%%%%%%%%%%%%%%%%%%%%%%
%   SETHEO MANUAL
%	(c) J. Schumann, O. Ibens
%	TU Muenchen 1995
%
%	%W% %G%
%
% BUILT-IN:	
% SCCS:		%W% %G%
% AUTHOR:	J. Schumann
%
%%%%%%%%%%%%%%%%%%%%%%%%%%%%%%%%%%%%%%%%%%%%%%%%%%%
\subsection{\$isnonvar/1}

\begin{description}
\item[Synopsis:]
	{\tt \$isnonvar(T)}
\item[Parameter:]\ \\[-0.5cm]
	\begin{description}
	\item[{\tt T}] term
	\end{description}
\item[Low Level Name:]
	{\tt isnon\_var}
\end{description}

\vspace*{0.5cm}
\noindent
{\bf Description.}
This built-in predicate succeeds, if {\tt T} is
{\em not\/} a variable or bound to a variable.

\vspace*{0.5cm}
\noindent
{\bf Example.}
\begin{verbatim}
p(X) <- $isnonvar(X),...
p(X) <- $isvar(X),...
\end{verbatim}
The built-in predicates {\tt \$isvar} and {\tt \$isnonvar} can be used
to distinguish clauses which can be entered, if a variable is bound
(here: first clause), or not (second clause).



