%%%%%%%%%%%%%%%%%%%%%%%%%%%%%%%%%%%%%%%%%%%%%%%%%%%
%   SETHEO MANUAL
%	(c) J. Schumann, O. Ibens
%	TU Muenchen 1995
%
%	%W% %G%
%
% BUILT-IN:	
% SCCS:		%W% %G%
% AUTHOR:	O. Ibens
%
%%%%%%%%%%%%%%%%%%%%%%%%%%%%%%%%%%%%%%%%%%%%%%%%%%%
\subsection{\$setcounter/2}

\begin{description}
\item[Synopsis:]
	{\tt \$setcounter(N,T)}
\item[Parameters:]\ \\[-0.5cm]
	\begin{description}
	\item[{\tt N}] number of counter to access
	\item[{\tt T}] value to store in the {\tt N}--th counter
	\end{description}
\item[Low Level Name:]
	{\tt set\_counter}
\item[Result:]\ \\
        If {\tt N} is not a number or less than $1$ or greater than
        the number of provided counters this built--in returns an
        error state.
        If {\tt T} is not a number this built--in returns an
        error state as well.
        Otherwise this built--in succeeds.
\end{description}

\vspace*{0.5cm}
\noindent
{\bf Description.}
{\tt \$Setcounter} first checks if {\tt N} is a correct index to the
block of provided counters and if {\tt T} is a number. If so, {\tt
\$setcounter} stores the value of {\tt T} in the {\tt N}--th counter.

\vspace*{0.5cm}
\noindent
{\bf Notes.}
The values to be stored have to be numbers and they have to be in the
range $[-32768,32767]$.

\vspace*{0.5cm}
\noindent
{\bf Example.}
\begin{verbatim}
:- $initcounters(5), p(X), $setcounter(2,10), q(X).
\end{verbatim}



