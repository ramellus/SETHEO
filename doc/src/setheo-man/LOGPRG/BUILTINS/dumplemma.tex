%%%%%%%%%%%%%%%%%%%%%%%%%%%%%%%%%%%%%%%%%%%%%%%%%%%
%   SETHEO MANUAL
%	(c) J. Schumann, O. Ibens
%	TU Muenchen 1995
%
%	%W% %G%
%
% BUILT-IN:	
% SCCS:		%W% %G%
% AUTHOR:	J. Schumann
%
%%%%%%%%%%%%%%%%%%%%%%%%%%%%%%%%%%%%%%%%%%%%%%%%%%%
\subsection{\$dumplemma/0}

\begin{description}
\item[Synopsis:]
	{\tt \$dumplemma}
\item[Low Level Name:]
	{\tt dumplemma}
\item[Result:]\ \\
This built-in always succeeds.
%---------------------------------
\end{description}

\vspace*{0.5cm}
\noindent
{\bf Description.}
This built-in dumps all lemmata in the lemma-store
onto the current output-file (as set by {\bf tell/1}), or on
stdout.
Lemmata are printed in a LOP-like syntax. Therefore, the output can directly
be processed by {\bf inwasm(1)}.
Lemmata which have been marked deleted are preceeded by a {\bf \#}.
This option is present for debugging and testing purposes and may
be removed in future versions.

\vspace*{0.5cm}
\noindent
{\bf Example.}
\begin{verbatim}
printlemma :- $tell("out"),$write("Lemmata:\n"),
              $dumplemma,$told.

\end{verbatim}
All lemmata present in the index are printed into the file {\tt out}.


