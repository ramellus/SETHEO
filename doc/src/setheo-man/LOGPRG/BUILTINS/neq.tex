%%%%%%%%%%%%%%%%%%%%%%%%%%%%%%%%%%%%%%%%%%%%%%%%%%%
%   SETHEO MANUAL
%	(c) J. Schumann, O. Ibens
%	TU Muenchen 1995
%
%	%W% %G%
%
% BUILT-IN:	
% SCCS:		%W% %G%
% AUTHOR:	J. Schumann
%
%%%%%%%%%%%%%%%%%%%%%%%%%%%%%%%%%%%%%%%%%%%%%%%%%%%
\subsection{\$neq/2, =/=}

\begin{description}
\item[Synopsis:]
	{\tt \$eq(T1,T2) , T1 =/= T2 }
\item[Parameters:]\ \\[-0.5cm]
	\begin{description}
	\item[{\tt T1,T2}]
terms to be checked for syntactical inequality.
	\end{description}
\item[Low Level Name:]
	{\tt neq\_built}
\item[Result:]\ \\
This built-in fails, if {\tt T1} is syntactical equal to
{\tt T2}.
\end{description}

\vspace*{0.5cm}
\noindent
{\bf Description.}
This predicate checks two terms for syntactical equality.
For the definition of syntactical equality see {\tt \$eq/2}.

\vspace*{0.5cm}
\noindent
{\bf Example.}
\begin{verbatim}
p(X)<- $neq(f(a),X).
p(X)<- $neq(f(Y),X).
p(X)<- f(Y) =/= X.
\end{verbatim}
The first clause succeeds, unless $X$ is instantiated to $f(a)$.
The second and third clause, {\em always\/} succeed.
The third clause is equivalent to the second one.
 
