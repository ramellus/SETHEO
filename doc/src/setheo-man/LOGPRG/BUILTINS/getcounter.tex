%%%%%%%%%%%%%%%%%%%%%%%%%%%%%%%%%%%%%%%%%%%%%%%%%%%
%   SETHEO MANUAL
%	(c) J. Schumann, O. Ibens
%	TU Muenchen 1995
%
%	%W% %G%
%
% BUILT-IN:	
% SCCS:		%W% %G%
% AUTHOR:	O. Ibens
%
%%%%%%%%%%%%%%%%%%%%%%%%%%%%%%%%%%%%%%%%%%%%%%%%%%%
\subsection{\$getcounter/2}

\begin{description}
\item[Synopsis:]
	{\tt \$getcounter(N,T)}
\item[Parameters:]\ \\[-0.5cm]
	\begin{description}
	\item[{\tt N}] number of counter to access
	\item[{\tt T}] term to unify the value of the {\tt N}--th
                       counter with
	\end{description}
\item[Low Level Name:]
	{\tt get\_counter}
\item[Result:]\ \\
        If {\tt N} is not a number or less than $1$ or greater than
        the number of provided counters this built--in returns an
        error state.
        If {\tt T} is not unifiable with the value of the {\tt N}--th
        counter this built--in fails.
        Otherwise {\tt \$getcounter(N,T)} succeeds.
\end{description}

\vspace*{0.5cm}
\noindent
{\bf Description.}
{\tt \$Getcounter} first checks if {\tt N} is a correct index to the
block of provided counters. If so, {\tt \$getcounter} tries to unify
{\tt T} with the value of the {\tt N}--th counter.

\vspace*{0.5cm}
\noindent
{\bf Notes.}
\begin{enumerate}
\item{Since {\tt \$initcounters} does not initialize the memory cells
      {\tt \$getcounter(N,T)} should only be called after {\tt
      \$setcounter} has been applied to the {\tt N}--th counter.}
\item{After successful unification {\tt T} is interpreted as a
      number.} 
\end{enumerate}

\vspace*{0.5cm}
\noindent
{\bf Example.}
\begin{verbatim}
:- $initcounters(5), p(X), $setcounter(2,10), q(X), $getcounter(2,T).
\end{verbatim}


