%%%%%%%%%%%%%%%%%%%%%%%%%%%%%%%%%%%%%%%%%%%%%%%%%%%
%   SETHEO MANUAL
%	(c) J. Schumann, O. Ibens
%	TU Muenchen 1995
%
%	%W% %G%
%
% BUILT-IN:	
% SCCS:		%W% %G%
% AUTHOR:	O. Ibens
%
%%%%%%%%%%%%%%%%%%%%%%%%%%%%%%%%%%%%%%%%%%%%%%%%%%%
\subsection{\$initcounters/1}

\begin{description}
\item[Synopsis:]
	{\tt \$initcounters(N)}
\item[Parameters:]\ \\[-0.5cm]
	\begin{description}
	\item[{\tt N}] number of counters to provide
	\end{description}
\item[Low Level Name:]
	{\tt init\_counters}
\end{description}

\vspace*{0.5cm}
\noindent
{\bf Description.}
This built--in provides {\tt number} memory cells for storing counter
values which are not deleted in case of backtracking.

\vspace*{0.5cm}
\noindent
{\bf Notes.}
\begin{enumerate}
\item{The memory cells are not initialized. This ensures that they are
      not reinitialized resp.\ their values are not deleted in case of
      backtracking.}
\item{Make sure {\tt \$initcounters} is the first subgoal of the
      query. This will prevent overwriting the stored values in case of
      backtracking.}
\item{There should be only one subgoal {\tt \$initcounters} in a
      query. After the second call of {\tt \$initcounters} the first
      counter block is not accessible any more.}
\end{enumerate}

\vspace*{0.5cm}
\noindent
{\bf Example.}
\begin{verbatim}
:- $initcounters(5), p(X), q(X).
\end{verbatim}

