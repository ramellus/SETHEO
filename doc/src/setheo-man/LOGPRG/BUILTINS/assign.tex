%%%%%%%%%%%%%%%%%%%%%%%%%%%%%%%%%%%%%%%%%%%%%%%%%%%
%   SETHEO MANUAL
%	(c) J. Schumann, O. Ibens
%	TU Muenchen 1995
%
%	%W% %G%
%
% BUILT-IN:	
% SCCS:		%W% %G%
% AUTHOR:	J. Schumann
%
%%%%%%%%%%%%%%%%%%%%%%%%%%%%%%%%%%%%%%%%%%%%%%%%%%%
\subsection{Destructive Assignment, :=}

\begin{description}
\item[Synopsis:]
	{\tt \$V := T}
\item[Parameters:]\ \\[-0.5cm]
	\begin{description}
	\item[{\tt \$V}]
is the name of a global variable.
	\item[{\tt T}]

T is an arbitrary term which is gets (destructively) assigned
to the global variable.


	\end{description}
\item[Low Level Name:]
	{\tt assign}
\item[Result:]\ \\
This statement always succeeds.
\end{description}

\vspace*{0.5cm}
\noindent
{\bf Description.}
The term {\tt T} is not copied during the assignment.
This means that later substitutions of variables present in {\tt T}
also alter the contents of the global variable.
Copying of the entire term can be accomplished using
{\tt \$functor/3}.

\vspace*{0.5cm}
\noindent
{\bf Side-effects.}
Destructive term assignment to global variables.

\vspace*{0.5cm}
\noindent
{\bf Example.}
\begin{verbatim}
p <- $G := [p_entered | $G],...
\end{verbatim}

Here, we enter a symbol to a list which is kept in the global
variable {\tt \$G}.




