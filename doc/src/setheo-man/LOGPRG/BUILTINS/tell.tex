%%%%%%%%%%%%%%%%%%%%%%%%%%%%%%%%%%%%%%%%%%%%%%%%%%%
%   SETHEO MANUAL
%       (c) J. Schumann, O. Ibens
%       TU Muenchen 1995
%
%       %W% %G%
%
% BUILT-IN:     
% SCCS:         %W% %G%
% AUTHOR:       J. Schumann
%
%%%%%%%%%%%%%%%%%%%%%%%%%%%%%%%%%%%%%%%%%%%%%%%%%%%
\subsection{\$tell/1}\label{sec:tell}

\begin{description}
\item[Synopsis:]
        {\tt \$tell(T)}
\item[Parameter:]\ \\[-0.5cm]
        \begin{description}
        \item[{\tt T}] term or string
        \end{description}
\item[Low Level Name:]
        {\tt tell}
\item[Result:]\ \\
This built-in succeeds, if {\tt T} is bound to a valid file-name, and a
file with that name could be opened. Otherwise a fail occurs and a
warning message is issued.
\end{description}

\vspace*{0.5cm}
\noindent
{\bf Description.}
When this built-in is encountered, the \SAM\ tries to open a file
with the given name in the ``append mode'' ("+a"). 
Subsequent outputs (e.g., by
{\tt \$write} or {\tt \$dumplemma}) will be directed into that file.
If a named file has been open previously, that file is closed prior
to opening the new one.

In contrast to PROLOG, no nested {\tt \$tell}'s are allowed.

\vspace*{0.5cm}
\noindent
{\bf Note.}
Since strings and terms are treated likewise, {\tt \$tell(foo)} and
{\tt \$tell("foo")} have the same effect.

\vspace*{0.5cm}
\noindent
{\bf Example.}
\begin{verbatim}
p(X) :- $tell("output"),$write(X),$told.
\end{verbatim}


