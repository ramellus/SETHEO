%%%%%%%%%%%%%%%%%%%%%%%%%%%%%%%%%%%%%%%%%%%%%%%%%%%
%   SETHEO MANUAL
%	(c) J. Schumann, O. Ibens
%	TU Muenchen 1995
%
%	%W% %G%
%%%%%%%%%%%%%%%%%%%%%%%%%%%%%%%%%%%%%%%%%%%%%%%%%%%
\section{Syntax Definition of a Proof Tree}\label{sec:tree-syntax}
\subsection{The Tree File}

Files with the extension {\em .tree\/} are generated by the \SAM\ and
contain Model Elimination Proofs, i.e., representations of closed
tableaux.
For each inference step, its kind (extension step, reduction step,
factorization step), and additional information is kept. 
The literals of the clauses are stored with the variable-substitutions
applied.
The syntax of tree-file is defined in the following. All proof are stored
as PROLOG terms, such that {\bf inwasm} or a PROLOG program can read
and process the proof.


\begin{verbatim}
proof ::= treelist '.'

treelist ::= '[' seq_of_trees ']'

seq_of_trees ::= 
         '[' tree ']'
         | seq_of_trees ',' '[' tree ']'
\end{verbatim}

One tree-file can actually contain more than one Model Elimination tableaux.
Currently, however, {\bf xptree} can only display one tree. In case
more than one trees are given, obly the first one is displayed.

\begin{verbatim}
tree ::= literal { ',' '[' infnumber ',' infdescr ']' { ',' treelist } }
\end{verbatim}

Each node of the tree contains a literal, together with (optional)
information. Its is followed by the tree-representation of all its
immediate successors. As an example consider the following clause:
{\tt p :- q,r,s.} Then, given the subgoal {\verb+~p+}, the following
tree structure is generated (the {\tt \_\_} denote the addtional
information:

\begin{verbatim}
[ ~p , [____],       // goal
  [ p ],             // head of the clause
  [ ~q , [____],
     [ <subtree> ]  // for subgoal '~q'
  ] ,
  [ ~r , [____],
     [ <subtree> ]  // for subgoal '~r'
  ] ,
  [ ~s , [____],
     [ <subtree> ]  // for subgoal '~s'
  ]
]   
\end{verbatim}

For each extension step, the list of subtrees {\tt treelist} contains
at least one element. If, however, a subgoal was closed by a reduction-
or factorisation step,  this list is empty.

The {\tt literal} is defined in a similar way as in LOP (see Section~\ref{Sec:LOP-syntax}). Here, however, no infix operations 
(except {\tt =}) are allowed\footnote{
	Actually, all infix operations of LOP are handled with
	built-in predicates which do not show up in a proof tree.
	}.

\begin{verbatim}
infnumber ::= [1-9][0-9]*

infdescr ::= 'ext' '(' number '.' number ',' number '.' number ')'
           | 'red' '(' number '.' number ','  number ')'
           | 'fac' '(' number '.' number ','  number ')'
           | 'built_in'
           | 'not yet touched'
\end{verbatim}

The inference number is a unique sequence number for each node of the
tree. The maximal inference number corresponds to the number of inferences
in the proof. The item {\tt infdescr} gives information about the
current inference. The first pair of numbers identifies the literal,
{\em from\/} which the inference was made (``subgoal'').
A literal is always identified by its clause number (1st number) and
the relative number of the literal in the clause. Note, that subgoal
reordering does not affect the numbering.

For an extension step ({\tt ext}), the second pair of numbers indicate,
into which literal the extension step has been made.
The second argument of the other inference steps denote the inference
number of the node, into which a reduction (or factorization step)
has been made.

{\tt not yet touched} indicates a subgoal which has not been solved.
This situation can only occur when a proof tree is generated during
the serach of the \SAM, using the built-in {\tt ptree} 
(see Section~\ref{sec:logprg-ptree}).
Finally, the token {\tt buil\_in} exists, but it is not used within the
current version of SETHEO.

\subsection{The Operator Translation Table}
\label{sec:file-formats:opdefs}

In the SETHEO system, all operators (except certain built-ins) are
represented in prefix notation.
The graphical output of SETHEO, using xptree, however, allows to
display such operators in infix- or postfix notation, using definable
print-names.
A translation table must be present in a file and can be loaded using
xptree's {\tt -xtab {\em file}} option.

The file, describes the type of the operators
(infix,  prefix,  postfix),  their  binding  power and their
print-names is as follows: Each line of  the  file  contains
one definition. Blank lines or comment lines are not permitted. 
A definition consists of four fields,  separated  by  a colon (':'):
      \begin{description}
      \item[the parse-name]
           {\ \\
            is the name of the predicate or function symbol  as  it
            appears  in  the  tree file produced by the \SAM.  It
            starts with a lower case letter and may  contain  lower
            and  uppercase letters, digits, and the underscore '\_'.
            Blanks and other special characters are not allowed.}
      \item[the print-name]
           {\ \\
            is the string which is displayed whenever a  term  with
            the  given  operator is displayed. Print--names may
            contain arbitrary characters (except the colon, which  has
            to  be  preceeded  by  a back--slash).  In particular, a
            print--name can contain blanks to enhance the  readability
            of a term.}
      \item[binding power]
           {\ \\
            is an integer, defining the binding power of the operator.}
      \item[type]
           {\ \\
            is a character, indicating the type  of  the  operator:
            'i'  or 'I' for infix, 'p' or 'P' for prefix and 'q' or
            'Q' for postfix.}
      \end{description}

\begin{verbatim}
convtab ::=
       ctabline
       | ctabline convtab

ctabline ::=
    pt_symbol^':'^char_sequence^':'^number^':'^optype^'\n'

pt_symbol ::=
	symbol | '~' | '[' | ']'

char_sequence ::= [^:]+           /* string, must not contain ':' */

optype ::= 'I' | 'i' | 'p' | 'P' | 'q' | 'Q'
\end{verbatim}

      Predefined operators are `$\sim$` and '$[$' as prefix operators with a
      binding power of 50. All entries in the given file are processed
      in reverse order.  Therefore, the definitions of '$\sim$' and '$[$'
      can be changed by the user.  The following example shows the
      format of the entries into the file:

\begin{verbatim}
l: <= :100:i
equal: = :100:i
in: in :100:i
ispre: [= :100:i
less: < :100:i
filt:@:200:i
cons:*:200:i
gt: > :100:i
\end{verbatim}


	{\tt left-right associativity?}
