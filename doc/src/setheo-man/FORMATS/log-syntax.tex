%%%%%%%%%%%%%%%%%%%%%%%%%%%%%%%%%%%%%%%%%%%%%%%%%%%
%   SETHEO MANUAL
%	(c) J. Schumann, O. Ibens
%	TU Muenchen 1995
%
%	%W% %G%
%%%%%%%%%%%%%%%%%%%%%%%%%%%%%%%%%%%%%%%%%%%%%%%%%%%
\section{The \SAM\ Logfile}\label{sec:logfile}

While the \SAM\ tries to find a proof, statistical information
is generated. This information is printed on the screen as well as into
the \SAM\ logfile. The extension of the logfile is {\bf .log}. An
example of a \SAM\ logfile is shown in Figure~\ref{fig:logfile}. 
\begin{figure}[htb]

\begin{verbatim}
SAM V3.3 Copyright TU Munich (December 22, 1995)

Options : -dr -cons -dynsgreord 2 WOS4 

using antilemma-constraints
using regularity-constraints
using tautology-constraints
using subsumption-constraints

Start proving...

-d:    2  time     < 0.01 sec  inferences =        5  fails =       29
-d:    3  time =     0.05 sec  inferences =      266  fails =     1176
-d:    4  time =     0.03 sec  inferences =      105  fails =      197

                    ******** SUCCESS ************

Number of inferences in proof      :       13
        - E/R/F/L                  :        9/       4/       0/       0
Intermediate free variables        :        3
Intermediate inferences            :       13
Intermediate open subgoals         :        5
Generated antilemmata              :        2
Number of unifications             :      376
    - E/R/F/L                      :      330/      39/       7/       0
Number of generated constraints    :       44
    - anl/reg/ts                   :        0/       6/      38
Number of fails                    :     1402
    - unification                  :      197
    - depth bound                  :     1165
    - constraints                  :       40
        - anl/reg/ts               :        2/       6/      32
Number of folding operations       :       12
    - one level                    :        5
    - root                         :        8
Instructions executed              :     2992
Abstract machine time (seconds)    :     0.08
Overall time          (seconds)    :     0.43
\end{verbatim}
\caption{A \SAM\ logfile.}\label{fig:logfile}
\end{figure}


At top of the logfile the number of the \SAM's version is
printed, e.g.\ 
\begin{description}
\item [\phantom{1234}] 
      {\bf SAM V3.3 Copyright TU Munich (December 22, 1995)}
\end{description}
This is important because different \SAM\ versions may produce
different statistics. 
The next thing to be printed in the logfile are the options and the
name of the problem file the \SAM\ is called with, e.g.\ 
\begin{description}
\item [\phantom{1234}] 
      {\bf Options : --dr --cons --dynsgreord 2 WOS4}
\end{description} 
Each sort of turned on constraints is listed as well\footnote{This
list refers only to the \SAM's parameters and not to which constraints
are really effecting the proof. I.e.\ some of the used constraints
might have no effect if the corresponding options have not been turned
on in the preprocessing phase of \inw\ (see
Remark~\ref{rem:constr-options} in Section~\ref{sec:sam}).}.  
The beginning of the search to the proof is indicated by 
\begin{description}
\item [\phantom{1234}] 
      {\bf Start proving \dots}
\end{description} 
If the \SAM\ is called with an iterative bound (e.g.\ {\bf --dr}), for
each iteration step during computation the following information is
printed. 
\begin{enumerate}
\item The value of the increased bound. Since the \SAM\ provides
      different iterative bounds, a short string before the value
      indicates which bound is increased.
      \begin{description}
      \item [--d:] Depth bound.
      \item [--wd:] Weighted depth bound.
      \item [--i:] Inference bound.
      \item [--loci:] Local inference bound.
      \end{description}
\item The runtime (CPU--time) in seconds, introduced by the string
      {\bf time}. 
\item The total number of inferences, introduced by the string {\bf
      inferences}. 
\item The number of fails, introduced by the string {\bf fails}.
\end{enumerate}

When a proof is found,
\begin{center} 
{\bf ***** SUCCESS *****}
\end{center}
is printed into the logfile. Otherwise the reason for not finding a
proof is given. 
\begin{center} 
{\bf ***** TOTAL FAILURE *****}
\end{center}
means that the given problem is satisfiable. This message also
indicates that this failure does not depend on \SAM's invocation
parameters. 
\begin{center} 
{\bf ***** BOUND FAILURE *****}
\end{center}
indicates that the failure was caused by the given bounds. With
different bounds set a proof might be found.
\begin{center} 
{\bf ***** INTERRUPT FAILURE *****}
\end{center}
is printed, if the \SAM\ is running in the batch mode and the run is
interrupted, e.g.\ by \Cc. Read Section~\ref{sec:sam} for an
explanation of the {\bf --batch} option. 
\begin{center} 
{\bf ***** REAL TIME FAILURE *****}
\end{center} 
is printed, if the \SAM\ is running with limited real time and the
given time--limit has been exhausted. Read Section~\ref{sec:sam} for an
explanation of the {\bf --realtime} option.
\begin{center} 
{\bf ***** CPU TIME FAILURE *****}
\end{center} 
is printed, if the \SAM\ is running with limited CPU time and the
given limit has been exhausted. Read Section~\ref{sec:sam} for an
explanation of the {\bf --cputime} option.
\begin{center} 
{\bf ***** E R R O R *****}
\end{center}
is printed, if an internal \SAM\ error occured. This indicates
inconsistent changes to the source code or an illegal use of
built--ins. 

When the run is over all non--zero statistical counters are written to
the logfile. The statistical counters are the following. 
\begin{description}

\item[Number of inferences in proof:]
     number of successful, not backtracked unifications
     \begin{description}
     \item [--E:] number of successful, not backtracked extension steps
     \item [--R:] number of successful, not backtracked reduction steps
     \item [--F:] number of successful, not backtracked factorization
                  steps 
     \item [--L:] number of successful, not backtracked unit--lemma
                  uses (always zero in the current version) 
     \end{description}
     
\item[Intermediate free variables:]
     intermediate maximum number of simultaneously free variables
     
\item[Intermediate term complexity:]
     intermediate maximum term complexity (only displayed, if the
     termcomplexity bound is set) 
     
\item[Intermediate inferences:]
     intermediate maximum number of inferences (i.e.\ successful, not
     backtracked unifications) 
     
\item[Intermediate open subgoals:]
     intermediate maximum number of simultaneously open subgoals
     
\item[Generated antilemmata:]
     total number of generated antilemmata
     
\item[Elapsed Time:]
     elapsed time since last interrupt (only displayed, if the
     computation is interrupted, e.g. by \Cc)
     
\item[Number of unifications:]
     total number if tried unifications (including unsuccessful and
     backtracked unifications)
     \begin{description}
     \item [--E:] number of tried extension steps
     \item [--R:] number of tried reduction steps
     \item [--F:] number of tried factorization steps
     \item [--L:] number of tried unit--lemma uses (always zero in the 
                  current version) 
     \end{description}
     
\item[Number of generated constraints:]
     total number of constraints generated during computation (only
     displayed, if constraints have been generated)
     \begin{description}
     \item [--anl:] number of antilemma constraints
     \item [--reg:] number of regularity constraints
     \item [--ts:] number of tautology or subsumption constraints 
     \end{description}
     
\item[Number of fails:]
     total number of fails (the following numbers are only displayed,
     if they are greater than zero) 
     \begin{description}
     \item [--unification:] number of unification fails
     \item [--depth bound:] number of depth bound fails
     \item [--inference bound:] number of inference bound fails
     \item [--variable bound:] number of variable bound fails
     \item [--term complexity bound:] number of termcomplexity bound
                                      fails 
     \item [--open subgoals bound:] number of subgoal bound fails
     \item [--local--inf bound:] number of local inference bound fails
     \item [--constraints:]
           total number of constraint fails
          \begin{description}
          \item [--anl:] number of antilemma constraint fails
          \item [--reg:] number of regularity constraint fails
          \item [--ts:] number of tautology or subsumption constraint
                       fails 
          \end{description}
     \end{description}
     
\item[Number of folding operations:]
     total number of folding operations
     \begin{description}
     \item [--one level:] number of folding operations one level up
     \item [--root:] number of folding operations up to root, i.e.\
                     number of possible lemmata 
     \end{description}
     
\item[Instructions executed:]
     number of executed \SAM\ instructions\footnote{Examples for \SAM\
     instructions are {\bf calling a clause}, {\bf calling a subgoal},
     {\bf backtracking}, {\bf finishing a subproof}.}
     
\item[Abstract machine time:]
     pure proving time\footnote{These times are measured
     with a granularity of $1/60$~seconds (depending on the UNIX
     system).} 
     of the \SAM\ program in seconds (i.e.\ runtime
     without time for reading the options and the formula and without
     time for displaying the statistics). 
     
\item[Overall time:]
     total runtime\footnote{These times are measured
     with a granularity of $1/60$~seconds (depending on the UNIX
     system).}  
     of the \SAM\ program in seconds (i.e.\ time for
     reading the options and the formula, proving time, time for
     displaying the statistics)
     
\item[Genlemma]
     statistics concerning the generation of unit--lemma (only
     displayed, if the concerning numbers are greater than zero). For
     details see the {\bf genlemma} built--in predicate in
     Section~\ref{subsec:built-in-genlemma}. 
     \begin{description}
     \item [--entered:] 
           number of times, the built--in to generate unit--lemtata has
           been entered
     \item [--no-unitlemma:]
           number of times, it had been tried to generate a non--unit
           lemma 
     \item [--GEN/w.match:] 
           {\bf GEN } indicates, how many more general unit--lemmata 
           could be found in the index. {\bf w.match} is the
           number of times, a really more general lemma could
           be found in the index. The ratio between {\bf GEN/w.match}
           indicates, how good the filter provided with the path index 
           really is. Note: Whenever a more general unit--lemma
           already exists in the index, the new lemma is not entered.
     \item [--INST/w.match:] 
           {\bf INST } indicates, how many more instantiated
           unit--lemmata could be found in the index. {\bf w.match} is
           the number of times, a really more instantiated lemma could
           be found in the index. The ratio between {\bf INST/w.match}
           indicates, how good the filter provided with the path index
           really is. Note: Whenever a more instantiated unit--lemma
           is found, it will be removed from the index (i.e.\ it
           cannot be used any more). The new lemma will be stored in the 
           index. 
     \item [--stored:] 
           number of unit--lemmata actually stored in the index
     \item [--Nr. in Index:] 
           number of unit--lemmata which can be used by the prover
     \end{description}

\item[Uselemma]
     statistics concerning the use of unit-lemmata (only displayed, if
     the concerning numbers are greater than zero). For
     details see the {\bf uselemma} built--in predicate in
     Section~\ref{subsec:built-in-uselemma}.  
     \begin{description}
     \item [--entered:]
           number of times the built--in to use unit--lemmata has been
           called 
     \item [--no:]
           number of unsuccessful tries to use a lemma (In that case,
           the index does not return any unifiable unit--lemma.)
     \item [--Nr. pushed:]
           number of times, unit--lemmata are activated to be used
           (i.e.\ appended to the current choice point)
     \end{description}

\item[\#\#\# :]
     memory statistics (only displayed, if the
     \SAM\ is running in verbose mode\footnote{Read
     Section~\ref{sec:sam} for an explanation of the {\bf --verbose}
     option.}), including
     \begin{description}
     \item [Pagesize]
     \item [Marksize]
     \item [Memory demanded]
     \item [Memory freed]
     \item [Memory remaining]
     \item [Pages allocated]
     \end{description}
     
\end{description}


