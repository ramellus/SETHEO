%%%%%%%%%%%%%%%%%%%%%%%%%%%%%%%%%%%%%%%%%%%%%%%%%%%
%   SETHEO MANUAL
%	(c) J. Schumann, O. Ibens
%	TU Muenchen 1995
%
%	%W% %G%
%%%%%%%%%%%%%%%%%%%%%%%%%%%%%%%%%%%%%%%%%%%%%%%%%%%

\section{The LOP-Syntax}
\label{sec:lop-syntax}

The general input language of the SETHEO-system is
LOP (LOgic Programming Language) and is used to write
formulas in clausal normal form (CNF) and to write logic
programs, using SETHEO's procedural facilities.
LOP formulas (or programs) are normally located in files
with the extension {\bf .lop} and are read by {\bf inwasm}
and the modules {\bf setheo}, {\bf clop} and {\bf delta}.
In this section, the exact definition of the syntax of a LOP
formula or program is given in a BNF-like form\footnote{
	This section is based from a definition of the LOP
	syntax by R.~Letz
}.

The usage of white spaces (blanks, tabs and new-lines) is
kept as flexible as possible. White spaces can be inserted or
removed arbitrarily, unless otherwise noted.
Comments are written in a C-like style surrounded
by {\tt /* */}.  Nested comments are not allowed.
A line starting with a {\tt \#} or a {\tt \%} is also treated as a comment.

LOP-files consist of a non-empty list of clauses\footnote{
	In this manual, we only describe the syntax of pure
	LOP; the syntax of MPLOP, which further allows to specify
	modules, is not covered here.}
of different types (axioms, goals, queries and rules).
Each clause is ended by a full stop '{\tt .}'.

\begin{verbatim}
lop-program   ::=
               clause '.'
               | clause '.' lop-program

clause        ::=
               axiom
               | goal
               | query
               | rule
\end{verbatim}

Let's start with the types of clauses which are used for
theorem-proving: axioms and goals.
\begin{verbatim}
axiom         ::=
               head '<-' tail constraints
               | head '<-' constraints

goal          ::=
               '<-' tail constraints
\end{verbatim}
The head and tail of a clause are non-empty lists of literals;
the {\tt <-} separates the positive literals from the negative
ones.
{\tt constraints} denotes (a possible empty) list of
inequality constraints.

The logic-programming style clauses (query and rule) have a
similar syntax which is close to PROLOG. The head of each rule
can be just one literal only.
\begin{verbatim}
rule          ::=
               literal ':-' tail constraints
               | literal ':-' constraints

query         ::=
               '?-' tail constraints
\end{verbatim}


In general, the following differences and similarities
exist in the handling of these 4 types of clauses by SETHEO:
\begin{enumerate}
\item
All literals in front of the separator have per default
a {\em positive\/} sign, those after the separator (i.e. literals
in the tail) have a negative one.
\item
Each {\tt goal} and {\tt query} is used as a start clause 
(in the Model Elimination Calculus)\footnote{
	If a clause, consisting of negative literals only, is not
	to be used as a start clause, this clause can be transformed
	as described in Section~\ref{sec:startclauses}.
	}.
\item
If there exists at least one Non-Horn clause in the formula, each
{\tt axiom} and {\tt goal} is transformed into contrapositives.
\item
No {\tt query} or {\tt rule} is transformed into contrapositives.
\item
An axiom with one literal only is syntactically and semantically
the same as a rule with one literal (``fact'').
\item
Literals in the tail of a
{\tt axiom} and {\tt goal} are subject to subgoal reordering
(unless turned off with the {\bf inwasm} switch {\tt -nosgreord}),
whereas those in a
{\tt query} or  {\tt rule} are not reordered.
\end{enumerate}


The positive and negative parts of the clauses are lists
of literals. A literal itself is an atom or its negation.
Built-ins can occur in the tail of a clause only.
\begin{verbatim}
head          ::=
               literal
               | literal ';' head

tail          ::=
               tail-literal
               | tail-literal ',' tail

tail-literal  ::=
               literal
               | built-in

literal       ::=
               atom
               | '~' atom
\end{verbatim}

In the following, we define the structure of an {\tt atom}, which
follows the usual lines of a definition of terms. Please note
that additionally, {\em lists\/} of terms are allowed within
terms. They are used in a PROLOG-like way.

As in PROLOG, constants start with a lower-case letter,
variables with an upper-case letter or an underscore {\tt \_}.
Anonymous variables are denoted like in PROLOG, by an {\tt\_}.
Furthermore, short strings (quoted in ") are allowed. Please note
that, however, a string {\tt "plus"} and a constant {\tt plus} are
equal. This e.g. disallows a string {\tt "*"}, since the asterisk is
a reserved symbol.
Special characters are expressed in a C-like manner:
predefined are 
\verb+\t+ for tabulator,
\verb+\n+ for newline and
\verb+\OOO+ for a 3-digit octal value of the character's
ASCII code. E.g. \verb+\007+ rings the bell.
%
Numbers must be in the range of [-32768, 32767].
%
Additionally, the following consistency conditions must be
fulfilled:
\begin{enumerate}
\item
All predicate and function symbols can occur with one arity and
as a constant of arity $0$ only. This means that {\tt f(a)} and
{\tt f} are allowed in one formula, but not {\tt f(a)} and
{\tt f(a,b)}.
\item
Predicate symbols may also occur as function symbols of the
same arity (or used as a constant with arity 0), 
as long as the {\em first} occurrence
of the symbol in that file is on the level of predicates. E.g.
{\tt <- p(a,b). q(p)<-...} is allowed, but not the
other way round. (Sorry)\footnote{
	The best way to avoid this problem (in the inwasm compiler),
	is to preceed the formula with a line of the
	form $p(X_1,\ldots,X_n) \leftarrow false.$ for
	each predicate symbol $p$ with arity $n$.
	}
\end{enumerate}
\begin{verbatim}
atom           ::=
                constant
                | constant '(' termlist ')'

termlist       ::=
                term
                | term ',' termlist

term           ::=
                constant
                | constant '(' termlist ')'
                | number
                | variable
                | global-variable
                | list

constant       ::=
                '[a-z][A-Za-z0-9_]*'
                | '"[^"]*"'

number         ::=
                '[0-9]+'
                | '-[0-9]+'

variable       ::=
                '[A-Z_][A-Za-z0-9_]*'

global-variable::=
                '$[A-Z_][A-Za-z0-9_]*'

list           ::=
                '[]'
                | '[' term '|' term ']'
                | '[' termlist ']'
\end{verbatim}
	
Built-ins are either infix expressions or have a similar
structure as atoms. They only can occur in the tail of
a clause (i.e. they occur negatively).
The names of the pre-fix built-ins
start with a {\tt \$}. All built-ins are described
in detail in Chapter~\ref{chap:6}.

\begin{verbatim}
built-in       ::=
                built-in-pred
                | built-in-pred '(' termlist ')'
                | variable 'is' numexpr
                | global-variable ':is' numexpr
                | global-variable ':=' term
                | term '=' term
                | term '==' term
                | term '=/=' term
                | numexpr relop numexpr

numexpr        ::=
                number
                | '(' numexpr ')'
                | numexpr '+' numexpr
                | numexpr '-' numexpr
                | numexpr '*' numexpr
                | numexpr '/' numexpr

relop          ::=
                '<' | '>' | '<=' | '=='

buit-in-pred   ::=
                '$[a-z][A-Za-z0-9_]*'
\end{verbatim}

Constraints are inequality constraints, i.e. lists of
disequations of terms.
Constraints can be considered as additional literals in the tail
of a clause which are being checked permanently.

Please note that in the current version of SETHEO, constraints
which have been added to the formula with a semantic meaning
(e.g. {\tt [X] =/= [0]}), are not compatible with \SAM's built-in
antilemmata mechanism. 
If you want to use such constraints, you must
invoke \SAM\ with the {\tt -noanl} option
or without {\tt -anl} or {\tt -cons}.
All constraints generated by the compiler inwasm are pure syntactic
constraints which do not influence soundness of the prover.


\begin{verbatim}
constraints   ::=            // empty
               | ':' constr-list

constr-list   ::=
               constraint
               | constraint ',' constr-list

constraint    ::=
               '[' ctermlist ']' '=/=' '[' ctermlist ']'

ctermlist     ::=
               cterm
               | cterm ',' ctermlist

cterm         ::=
               term
               | struct-variable
               | constant '(' ctermlist ')'

struct-variable::=
               '#[A-Z_][A-Za-z0-9_]*'
\end{verbatim}

Structure variables {\tt struct-variable} in constraints
are treated as universally quantified variables (w.r.t.\ the clause
in which they occur). A constraint {\tt [X] =/= [f(\#Y)]} is
violated, if X is instantiated to a term with the top-level
functor {\tt f}, e.g. {\tt f(a)}.
