%%%%%%%%%%%%%%%%%%%%%%%%%%%%%%%%%%%%%%%%%%%%%%%%%%%
%   SETHEO MANUAL
%	(c) J. Schumann, O. Ibens
%	TU Muenchen 1995
%
%	%W% %G%
%%%%%%%%%%%%%%%%%%%%%%%%%%%%%%%%%%%%%%%%%%%%%%%%%%%
\section{\SAM\ Machine Code Syntax}

Machine code for the SETHEO Abstract Machine \SAM\ is always located
in files with the extension {\bf .hex}. 
Each entry in this file occupies one line.
A one-character identifier is used
to select the appropriate type of data contained in the current line.
Blank lines, comments or extra spaces are not allowed.

The following grammar shows the definition of the \SAM\ Machine Code:

\begin{verbatim}
file           ::= lines

lines          ::= line
                | lines line

line           ::= ':' ident ':' address ':' data ':' string '\n'

ident          ::= 'C' | 'Y' | 'E' | 'S' | 'M' | 'N'


address        ::= [0-9A-F]{8}
data           ::= [0-9A-F]{8}
string         ::= [A-Za-z_0-9]       /* see Note 4 */
\end{verbatim}

The following table shows the meaning of the fields, corresponding to
the given identifier:

\begin{center}
\begin{tabular}{lllll}
C & word in code area of \SAM &
	\SAM-code-address & contents & --- \\
Y & symbol &
	type of symbol & arity & printname \\
S & set start-address &
	\SAM-code-address & --- & --- \\
M & highest address &
	 0 & highest address & --- \\
N & number of symbols &
	 0 & symbols & --- \\
\end{tabular}
\end{center}

\noindent{\bf Notes:}
\begin{enumerate}
\item
A ``highest memory address'' (identifier: ``M'')
directive must be placed prior to any code words (``C'') in order
to be effective. Otherwise, the size of the code-area of the \SAM\ is
determined by a default value (or the {\bf sam} parameter -code).
\item
The relative line number of a ``symbol'' directive determines the index
of that symbol in the symbol table. Therefore, the order of these
directives must not be changed and their number must not be 
increased or decreased.
\item
Special characters in a symbol are encoded by: \verb+\OOO+
where {\tt OOO} is the 3-digit octal value of the ASCII code of that
character. E.g. \verb+\007+ is for {\tt BELL}.
\item
The maximal length of a symbol is limited to 42 on the 
generation side ({\bf wasm}), and to 200 in the reader of
the abstract machine.
This limit includes special characters (see above). Each such character
contributes 4 to the total length of the symbol.
\item
all other directives may be intermixed and arbitrarily changed in order.
\item
A file with machine code must at least contain one line.
\end{enumerate}
