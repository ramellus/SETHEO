\newcommand\bild[5]{{\tt #1 #2 #3 #4 }\label{#5}}

\subsection{Getting Started}
\subsubsection{Introduction} 
XTHEO is a graphic user interface for the SETHEO theorem prover.
This document introduces you to the user--interface by giving a 
guided tour through all stages of automatically proving an actual theorem.
To get an impression of the interface have a glimpse at figure~\ref{fig:whole-thing}.
%\begin{figure}[htbp]
%\begin{picture}(100,170)
%\put(0,0){\framebox(180,170)}
%\end{picture}
%\caption{Full fledged XTHEO}\label{fig:whole-thing}
%\end{figure}

\bild{xtheo01.ps}{12cm}{17cm}{Full fledged XTHEO}{fig:whole-thing}

\subsubsection{The Example--Theorem} 
The theorem we are about to prove is a classic example~\footnote{Chang and Lee, 
Symbolic Logic and Mechanical Theorem Proving, Academic Press, Orlando, 1973.} from ATP.
It alleges that from the axioms of group theory it follows that any
left--identity element is also a right--identity element. 
To recollect the fundamentals of group theory, cast a look at the definition coming up.
\\ \\
{\bf Definition}
\begin{quote}
A {\it group\/} consists of 
\begin{quote}
a set $\cal G$ 
and 
a function $* : S \times S \rightarrow S; (s1,s2) \mapsto s1 * s2$. 
\end{quote}
Additionally, two axioms must hold.
\begin{enumerate}
\item There exists an element $e$ in $\cal G$ such that: \\
\begin{itemize}
\item[a.]$e * x = x$ for any element $x$ in $\cal G$. 
(We call $e$ a left--identity element.) 
\item[b.] For any element $x$ in $\cal G$ there exists an element $x'$ in $\cal G$ such that 
$x * x' = e.$ \\ 
\end{itemize}
\item For all $x, y, z$ from $\cal G$ it holds that $(x * y) * z = x * (y * z)$.
\end{enumerate}
\end{quote}

An example of a group is $(Q\setminus \{0\},*)$, the set of rational numbers with
standard multiplication. However, $(Q,*)$ is not a group, since axiom 1.b. cannot
be satisfied\footnote{1 is the only candidate for a (left--) identity but
$0 * x' = 1$ holds for no $x'$ in $\cal G$.}.
It also follows from the group axioms that  for any group there is exactly
one identity element, but we are not to take advantage of this knowledge.
As far as we are concerned, there
exist one or more left--identity elements, which may or may not be right--identities.
Now, rendering axiom 1 in logical notation results in:
\begin{center}
	$\exists y (\forall x  (y * x = x) \:\wedge\: \forall x  \exists x' (x' * x = y))$
\end{center} 
Axiom 2 reads in logical notation as follows:
\begin{center}
	$\forall x \forall y \forall z  ((x * y) * z) = (x * (y * z))$
\end{center} 
To prove that indeed every left--identity is a right--identity we proceed
indirectly, i.e. we assume that there is an element, which is a left--identity 
but not a right--identity.
So what we want to do is to state that there is an element that is characterised by axiom 1
but is not a right--identity.
In logic notation this can be expressed as:
\begin{center}
	$\exists y (\forall x  (y * x = x) \:\wedge\: \forall x  \exists x' (x' * x = y) \:\wedge\:  \sim \forall x  (x * y = x))$
\end{center} 
which is equivalent to: There is a left-identity which is not a right--identity.
These formulas are almost, but not quite, what you can feed the theorem--prover with.
For one, since there are no provisions for equality in SETHEO's underlying calculus,
we have to paraphrase equality by introducing the 3--place predicate times\_is. The intended
interpretation for times\_is(x,y,z) is obviously $x * y = z$. Also, some of the conjunctives
we used above, do not belong to the ASCII--set. Thus, e.g. $\forall$ must be explicitly
written as {\it forall\/}\footnote{For a complete description of SETHEO's input--language consult
the chapter ``Syntax of the Input Language''.},
$\wedge$ as \&, and $\neg$ as \~\ .
Introducing these modifications to our input formulas they
turn out like this:
\\
\begin{tabbing}
exists Y \= \\
          \> ~ forall X times\_is(X,Y,X)) \\
          \> (forall X  times\_is(Y,X,X) \& \\
          \> forall X exists Z times\_is(Z,X,Y) \& \\
forall X, Y, Z, U, V, W \\
          \>(times\_is(X,Y,V) \& times\_is(Y,Z,U) -$>$ (times\_is(X,U,W) $<$-$>$ times\_is(V,Z,W)))
\end{tabbing}
Before you enter the formula, read on!
\\
\subsubsection{XTHEO--Window} 
XTHEO is a window--based interface for the SETHEO theorem--prover. To call the main window into existence,
change to xtheo's example subdirectory and type {\it xtheo\/}. XTHEO's main windows should now be visible
in the left bottom corner of the screen. It looks similar to figure~\ref{fig:xtheo-main-window}.
%\begin{figure}[htb]
%\begin{center}
%\begin{picture}(131,95)
%\input{/u1/johann/papers/setheo/man/xtheo_main_pic}
%\end{picture}
%jfjfj
%\caption{XTHEO's Main Window}\label{fig:xtheo-main-window}
%\end{center}
%\end{figure}

\bild{win01.ps}{9cm}{15cm}{XTHEO's Main Window}{fig:xtheo-main-window}

The main--window is subdivided into two parts; the upper part is called a {\it panel\/} in SUN's terminology,
the lower part a {\it tty--window\/}. On the panel you notice a couple of small ovals; these are called
{\it buttons\/} and this is quite literally what they are: moving the cursor to these buttons and clicking
the left mouse--button will call forward the action selected just as pushing buttons of a vending
machine does. 
While the cursor moves over a button you will have noticed that the first two lines in the panel change;
they provide some very concise information on the actions behind the push--buttons.

The lines marked {\it Directory\/}, {\it File\/}, and {\it Flags\/} in the panel are input--lines.
You can move from line to line with either the return--key of your terminal or by clicking a
mouse--button on a line. The line currently accepting input will show a small caret. 
Try now to enter {\it chang3\/} on the line marked {\it File\/}.
In XTHEO's examples subdirectory there should be a file {\it chang3.pl1\/}. This is identical to
the input--formula we developed above; so you don't need to type it again.

The buttons in the middle of the panel labelled {\it PLOP\/}, {\it preproc\/}, {\it inwasm\/}, {\it wasm\/}, {\it setheo\/}, and
{\it prooftree\/} represent, applied from left to right, the seqence of steps from the 
input--formula in standard logic notation to a graphical representation of the proof--tree, provided 
there is a proof and SETHEO finds it. Below the buttons is an additional line of buttons
marked {\it flags\/}; on clicking the left mouse--button while the cursor is on a button marked {\it flags\/}
an additional window will appear, which is called pop--up in SUN's terminology.
It seems to pop up out of blue air and appears to float slightly above everything else
on the screen.
You can open the pop--up if you are curious, but we do not need any flags for the PLOP program;
to close the pop--up again click at the {\it DONE\/} button. 
\paragraph{PLOP}
Now we are ready to get on our way proving the theorem. Click the {\it PLOP\/} button now.
There are two possibilities:
If anything went wrong, e.g. you misspelled {\it chang3\/} or you have not been in XTHEO's examples
subdirectory when calling XTHEO, you might see an {\it alert--popup\/} on your screen. An alert--popup
is a subwindow with a huge black arrow pointing into the window. Usually an
alert tells you that you made a mistake or that you are about to invoke an action that you better think
about twice. There is no way just to ignore an alert. You cannot go on unless having chosen one of the
options in the alert-- subwindow. \\
If, on the other hand, things are going smoothly, the PLOP program should have translated chang3 from
predicate logic to LOP, which is a language not unlike Prolog but with some extensions.
In the lower part of the main--window chang3 should appear in LOP--syntax:
\begin{center}
\begin{verbatim}
    <- times_is(a_2,a_1,a_2) . 
    times_is(a_1,X,X) <- . 
    times_is(f_1(X),X,a_1) <- . 
    times_is(V,Z,W) <- times_is(X,Y,V) , times_is(Y,Z,U) , times_is(X,U,W) . 
    times_is(X,U,W) <- times_is(X,Y,V) , times_is(Y,Z,U) , times_is(V,Z,W) . 
\end{verbatim}
\end{center}
If you are interested in how PLOP went about translating chang3 into LOP--form, consult the chapter on PLOP.
Just to shed some light on this particular example:
The formula in predicate logic syntax has been translated into a LOP formula, that consists of
five clauses which have the general structure $\ldots \leftarrow \ldots .$
Clauses one to three state that there is an element $a_1$ such that $a_1$ is a left--identity (clauses 2 and 3)
but not a right--identity (clause 1); i.e. there is an element, namely $a_2$, for which $a_2 * a_1 \neq a_2$.
Clauses four and five are a restatement of associativity and left to the interested reader to work out.
\\
PLOP also created a file {\it chang3.lop\/} which can be looked at in an extra window instigated by
clicking the left button at the {\it Input formula\/}--button of the main--menu.
XTHEO supplies an extra window for the {\it .lop\/} files but not for the {\it .pl1\/} files since
clauses in the {\it .lop\/} files will be used in the actual proof. Also
many problems in ATP are already in LOP--form. If you had the Chang \& Lee classic mentioned above in your library
you could check that example 3 is given there in a form very close to LOP.
\paragraph{preproc}
Often a formula to be proven conveys a lot of information that will not be needed for a
particular proof. But proving a theorem is essentially a search process and surplus clauses
will generally result in a vast amount of futile search. Thus there has been some research 
into limiting the set of clauses to those that stand a chance to be needed for the proof.
In theorem--proving this effort is called pre--processing, hence the name of the module.
The best possible kind of preprocessing would be to first find a proof and then throw away
the surplus clauses, but this obviously misses the underlying idea completely. Preprocessing
is meant to effectively reduce the number of clauses without engaging in a search--process.
The preprocessing module can be called by clicking the left mouse button at the button marked
{\it preproc\/}. With chang3 preprocessing does not result in a reduction of clauses.
By the way, the preprocessing module generates a file {\it p\_chang3.lop\/} containing the clauses left
in the formula. So in the general case, if preprocessing led to any reduction, do not forget
to put the {\it p\_\/}--prefix in front of the name of the file in the panel of the main--menu.
\paragraph{wasm and inwasm}
The {\it inwasm\/}--module translates LOP--code into assembler--code which is subsequently translated
to machine--code (compiled) by {\it wasm\/}.
If you look at wasm's flags pop--up you will notice that some of the options are already ticked\footnote
{to tick or untick an option, clicking a mouse button after having moved the arrow into
the respective box will do the job.}.
Standardly wasm will be called with the {\it -t\/}--flag for output of a file {\it name\/}.tree containing a
description of the proof--tree that will be rendered into an actual graphical display by the {\it prooftree\/}--module.
To get a discription of the wasm--module and its flags (--or of any other module--) you can have
the manual--page displayed by choosing from the menu to appear at clicking at the {\it Manual\/}--button.
Now start first inwasm and then wasm.
\paragraph{setheo}
Setheo is a first order predicate logic theorem prover featuring completeness, consistency, speed,
and a variety of proof modi. The standard proof mode is iterative deepening by clauses; i.e. the
number of (copies of) clauses is iteratively increases by one, thus securing a complete search of the search tree.
Experience has shown that the time needed for proving a theorem varies enormously depending on the
proof mode.
The prover is activated by clicking at the button marked {\it setheo\/}, the result will be displayed in the
{\it tty--subwindow\/} below the panel. In our case, i.e. the group--example {\it chang3\/}, the result looked like this:

\begin{verbatim}
        Start proving ...
        elapsed time 16ms
        Last run at depth: 4
        ********SUCCESS********
        Number of inferences:		10
        Total number of inferences:	-1
        Number of fails:		108
        Instructions executed: 442
\end{verbatim}

So Setheo needed 16ms (0.016 seconds) to solve this problem. Some details in the output might be a
bit confusing. Thus {\it depth\/} is actually the depth of the search--tree plus one and the total number of
inferences is set to -1 in the present implementation.
\paragraph{prooftree}
If you are interested in the proof you can have a look at the actual proof the prover came up with.
Click at the {\it prooftree\/} button to have a canvas displayed with the graphical representation of the
prooftree on it.
For your convenience the proof tree is reproduced in figure~\ref{fig:proof-tree}.
%\begin{figure}[htb]
%\begin{center}
%\begin{picture}(110,65)
%\input{ /u1/johann/papers/setheo/man/proof-pic}
%\end{picture}
%jfjfj
%\caption{Proof tree}\label{fig:proof-tree}
%\end{center}
%\end{figure}

\bild{tree01.ps}{9cm}{15cm}{Proof tree}{fig:proof-tree}

A canvas is yet another type of window in the SUN menagerie. As the name suggests a canvas is a window
type for drawing objects in like, e.g. prooftrees. The interesting thing, however, is that a canvas can be
much larger than the actual window. So with a large prooftree you might only see a part of the tree
at a given time, but you can have the canvas moved under your window like you can move a probe under
the microscobe. The way to do this is to use mouse clicks at the shaded bars to the left and on top
of the canvas. Left mouse--clicks cause the canvas to be moved left or downwards respectively, right mouse--clicks
have the canvas moved right or upwards. The amount of movement at each click can be controlled via
the cursor's position on the bar. The further you swerve to the right or downwards the larger the steps
will get.
But for now, we need not be concerned about this since our proof fits snugly into its frame.
What you might be wondering about, though, are the strange bullets and diamonts on the canvas.
Why not get a legend? Just click the {\it settings\/} button, make the {\it no\/} on the {\it legend\/} option
change to {\it yes\/} by clicking right in the middle of the diagram consisting of two arrows, hit
the {\it Done\/} button and then have the canvas redisplayed by clicking at the {\it Redisplay\/} button.
The legend should appear in the upper left corner of the window. 
Intuitively the root of the prooftree is what you want to show and the
leaves of the tree are the hard facts. 
Every symbol in the tree stands for a literal, the integers preceding a sequence of symbols stand for
the number of the clause\footnote{As determined by the succession of clauses in the {\it .lop\/} input 
file.} and its incarnation\footnote{clauses can occur arbitrarily often in a proof} respectively.
If you open the {\it Substitution\/} window (- you will probably know how to by now -) and click at
the number of a clause the uninstatiated clause will be displayed; a click at a symbol standing for
a literal will show you both the uninstantiated and instantiated literal.
Now we can reconstruct the proof, retranslating times$\_$is, and sustituting $\,^{-1}$ for f\_1, e for a$\_$1 and
a for a$\_$2.

Setheo first showed that $a^{-1^{-1}} * e = a$ using the facts $a^{-1^{-1}} * a^{-1} =e$ and $a^{-1} * a = e$
and $ e * a = a$ and the instantiation of rule 5: if $a^{-1^{-1}} * a^{-1} =e$ and $a^{-1} * a = e$ and $e * a = a$ then
$a^{-1^{-1}} * e = a$.\\
Setheo actually showed this twice.\\
It than showed that $a * e = a$ using $a^{-1^{-1}} *e = a$ twice, the fact $e * e = e$, and the instantiation of rule 4:
if $a^{-1^{-1}} * e = a$ and $e * e = e$ and $a^{-1^{-1}} * e = a$, then $a * e = a$.
So what did we show? We presumed that there is (at least one) left--identity
which is not a right--identity. Under this assumption we could show that
any left--identity picked at random is a right--identity~\footnote{We actually
picked $e$ but there is nothing special about $e$ to make it behave
differently from any other left--identity}. So we come up with a 
contradiction. Since there is nothing wrong with the axioms, our assumption 
that there exists a left--identity which is not a right--identity
must be a mistake. This concludes the proof.

Proving this theorem obviously was a very easy task for setheo and it only used a tiny fraction of its 
reasoning power. So we did not need built-in predicates and there did not occur any reduction steps in the proof.
Reduction steps are needed, e.g. for proving that Sue will be at your party from knowing that:
(1) if Martin does not come, Sue will be there, and
(2) if Sue does not come, Martin won't come.
%\ssubsection{A more advanced example}
%Let's do one more example and concentrate on the proof rather than on SUN's
%windowing environment~\footnote{Actually, the environment is called SUNVIEW,
%being a more advanced version of X--WINDOWS, the standard of the UNIX
%X--OPEN group.}.
%The theorem we want to prove is $$There is no largest prime.$$
%Or, which is the same thing, there exist infinetly many primes.
%The proof is attributed to Euclid (300 B.C.). Again, the proof is
%indirect: Assume there is a largest prime and call it $P$. Then, since
%the property of being prime is decidable for any number you can
%collect all primes smaller than $P$, form the product of all these
%primes including $P$.






%
%
%
%
%
%
%
%
%
%
%
%
%
%
%

\paragraph{Further hints}
\begin{itemize}
\item A menu displayed on clicking at the {\it examples\/} button lets you choose among a number of
shellscripts running the complete inwasm-wasm-setheo cycle. The shellscript are stored in files with
extension {\it .pr\/}. 
\item SUN's windows can be moved and/or enlarged. Clicking the {\it right\/} button on the black bar atop
each window will give you the respective options.
\item The canvas settings allow also for a generally more dense display of the prooftree possibly using a
smaller font. This might be useful if you want to get a birds--eye view on the whole proof.
You can also open a second canvas with the the same proof on it using one to get an overview and
the other to look at details.
\item Code produced with {\it PLOP\/} might not always be understood by the {\it inwasm\/}--module. In case of problems
check whether:
\begin{itemize}
\item In case there is one query this should be the first clause in your {\it .lop\/}--file. It is the first
clause if it had been the first clause in your {\it .pl1\/}--file.
\item In case there is more than one query, automatic query generation should be {\it on\/} in the
inwasm options.
\end{itemize}
\item If you want to prove theorems the files of which are not in the current directory,
you can use the {\it Directory\/} entry on the {\it panel\/} to enter the paths.
But beware!
In case any modules do not find input files make sure that you are allowed to write in the
respective directories. The setheo modules try to write output files in the same directory where
they got input files from. 
\item Be sure to set your UNIX path variable so you can access the xtheo modules after installation.
\end{itemize}
%\ssubsection{BUGS}
%There are problems with the prooftree output  by setheo. Most of the times the information is not
%sufficient for the construction of a tree. If a prooftree can be constructed there are frequent
%mistakes in the succession of literals and their instantiation. Often variables are not fully
%instantiated as this would afford an additional run trough the prooftree. $\not\models$
