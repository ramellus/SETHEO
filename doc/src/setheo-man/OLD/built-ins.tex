Notes:
\begin{itemize}
\item
currently all references to the index are done via the LCB,
i.e., by giving the literal-number, and NOT the term itself.
\end{itemize}

\subsection{genlemma/2}

\begin{description}
\item[Synopsis:]
	{\tt genlemma(Nu1, Nu2)}
\item[Parameters:]\ \\
	\begin{description}
	\item[{\tt Nu1}: Number]\ \\
	If this number is $\neq 0$, the generation of a unit-lemma
	is considered. Otherwise, this predicate just succeeds.
	\item[{\tt Nu2}: Number]\ \\
	If a unit-lemma is stored in the index, this number will be
	stored together with the lemma. This information can the
	be used when unit-lemmata are being used (e.g., by {\bf uselemma/1}).
	\end{description}
\item[Result:]\ \\
	This built-in predicate always succeeds. Memory overflow
	will result in a run-time error.
\item[Low Level Name:]
	{\tt genlemma argvector}

\item[Description:]\ \\
This built-in generates a unit-lemma {\tt H<-.} for the
head {\tt H} of the current clause, if all of the following
conditions hold:
\begin{itemize}
\item
both arguments are instantitated to a number,
\item
the value of the first argument is $\neq 0$,
\item
The solution of the head of the current subgoal does not involve
any reduction steps above the current node in the tableau.
(This means, that we really have a unit-lemma).
\item
The current head (with its current instantiation) is not subsumed
by any unit-lemma in the lemma store.
\end{itemize}

The lemma will be annotated by the value of the second argument
(must be a number).

If the unit-lemma subsumes one or more unit-lemmata in the lemma-store,
these unit-lemmata are deleted.

\item[Side-effects:]\ \\
If the {\bf sam(1)} is called with the inline-command flag {\tt -lemmatree},
the proof tree for the current lemma (if it gets stored in the index)
is appended to the proof-tree file.

\item[Notes:]\ \\
The {\bf genlemma/1} built-in must be the last subgoal of a given
clause or contrapositive.
Otherwise, correctness of SETHEO is not ensured.

\item[Example1:]\ \\
\begin{verbatim}
q(X,Y) <- p(X),r(X,a),genlemma(1,5).
\end{verbatim}

In this example, a unitlemma {\tt q(X,Y)<-.} --- with the current substitutions of {\tt X} and {\tt Y} --- is generated and labeled with the number $5$.

\item[Example2:]\ \\
\begin{verbatim}
q(X,Y) <- p(X),r(X,a),getdepth(D),genlemma(1,D).
\end{verbatim}

In this example, the generated unit-lemma is labeled with the current
depth. When lemmata are selected for usage ({\bf uselemma/1}), one
can select only those unit-lemmata which have been generated on a lower
depth.
\end{description}

\subsection{genlemma/2}

This built-in generates a unit-lemma {\tt H<-.} for the
head {\tt H} of the current clause, if all of the following
conditions hold.
The lemma will be annotated by the value of the second argument
(must be a number).

\begin{itemize}
\item
both arguments must be instantitated to a number,
otherwise, a run-time error is generated.
\item
the value of the first argument is $\neq 0$,
\item
The solution of the head of the current subgoal does not involve
any reduction steps above the current node in the tableau.
(This means, that we really have a unit-lemma).
\item
The current head (with its current instantiation) is not subsumed
by any unit-lemma in the lemma store.
\end{itemize}

In all cases, this built-in predicate succeeds or generates a run-time
error.

If the unit-lemma subsumes one or more unit-lemmata in the lemma-store,
these unit-lemmata are deleted.

The {\bf genlemma/1} built-in must be the last subgoal of a given
clause or contrapositive.

Example:
\begin{verbatim}
q(X,Y) <- p(X),r(X,a),genlemma(1,5).
\end{verbatim}


\subsection{genulemma/3}

This built-in generates a unit-lemma {\tt L<-.} for a literal
{\tt ~L} in the current clause. Its number is given as the first argument.
The lemma is generated, if all of the following
conditions hold.
The lemma will be annotated by the value of the third argument
(must be a number).

\begin{itemize}
\item
all arguments must be instantitated to a number,
otherwise, a run-time error is generated.
\item
the value of the first argument is $\neq 0$,
\item
The current instantiation of the given literal is not subsumed
by any unit-lemma in the lemma store.
\end{itemize}

In all cases, this built-in predicate succeeds or generates a run-time
error.

If the unit-lemma subsumes one or more unit-lemmata in the lemma-store,
these unit-lemmata are deleted.

Notes:
\begin{itemize}
\item
The range of the first argument is NOT checked
\item
it is not checked, if the solution of the given literal required
reduction steps above the current node in the tableau.
\item
All built-in literals (including this one) count as literals
\end{itemize}



Example:
\begin{verbatim}
q(X,Y) <- p(X),genulemma(2,1,5),r(X,Y).
\end{verbatim}

generates a lemma {\tt p(a)<-.}, if X is instantiated to {\tt a}.


\subsection{uselemma/1}

This built-in tries to use lemmata from the lemma-store.
This built-in must be given within a special clause, as shown
in the following example:

\begin{verbatim}
...
p(a,b) <-..
p(b,c) <-..

p(_,_) <- uselemma(5),fail.

p(e,f) <-.
..
\end{verbatim}

If the clauses are not reordered, SETHEO first tries the alternatives
{\tt p(a,b)..} and {\tt p(b,c)..}.
Then it extracts all lemmata from the lemma store, which
\begin{itemize}
\item
are unifiable with the current subgoal $\neg p(\ldots)$, and
\item
are annotated by a numeric value which is {\em smaller} than the
argument of {\bf uselemma}.
\end{itemize}

These lemmata are pushed upon the stack as additional alternatives
which are tried, before {\tt p(e,f)..} are tried.

This built-in always succeeds. If the pushed lemmata and the remaining
alternatives are to be tried, this built-in must be directly followed
by a {\bf fail}.

\subsection{getnrlemma/1}

This built-in unifies the given argument with the number
of lemmata which are currently stored (and not deleted) in the
lemma store.

Example:
\begin{verbatim}
p(X) <- getnrlemmata(Y), X > Y.
\end{verbatim}

\subsection{dumplemma/0}

This built-in dumps all lemmata in the lemma-store
onto the current output-file (as set by {\bf tell/1}), or on
stdout.
Lemmata are printed in a LOP-like syntax. Therefore, the output can directly
be processed by {\bf inwasm(1)}.
Lemmata which have been marked deleted are preceeded by a {\bf \#}.

Example:
\begin{verbatim}
printlemma <- tell("out"),write("Lemmata:\n"),
              dumplemma,told.
\end{verbatim}

\subsection{dlrange/2}

This built-in dumps all lemmata in the lemma-store
with annotated values between the first and the second parameter
(inclusive; both parameters must be instantiated to numbers)
onto the current output-file (as set by {\bf tell/1}), or on
stdout.
Lemmata are printed in a LOP-like syntax. Therefore, the output can directly
be processed by {\bf inwasm(1)}.

Example:
\begin{verbatim}
printlemma <- tell("out"),write("Lemmata:\n"),
              dlrange(3,4),told.
\end{verbatim}

\subsection{delrange/2}
This built-in deletes all lemmata in the lemma store
with annotated values between the first and the second parameter
(inclusive).
Both parameters must be instantiated to a number.

\subsection{checklemma/1}
check, if the lemma would be stored at all:
argument unified with 0 if not.
\subsection{checkulemma/2}
similar to genulemma
check, if the lemma would be stored at all:
argument unified with 0 if not.

\subsection{addtoindex/2}
add literal unconditionally to index

\begin{verbatim}
<-... checklemma(X),addtoindex(1,X,5).

same as 

<-... genlemma(1,5).
\end{verbatim}

\subsection{clearindex/0}
clears the entire index
\subsection{clearlemma/2}
delete all entries, which are unifiable with the given literal

\section{Non-Backtrackable Counters}

\subsection{counters/1}
This built-in allocates $N$ non-backtrackable counters. They are
initialized to the numeric value $0$.
Note, that this built-in must be used prior to any other computation
(but after the {\bf galloc}).

\subsection{setcounter/2}

The built-in {\bf setcounter(I,N)} destructively sets the counter $I$
to the numeric or constant value $N$. A run-time error occurs, if
the parameters are not of the required type.
The validity of $I$ is {\em not\/} checked.

Example:
\begin{verbatim}
p(U) <- X is $D + U, setcounter(3,X).
\end{verbatim}

\subsection{getcounter/2}

The built-in {\bf getcounter(I,N)} unifies the second parameter $N$
with the value of the $I$'s counter.

Example:
\begin{verbatim}
p(U) <- getcounter(3,X), U is X + 1.
\end{verbatim}

