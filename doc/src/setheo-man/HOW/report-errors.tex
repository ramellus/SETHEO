%%%%%%%%%%%%%%%%%%%%%%%%%%%%%%%%%%%%%%%%%%%%%%%%%%%
%   SETHEO MANUAL
%	(c) J. Schumann, O. Ibens
%	TU Muenchen 1995
%
%	%W% %G%
%%%%%%%%%%%%%%%%%%%%%%%%%%%%%%%%%%%%%%%%%%%%%%%%%%%

\subsection{Reporting SETHEO Errors}
\label{subsec:reporting-errors}

If the SETHEO-system behaves strangely (e.g., core dump, irregular
output, funny error messages) and it is suspected that one or more
modules of the SETHEO system contain a bug, the user is
adviced to report this bug to the SETHEO's development group
in Munich.
Before doing so, the user should carry out the following steps.

\begin{enumerate}
\item
Check,
if an error message is produced by one (or some) of the modules.
If an error is reported by one of the modules, one should not run the
others with faulty output. E.g., if the inwasm reported an error,
the assembler wasm and the sam should not be executed, even if code has
been produced.
\item
Check,
if the formula file is non-empty.
\item
Check,
if at least one start clause (with a {\tt <-} or a {\tt ?-}
in the beginning)
is present (and not being removed by inwasm's purity reduction).
\item
Check the command-line parameters used or this example.
Does the error occur with a different parameter setting as well?
\item
Check, if the error is reproducable.
\item
If logic-programming features of LOP and built-ins are used:
is the subgoal-reordering (inwasm {\tt -sgreord})
and the clause reordering (inwasm {\tt -clreord})
turned off where necessary?
Do all built-in predicates start with a {\tt\$}?
Be aware of the fact that all clauses with a {\tt <-} separator
are being fanned (if there is at least one Non-Horn clause in the formula)
and their subgoals are reordered.
When subgoals are reordered, built-in predicates are placed
at the beginning of the clause.
\item
If logic-programming features of LOP and built-ins are used:
are the required variable bindings for the built-in predicates OK?
E.g., a {\tt X is Y + 1} requires {\tt Y} to be instanitated to
a numeric value.
\item
Try to minimize the example by leaving out clauses or literals or
by simplifying them, until the error disappears.
The smaller the resulting formula, the easier it is to detect a possible
error (both for you and the SETHEO development group).
\end{enumerate}

If all of the above steps have been tried to no avail,
send in your error report to
\begin{center}
{\tt setheo@informatik.tu-muenchen.de}
\end{center}
The report should contain:

\begin{enumerate}
\item
Name and address of the user,
\item
Version number of the SETHEO system (any modifications of the
code by the user?),
\item
Hardware platform and version of the operating system,
\item
a short description of the error,
\item
a complete set of parameters which have been used to produce the
error,
\item
a run-time protocol (e.g., the {\tt .log} file), and
\item
the formula which produced the error (as small as possible).
\end{enumerate}

The SETHEO development group in Munich then will have a look at the
problem and will try to solve it ASAP.
