%%%%%%%%%%%%%%%%%%%%%%%%%%%%%%%%%%%%%%%%%%%%%%%%%%%
%   SETHEO MANUAL
%	(c) J. Schumann, O. Ibens
%	TU Muenchen 1995
%
%	%W% %G%
%%%%%%%%%%%%%%%%%%%%%%%%%%%%%%%%%%%%%%%%%%%%%%%%%%%

\section{Assembly and Display of Answer Substitutions}

In contrast to PROLOG systems, SETHEO does not automatically
display answer substitutions of the variables in the query of
a logic program or a theorem.
Answer substitutions, however, are always present in the proof,
SETHEO generates. They can be displayed, using the {\bf xptree}
tool.

A more convenient and elegant way to display the answer substitutions
(both for the Horn and Non-Horn case) can be easily accomplished, using
SETHEO's logic programming facilities.

In the following, we will describe, how it is possible to print
one answer substitution or all substitutions in the Horn and Non-Horn case.

\subsection{The Horn Case}

For a formula (or logic program) which only consists of Horn-clauses,
the answer substitution consists of exactly one substitution for each
variable in the query.
E.g., for a query {\tt <- p(X,Y)}, an answer substitution could be
$ X\backslash a, Y\backslash b$. 

Such a query can easily be instrumented in such a way that such
answer substitutions are printed, as soon as a proof could be found.
We just print the substitution, using the {\em \$write/1\/} built-in.
In our above example, one would have to write:
\begin{center}
{\verb+?- p(X,Y),$write("X="),$write(X),$write(" Y="),$write(Y),$write("\n").+}
\end{center}

Please note that the {\tt <-} now has been converted into a {\tt ?-}
in that clause in order to prevent reordering of subgoals 
(i.e., to prevent the output of the variables before the main goal is called).

If you want to print all answer substitutions whcih can be found within
given bounds, you have to add a {\tt \$fail} at the end of the clause.
This forces SETHEO to backtrack, after a proof has been found.

\subsection{The Non-Horn Case}

For a non-horn formula, the answer substitution can be {\em disjunctive}.
As an example, consider the following (non-Horn formula):

\begin{verbatim}
<- p(X).
p(a); p(b)<-.
\end{verbatim}
The disjunctive answer substitution in that case is
\[ X = a \vee X = b. \]

If we want to print such an answer substitution, our query must be transformed
as follows:
\begin{verbatim}
?- $ANS := [], query, $write("X= "), $write($ANS), $Write("\n").  /*(1)*/

query :- p(X), $ANS := [X|$ANS].
~p(X) :- $ANS := [X|$ANS].
\end{verbatim}
The rest of the formula remains unchanged.
When SETHEO is started and finds a proof, it prints the list of
all disjuncts of variable substitution. In our case, we get:
\verb+X= [a|[b|[]]]+.

If a more convenient output is needed, however, additional clauses
must be added in order to convert the list into a correct form.
Note, that additional built-ins are needed to bypass the bounds, and
to keep the number of inferences in the proof correct.
The following program produces the desired output:

\begin{verbatim}
?- $ANS := [], 
      query, 
      $getinf(I), 
      print_answersubst($ANS), 
      I1 is I -1,
      $setinf(I1).  /*(1)*/
 
query :- p(X), $ANS := [X|$ANS].
~p(X) :- $ANS := [X|$ANS].

p(a);p(b)<-.

print_answersubst([X]) :- $write("X="),$write(X),$write("\n").
print_answersubst([X|Y]) :- 
        $setdepth(1),$setinf(1),
        $write("X="), $write(X), $write(" ; "),
        print_answersubst(Y).
\end{verbatim}

If more than one query exists in the formula, this piece of code
can easily be generalized.

