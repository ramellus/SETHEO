%%%%%%%%%%%%%%%%%%%%%%%%%%%%%%%%%%%%%%%%%%%%%%%%%%%
%   SETHEO MANUAL
%	(c) J. Schumann, O. Ibens
%	TU Muenchen 1995
%
%	%W% %G%
%%%%%%%%%%%%%%%%%%%%%%%%%%%%%%%%%%%%%%%%%%%%%%%%%%%

% new getting started
%
After you have fetched and installed SETHEO on your system,
you are ready to go.
In this chapter, we present three tutorials which shall enable you to
successfully use the SETHEO system.

The tutorials are suited for different levels of experience in
automated theorem proving and thus have  different prerequisits.

The first tutorial starts with a problem, given in mathematical
notation. In a step by step fashion, all steps necessary from entering
the formula to understanding SETHEO's proof will be described.
The only prerequisits needed to understand this tutorial is a general
knowledge of logic and mathematics.

The second tutorial aims at readers who are somewhat familiar 
with the Model Elimination calculus and the basics of automated theorem
proving. Given a formula in first order logic (or as a set of clauses)
(in our tutorial, it will be the same formula as in the first tutorial),
the user is guided to explore the different possibilities to set
parameters and watch the influence the setting has on the search
for the proof.

The third tutorial assumes familiarity with the basics of SETHEO
and logic. It describes the steps which are necessary to go from
given proof tasks from an application towards the usage of SETHEO.
For each step, practical hints will be given and problems which might
(or will) occur will be discussed.

A number of exercises for the user of SETHEO concludes this chapter.
