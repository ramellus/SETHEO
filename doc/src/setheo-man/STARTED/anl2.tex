\begin{figure}[htb]

\psset{levelsep=8ex,linewidth=0.5pt,treefit=loose,nodesepB=4pt}

\begin{center}

\parbox{7cm}{
\pstree{\TR{}}
       {\pstree[linestyle=dotted,nodesepA=4pt]
               {\TR{$\neg p(X), [X \neq a]$}}
               {\TR{}}
        \pstree[linestyle=dotted,nodesepA=4pt]
               {\TR{$\neg q(X)$}}
               {\TR{}}}
}%
\parbox{7cm}{
\pstree{\TR{}}
       {\pstree[nodesepA=4pt]
               {\TR{$\neg p(b), [b \neq a]$}}
               {\TR{$p(b)$}}
        \pstree[nodesepA=4pt]
               {\TR{$\neg q(b)$}}
               {\TR{q(b)}}}
}

\end{center}

\caption{The use of Antilemmata leads to the generation of
         an Antilemma-Constraint $X\neq a$ after the first fail. So
         after backtracking recomputation of the solution $X=a$ is
         avoided. Instead immediately the the solution $X=b$ is
         obtained.}  
\label{fig:anl2}
\end{figure}
