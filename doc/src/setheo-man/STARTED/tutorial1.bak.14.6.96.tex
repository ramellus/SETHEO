%%%%%%%%%%%%%%%%%%%%%%%%%%%%%%%%%%%%%%%%%%%%%%%%%%%
%   SETHEO MANUAL
%	(c) J. Schumann, O. Ibens
%	TU Muenchen 1995
%
%	%W% %G%
%%%%%%%%%%%%%%%%%%%%%%%%%%%%%%%%%%%%%%%%%%%%%%%%%%%

\section{Tutorial~1}

Assume you have been given the following problem to prove\footnote{
	The problem has been taken from \cite{PELLAAR}. A formulation
	of that problem is also present in the TPTP \cite{SSY94} as problem
	{\tt MSC006-1}.}:
\begin{quote}
Let $p$ and $q$ be two binary relations.
For the relations, we know the following properties:
$p$ and $q$ are transitive. Furthermore, $q$ is symmetric.
Additionally, we know that for each pair of elements, the pair is
either in relation $p$ or in relation $q$.

Now, we have to show that at least one of the relations $p$ or $q$
is total.
\end{quote}

As a first step, we have to write down this problem in a formal
notation, expanding the definitions of ``transitive'', ``symmetric'',
and ``total''. 
A relation\footnote{
	With $XrY$ we denote that $X$ is in relation $r$ with $Y$.
	}
$r$ is transitive, if
\[ \forall X,Y,Z \mbox{\ \ \ } XrY \wedge YrZ \rightarrow XrZ,\]
it is symmetric, if
\[ \forall X,Y \mbox{\ \ \ } XrY \rightarrow YrX,\]
and total (i.e., all elements are in the relation $r$), if
\[ \forall X,Y \mbox{\ \ \ } XrY.\]

Now, given the formal definitions of all properties, you are invited
to take a sheet of paper and try to prove the theorem.

Did you make it?
If not, don't be desparate. There is a proof in Table~\ref{tab:started:nonob-proof} at the end of this Chapter and furthermore, there is SETHEO to
support you.

In order to use SETHEO to automatically find a proof, you have to write
down the problem in first order predicate logic (FOL).
In general, such a formula is of the form
\[ 
{\cal A}_1 \wedge \ldots  \wedge {\cal A}_1 \rightarrow {\cal T}
\]
where ${\cal A}_i$ are Axioms (in our case, the properties of the relations
$p$ and $q$, and ${\cal T}$ is the theorem we want to show.
When we insert the definitions, we obtain:

\[
\begin{array}{lll}
(( \forall X,Y,Z  & XpY \wedge YpZ \rightarrow XpZ ) & \wedge \\
( \forall X,Y,Z  & XqY \wedge YqZ \rightarrow XqZ ) & \wedge \\
( \forall X,Y  & XqY \rightarrow YqX )& \wedge \\
( \forall X,Y  & XpY \vee XqY)) &  \\
& \rightarrow & \\
(( \forall X,Y  & XpY ) & \vee \\
( \forall X,Y  & XqY )) & \\
\end{array}
\]

Before starting SETHEO, however, two further steps must be done:
(1) the syntax must be changed, and
(2) the formula must be converted into Conjunctive Normal Form (CNF).
Although the second step can be performed automatically, we will 
give hints, how our formula is converted into CNF.
The transformation of an arbitrary formula into CNF is described in detail
in many textbooks. Here, we refer to \cite{Lov78,Chang75}.
The transformation consists of essentially two steps: 
removing all logical operators which do not belong to CNF, and
elimination of the quantifiers by skolemization.

\subsection{Preparing the formula}
Since SETHEO does not understand \LaTeX\ and infix mathematical
notation, we have to modify the syntax according to the following
rules.
\begin{itemize}
\item
SETHEO does not understand infix notation (well, there are some exceptions).
Therefore, instead of $XrY$ we have to write our binary relation $r$
as a predicate of arity two: {\tt r(X,Y)}\footnote{
	Actually, there is quite a number of different ways to
	convert the syntax. E.g., one could have written instead:
	{\tt isInRelation(r,X,Y)}. For further details see our third
	tutorial.}.
This predicate has the meaning: $\mbox{{\tt r(X,Y)}} \equiv {\rm\bf T}$
iff $X$ and $Y$ are in the relation $r$.

\item
All variables must start with an uppercase letter or an underscore
e.g.,{\tt X, Y, \_99, \_}, all
predicate-, function-, and constant symbols must start with a lowercase
letter (e.g., {\tt r, f(X)}.
\item
The connectives and qunatifiers must be changed into an ASCII representation
(e.g., $\forall$ must be written as {\tt forall}).
\end{itemize}

For details on the FOL syntax of SETHEO see Chapter~\ref{chap:file-formats}.
If you choose to do the conversion of a formula into CNF by hand,
all steps of transforming the syntax execpt the last one apply.

After applying all syntactic sugar to our formula, we obtain:
\begin{verbatim}
( 
  ( forall X,Y,Z p(X,Y) & p(Y,Z) -> p(Y,Z)) &
  ( forall X,Y,Z q(X,Y) & q(Y,Z) -> q(Y,Z)) &
  ( forall X,Y   q(X,Y) -> q(Y,X)) &
  ( forall X,Y   p(X,Y) ;  q(X,Y)) 
) -> (
  ( forall X,Y   p(X,Y)) ; 
  ( forall X,Y   q(X,Y)) 
)
\end{verbatim}

This formula must be kept in a file with the extension ``.pl1''
(for 1st order predicate logic). Our example is contained in file
{\em nonob.pl1}\footnote{
	In your SETHEO distribution, this file as well as the other
	examples described in this manual can be found in the
	directory {\tt \$SETHEOHOME/examples}.}.
Now, we are ready to start the SETHEO system. In a first step, 
we have to convert this formula into CNF (i.e., a set of clauses).
This is done using the module {\bf plop}, by typing:
\begin{center}
{\bf\tt plop -neg nonob}.
\end{center}

The {\tt -neg} option must be given, because SETHEO's proof mode is
a {\em refutation\/} procedure. Instead of proving out theorem, 
SETHEO tries to refute it. 

After invoking {\bf plop},
a lot of data are printed on the screen, until finally, the message
\begin{center}
{\tt Erzeugte Klauseln: 6} 
\end{center}
(number of generated clauses is 6) apprears.
Above this message, these clauses are printed on the screen.
Additionally, these clauses have been written into the file
{\em nonob.lop}.
For reference, the resulting 6 clauses are shown below (Table~\ref{tab:started:tut1:lop}.
One can easily identify the clauses for transitivity, symmetry and the other
properties of the relations.
Please note that the clauses are written in LOP-syntax\footnote{
	The full definition of the LOP syntax can be found
	in Chapter~\ref{formats}.}
a PROLOG-like syntax, i.e., 
a clause with positive atoms $P_1,\ldots,P_n$ and negative atoms
$L_1,\ldots,L_m$ are written as
\[ P_1 {\tt ;} \ldots {\tt ;} P_n \mbox{\tt \ \ <-\ \ }
L_1 {\tt ,} \ldots {\tt ,} P_m{\tt .} \]
with $m,n \geq 0$.
Our theorem ($p$ or $q$ is total) has been negated and thus being
transformed into two clauses. The all-quantified variables in the 
theorem have been converted into the constants {\tt c\_1},\ldots,{\tt c\_4}
by Skolemization.
Please note, that all variables which occur in the clauses
(they are implicitly all-quantified) only range over one clause.

Although our formula now looks rather like a PROLOG program, it
actually is a non-Horn formula, as can be seen in the fourth clause.
This clause contains two positive atoms, separated by a {\tt ;}.
Horn-clauses, on the contrary, only may contain at most one
positive literal per clause.

\begin{table}[htb]
\begin{verbatim}
p(X,Z) <- p(X,Y),p(Y,Z).
q(X,Z) <- q(X,Y),q(Y,Z).
q(Y,X) <- q(X,Y).
p(X,Y) ; q(X,Y)<-.
<- p(c_1,c_2).
<- q(c_3,c_4).
\end{verbatim}
\caption{LOP formula for the ``Nonobviosness Problem''}
\label{tab:started:tut1:lop}
\end{table}

\subsection{Running SETHEO}

Given a set of clauses in the LOP-notation (Tavle~\ref{tab:started:tut1:lop}; either generated by hand
or by {\bf plop}), we are ready to start the prover itsself.
In our example, the clauses are in the file {\tt nonob.lop}.
Please, type {\tt setheo nonob} on your computer to start SETHEO.
Now, a lot of things happen (hopefully no ``command not found'', or
``core dumped'') and a huge amount of output is printed on the
screen. Here, it is advisable to use a command-screen with a scroll bar.

Of most interest for us at the moment is the message
\begin{center}
{\tt ***** SUCCESS *****}
\end{center}
 which should appear after a few seconds 
on the screen.
This means that SETHEO could refute the set of clauses, given in the
file {\em nonob.lop}, or equivalently, a proof has been found for
our theorem.

An abbreviated copy of the messages on the screen are given in following
(Missing lines are denoted by  {\tt \ldots}). For reference in the
text, the lines are numbered (the line numbers are not visible on
your screen)\footnote{
	Depending on your version of SETHEO, several messages
	can be stated differently, or different numbers may be
	present. This, however, is quite normal.
	}

\begin{verbatim}
01- inwasm V3.2.5 Copyright TU Munich (April 7, 1995) 
02- command line: /usr/local/jessen/DIR/setheo/inwasm -cons -foldup pellaar 
03- Parsing 
04- Preprocessing: 
         ...
05- Codegeneration.
06- pellaar.s generated in  0.07 seconds

07- wasm V3.2 Copyright TU Munich (March 25, 1994)
         ...
08- SAM V3.3 Copyright TU Munich (December 22, 1995)
09- Options : -dr -cons -dynsgreord 2 pellaar 
         ...
10- Start proving...
         ...
11- -d:    2  time     < 0.01 sec  inferences =        9  fails =       15
         ...
12- -d:    7  time =     0.62 sec  inferences =     6410  fails =     7741

13-                   ******** SUCCESS ************

14- Number of inferences in proof      :       20
15-         - E/R/F/L                  :       17/       2/       1/       0
         ...
16- Number of unifications             :     8465
         ...
17- Instructions executed              :    32093
18- Abstract machine time (seconds)    :     0.90
19- Overall time          (seconds)    :     1.05
\end{verbatim}


The SETHEO system processes the formula in three distinct steps:
\begin{enumerate}
\item
The input formula is being preprocessed (see tutorial~2 and
Section~\ref{sec:inwasm} for details), and assembler code for the
SETHEO abstract machine is generated (lines 1--6).
\item
Then, the code is assembled (line 7).
\item
Finally, the interpreter for the abstract machine \SAM\  is started.
This module actually tries to find the proof (lines 8--19).
\end{enumerate}

Each of the modules (which are separate programs) can be called with
a variety of command-line parameters. If you invoke the {\bf setheo}-command,
default settings are taken. For a description of the parameters and their
influence see our Tutorial~2 and Chapter~\ref{chap:basic-programs}.

After each of the modules has finished its work, it prints a line
which indicates the run-time of that module (e.g. line 6 and 19).
For the prover itself, we additionally give the time needed to
{\em search\/} for the proof (line 18). This run-time does not
take into account the overhead for book-keeping tasks, for loading the
formula and for printing the statistics.

The amount of search needed to find the proof can be seen at three
different points:
\begin{description}
\item[line 11--12]
	The prover does ``iterative deepeing''. This means that
	the search space is only searched until a certain limit
	is reached. If within these limits no proof can be found, the
	limit is increased and the search starts over again. Here,
	we limit the depth ({\tt -d}). For each level, the amount of
	time spent, the number of inferences, and the number of
	backtracking steps is given.
\item[line 16]
	This number gives the number of successful unifications which
	have been made during the search.
\item[line 17]
	This number is the amount of low-level abstract machine instructions
	executed to find the proof. This number is the most fain-grain
	measure of the work done by the \SAM. Please note however, that
	the execution time for each instruction differs.
\end{description}


The actual proof found by SETHEO consists of $20$ (Model Elimination)
Inference steps (line 14). Therr type and frequency
are given in line 15.
In order to unstand these figures and the proof itself, some
knowledge about the Model Elimination Calculus is necessary.
In the next section, we will give a short (informal) introduction into
the calculus. After that, we will have a detailed look at the proof
generated by SETHEO.

\subsection{The Model Elimination Calculus}

\subsection{Looking at the Proof}
\subsubsection{The graphical Display}
\subsubsection{The \LaTeX\ Output}

\begin{table}[htb]
proof for pellaar
\caption{proof for pellaar}
\label{tab:started:nonob-proof}
\end{table}

