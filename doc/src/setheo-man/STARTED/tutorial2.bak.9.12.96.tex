%%%%%%%%%%%%%%%%%%%%%%%%%%%%%%%%%%%%%%%%%%%%%%%%%%%
%   SETHEO MANUAL
%	(c) J. Schumann,, O. Ibens
%	TU Muenchen 1996
%
%	%W% %G%
% MOD:
%	11.9.96
%%%%%%%%%%%%%%%%%%%%%%%%%%%%%%%%%%%%%%%%%%%%%%%%%%%

\section{Tutorial~2}

For this tutorial, we assume the reader to be familiar with the
basics of the Model Elimination Calculus and SETHEO. This 
tutorial covers the influence of refinements and extensions
of the basic Model Elimiantion Calculus on the search behavior of SETHEO.
Again, we use an example to illustrate the techniques implemented in
SETHEO, in fact we use the same example as in the previous tutorial.
For reference, its clauses (in LOP syntax) are given below in 
Figure~\ref{tab:tut2:pellaar.lop}. This example is, of course, no
challenge problem for Automated Theorem Provers
(it used to be one about 10 years ago), but it has some nice 
properties.


\begin{table}[htb]
\begin{center}
\begin{verbatim}
(1)  p(X,Z) <- p(X,Y),p(Y,Z).
(2)  q(X,Z) <- q(X,Y),q(Y,Z).
(3)  q(Y,X) <- q(X,Y).
(4)  p(X,Y) ; q(X,Y)<-.
(5)  <- p(a_1,a_2).
(6)  <- q(a_3,a_4).
\end{verbatim}
\end{center}
\caption{LOP clauses of the Example (file: MSC006-1.lop)}
\label{tab:tut2:pellaar.lop}
\end{table}

For our first experiment, we run SETHEO, using the basic Model Elimination
Calculus, and performing depth-first iterative deepening over the
depth of the tabelau (A-literal depth)\footnote{
	Using different bounds
	will be discussed below in Section~\ref{sec:tut2:bounds}.}.
Such a run will be accomplished using the two commands:

\begin{center}
\begin{verbatim}
inwasm MSC006-1
sam -dr MSC006-1
\end{verbatim}
\end{center}

\begin{figure}[htb]
\begin{center}
\begin{verbatim}
SUCCESS
SETHEO-output for MSC006-1.lop
\end{verbatim}
\end{center}
\caption{Output of the \SAM\ for MSC006-1.lop}
\label{fig:tut2:pellaar.pure.log}
\end{figure}

After some time, SETHEO finds a proof, consisting of 18~inferences,
two of which are Model Elimination Reduction steps
(see Figure~\ref{fig:tut2:pellaar.pure.log} for SETHEO's output).
A graphical representation
of the tree can be obtained by using the tool {\bf xptree} (see
Chapter~\ref{chap:basic-modules} for details on {\bf xptree}).
%seen in Figure~\ref{fig:tut1:pellaar} in Tutorial~1.
For us of interest now, however, is not the proof itself, but the amount
of search involved to find the proof.
As shortly mentioned already in Tutorial~1, 
this is reflected in the following sizes and numbers:

\begin{description}
\item[Abstract Machine Time] 
This number gives the amount of time (CPU user time), the \SAM,
SETHEO's Abstract Machine, needed to find the proof.  
%However, on different machines, and even on different runs of the same
%problem, other values can be returned. 
However, on different machines, and even on different runs on the same
machine, this value may vary due to load-fluctuations of the machine
and due to inaccuracies of the time measuring. 
Therefore, this figure should be taken as a first approximation only.

\item[Total number of inferences $n_i$] gives the overall number of times,
a unification was tried during search (by trying to perform an
Extension or Reduction step). This number directly reflects the amount
of search performed, but does not take into account, how long each
attempted unification takes.

\item[Number of Fails $n_f$] is the number of times, a unification failed,
or a bound has been reached. This number is again splitted into those
two values.
\end{description}

The output SETHEO produces at the end of the run gives these values
for the overall search. Additionally, these values are also
printed, after the entire search space has been exhausted with a given
bound. Figure~\ref{fig:tut2:pellaar.pure.log} shows these figures
for the increasing depth-bound ({\tt -d}). As can be seen clearly,
the size of the search space grows substantially with each iteration
(at least exponentially).

In the following, we present several improvements of the Model
Elimination Calculus and SETHEO's search procedure and have a look
at these figures.
The list of improvements discussed in this tutorial
is only a small selections of techniques integrated into SETHEO.
For a list of all techniques and the corresponding command-line
switches see Chapter~\ref{chap:basic-modules}.
Furthermore, we do {\em not\/} present the theoretical background
nor any formal definitions or theorems. For such issues see,
e.g., \cite{LSBB92,LMG94,Letzdiss,Mayrdiss}.

\section{Static Refinements}
\subsection{Regularity}

One of the most powerful refinement of the pure Model Elimination
Calculus is its restriction to {\em regular\/} tableaux.

\begin{definition}[regular tableau]
A Model Elimination Tableau $\cal T$ is regular, if and only if on
each path from the root to a leaf, no literal occurs more than once.
\end{definition}

One can show (cf. \cite{LSBB92,LMG94}) that for each closed tableau
there also exists a closed regular tableau, i.e., we don't loose any
proofs if we are searching for regular tableaux only.

Figure~\ref{fig:tut2:reg-tab} contains two tableaux for the same
subgoal (taken from our example). The left tableau is not regular,
because the literal $\neg p(a,b)$ occurs twice in it; the right tableau
is regular. In our case, it is easy to see, why the left tableau is
not regular: in attempt to solve the goal $\neg p(a,b)$, an extension
step into the symmetry clause (clause number (3)) is made, yielding
a new subgoal $\neg p(b,a)$. Then this clause is used again to
obtain $\neg p(a,b)$. Comparing this subgoal to the original one,
we have gained nothing! Therefore, we can leave these two steps out,
yielding a regular tableau. Restricting the search to certain kinds
of tableaux thus reduces the search space considerably.

\begin{figure}[htb]
-figure-
\caption{Model Elimination Tableau and Regular Model Elimination Tableau}
\label{fig:tut2:reg-tab}
\end{figure}

Within SETHEO, there are two different ways to enforce regularity:
a {\em direct regularity check\/} which is performed, as soon as an
Extension or Reduction step is tried, and the generation of {\em
regularity constraints\/}. 

\noindent{\bf Direct Regularity Check.} This check can be
activated by calling {\bf inwasm} with the option {\tt -eqpred},
yielding the following sequence of commands:

\begin{center}
\begin{verbatim}
inwasm -eqpred MSC006-1
sam -dr MSC006-1
\end{verbatim}
\end{center}

The direct regularity check is realized as an {\em
equal-predecessor\/} check, i.e., when an Extension or Reduction step
is tried the path is searched for equal literals. Note that this is
only a restricted regularity check since {\em later\/} instantiations
to equal literals are not detected. 

When we look at the result SETHEO produces, a considerable 
reduction in the amount of search necessary to find the proof
can be seen.
Table~\ref{tab:tut2:results.regularity}, lines 1 and 2
shows typical figures, compared
to the basic calculus.

\begin{table}[htb]
\begin{center}
\begin{tabular}{|l|r|r|r||r|r|r|r|r|r|}
\hline
Method & $t_{\SAM}$ & $n_i$ & $n_f$ & 
	$n_i^3$ & $n_i^4$ & $n_i^5$ & $n_i^6$ & $n_i^7$ \\
\hline\hline
basic & 2.34 & 160606 & 25252 &
	25 & 145 & 2525 & 26262 & 282828 \\
\hline
{\tt -eqpred} & 12.40 & 117645 & 113085 &
	30 & 110 & 814 & 7492 & 109190 \\
\hline
{\tt -reg} & 4.37 & 35557 & 45162 &
	30 & 108 & 703 & 5037 & 29670 \\
\hline\hline
\end{tabular}
\end{center}
\caption{Search Space for Basic Model Elimination (line~1), 
and implementations of regularity. $n_i$ is the total number of inferences,
$n_f$ the total number of fails, and $n^k_i$ the number of inferences
on search level $k$.}
%(a ``--'' indicates that a proof has been found on a previous level).}
\label{tab:tut2:results.regularity}
\end{table}


\noindent{\bf Regularity Constraints.} 
A more elegant and powerful method, which provides a full
regularity check, is realized by syntactical
inequality constraints. These constraints of the form
$ [ X \not\in \{t_1,\ldots,t_n\}] $  are attached to variables (here $X$)
and checked permanently. As soon, as $X$ gets bound to a term,
its constraints are evaluated. If a violation occurs (i.e.,
$X$ gets instantiated to one of the $t_i$), backtracking
occurs within the \SAM\ Abstract Machine.
Looking at our above example of Figure~\ref{fig:tut2:reg-tab}
the regularity-checker would
generate the constraint $[ X,Y ] \not = [ b,a ]$. This constraint
would fire immediatedly, if the symmetry clause was tried for the
second time, causing the \SAM\ to fail and backtrack.

The results for running SETHEO with regularity constraints, which
is performed by the commands
\begin{center}
\begin{verbatim}
inwasm -reg MSC006-1
sam -dr -reg MSC006-1
\end{verbatim}
\end{center}
are shown in Table~\ref{tab:tut2:results.regularity}.
Although the figures for this example are quite similar for these
two methods of enforcing regularity, it is advisable to use regularity
constraints,
because it is (a) more powerful, but costs a little
more overhead, and (b) neatly fits into other Calculus Refinements
described below. As a general hint, it is {\em always\/} advisable to
activate the enforcement of regular tableaux. Note, that the
default parameters for {\bf setheo} incorporate this option.

\subsection{Tautology and Subsumption Constraints}

A Model Elimination Tableau can be further restricted, namely that
no instance of a clause in the tableau is a {\em tautology\/}, and
that no instance of a clause is subsumed by another clause.
Again, when these restrictions are enforced, no proofs are lost.
In SETHEO, these conditions are checked permanently using
the constraint mechanism. Constraints for checking for tautology and
sumbsumption are generated during the run of the {\bf inwasm} compiler,
if it is invoked with the {\bf -cons} or {\bf -taut -subs} options.
Looking at our example, we can easily detect instantiations of clauses
which will result in tautological clauses or clauses which are subsumed
by others. Let's have a look at clause (4) in 
Figure~\ref{tab:tut2:pellaar.lop} which expresses the symmetry of $p$.
If we instantiate $X$ to the same value as $Y$ (e.g., by calling this
clause with a subgoal $\neg p(a,a)$), we obtain the following
instatiation of that clause in the tableau
{\tt p(X,X) <- p(X,X)}. Obviously, this clause is tautologial and
its application does not lead us anywhere. Therefore, we can {\em forbid\/}
such an instantiation by adding the constraint {\tt [X] =/= [Y]}
to the clause.

In the case of the transitivity clauses (e.g., clause (1)), 
we obtain a similar sitation:
If two of the variables in that clause $X,Y,Z$ get instantiated to the
same value, the resulting clause is tautological, namely if
$X = Y$ or $Y = Z$. This can be forbidden, using two constraints.

Additionally, there exist instantiations, where this clause is subsumed
by clause (5) {\verb+<- p(a_1,a_2)+}, namely $X$ is instantiated
to $c\_1$ and $Y$ to $c\_2$, or $Y$ to $c\_1$ and $Z$ to $c\_2$.
In both cases, the resulting instance of the clause is subsumed by
clause (5).
Therefore, the following constraints are automatically added to
clause (1):

\begin{center}
\begin{verbatim}
p(X,Z) <-  p(X,Y), p(Y,Z)
    : [Y] =/= [Z]          /* tau      */, 
      [Y] =/= [X]          /* tau      */, 
      [X,Y] =/= [a_1,a_2]  /* sub by 5 */, 
      [Y,Z] =/= [a_1,a_2]  /* sub by 5 */.
\end{verbatim}
\end{center}

A list of all generated constraints can be obtained, when the {\bf inwasm}
is called with the option {\tt -lop}. In that case, a file 
{\tt {\em file}\_pp.lop} is generated which contains all contrapositives
of all clauses, the generated constraints as well as other information.

A variety of other refinements, working with constraints have been
developed and integrated into SETHEO (e.g., overlapping), 
but these will not be discussed
here. For details see Chapter~\ref{chap:manpages} and the Glossary.
Comparative run-times with our example is shown in 
Table~\ref{tab:tut2:results.constr}\footnote{
	For activating these constraints within the \SAM, the command
	{\bf sam} must be called with the option {\tt -st}}.

\begin{table}[htb]
\begin{center}
\begin{tabular}{|l|r|r|r||r|r|r|r|r|r|}
\hline
Method & $t_{\SAM}$ & $n_i$ & $n_f$ & 
	$n_i^3$ & $n_i^4$ & $n_i^5$ & $n_i^6$ & $n_i^7$ \\
\hline\hline
basic & 2.34 & 160606 & 25252 &
	25 & 145 & 2525 & 26262 & 282828 \\
\hline
{\tt -taut} & 2.03 & 16411 & 21166 &
	30 & 105 & 396 & 1946 & 13925 \\
\hline
{\tt -subs} & 2.34 & 160606 & 25252 &
	25 & 145 & 2525 & 26262 & 282828 \\
\hline
{\tt -subs -taut -reg} & 2.32 & 16325 & 21052 &
	30 & 101 & 380 & 1880 & 13925 \\
\hline\hline
\end{tabular}
\end{center}
\caption{Search Space for Basic ME and several comile-time
refinements. {\tt -cons} is the abbreviation for
	{\tt -reg -taut -subs -anl}.}
\label{tab:tut2:results.constr}
\end{table}

\subsection{Deletion of Links}

The search space for finding a Model Elimination proof is spanned
by the possible {\em connections\/} between literals of the clauses. The more
connections, the larger the search-space (which increases at least
exponential with their number). Therefore, any method which can
remove connections without sacrificing completeness of the search procedure
can lead to a considerable decrease in run-time needed to find a proof.

The method built into SETHEO tries to remove connections which
are of no use. This option can be activated by calling {\bf inwasm}
with {\tt -linksubs} or {\tt -rlinksubs}\footnote{
	Note that this option has to be set for the {\bf inwasm} only.}.
In this tutorial, wo won't describe in detail, how this preprocessing
step works. Rather, we explain what happens 
with selected clauses of our example.
Let us consider clause (3) (Symmetry of $q$). The second literal
(the tail literal) has several connections: some going to positive
$q$ literals of other clauses, and one to the head-literal
of our clause (3). The latter connection, we call it a 
``back-connection'',
is of interest. If clause (3) has been called from a subgoal
$\neg p(a,b)$, then our tail literal gets $\neg p(b,a)$. If we now follow
the back-connection, we end up with a new subgoal $\neg p(a,b)$ which
is exactly our original one. In this case, following the back-connection
does not make any sense\footnote{
	In this case, our situation can be detected by the
	regularity-contraints as well, but in general, these
	two methods are independent from each other.
	}.
Thus this connection can be removed without
harming completeness.

Within the rules of transitivity (clauses (1) and (2)), one
back-connection in each clause can be removed as well. 
Transitivity clauses are in particular harmful, because they have two
subgoals (the longer a clause, the more search space it induces), and
these subgoals always introduce new variables (in our case variable $Y$).

Results of experiments with deletion of links are shown in
Table~\ref{tab:tut2:results.linksubs}\footnote{
	The option {\tt -linksubs} differs from 
	{\tt -rlinksubs} in such a way that
	{\tt -rlinksubs} ignores connections into single-literal clauses 
	(facts), thus reducing the required run-time to find 
	removable clauses.}.

\begin{table}[htb]
\begin{center}
\begin{tabular}{|l|r|r|r||r|r|r|r|r|r|}
\hline
Method & $t_{\SAM}$ & $n_i$ & $n_f$ & 
	$n_i^3$ & $n_i^4$ & $n_i^5$ & $n_i^6$ & $n_i^7$ \\
\hline\hline
basic & 2.34 & 160606 & 25252 &
	25 & 145 & 2525 & 26262 & 282828 \\
\hline
{\tt -linksubs} & 52.42 & 564876 & 722977 &
	27 & 81 & 556 & 6437 & 557766 \\
\hline
{\tt -linksubs -cons} & 0.87 & 4634 & 4809 &
	27 & 72 & 213 & 868 & 3445 \\
\hline\hline
\end{tabular}
\end{center}
\caption{Search Space for Basic ME, removal of connections, and removal of connections
combined with constraints}
\label{tab:tut2:results.linksubs}
\end{table}


\section{Dynamic Constraints: Antilemmata}

The search mechanism of basic Model Elimination is based on enumerating all
possible solutions for the involved subgoals. Since in general
variables are shared, solutions found for one subgoal 
influence the solvability of the still unsolved subgoals. Thus,
the solution of an early subgoal may cause a fail in a later subgoal. 
In that case, a new solution must be computed for the first subgoal.

If, however, this solution is the same as the previous one,
nothing could be accomplished, because again it won't be possible to
solve the later subgoal.
Therefore, this situation must be detected and must lead to a fail.
If the recomputation of a solution
appears very often, e.g., thousands of times, 
a considerable amount of redundancy in the search is visible. 
See Figure~\ref{fig:anl1} for a simple example. 

%%%%%%%%%%%%%%%%%%%%%%%%%%
%\begin{figure}[htb]

\psset{levelsep=8ex,linewidth=0.5pt,treefit=loose,nodesepB=4pt}

\begin{center}

\parbox{8cm}{
$
\begin{array}{l}
    \begin{array}{l}
        \left.
        \begin{array}{l}
            \longleftarrow p(X), q(X).  \\
        \end{array}
        \right.
    \end{array}  \\
    \begin{array}{ll}
        \left.
        \begin{array}{c}
            p(a).  \\
            \vdots \\
            p(a).  \\
        \end{array}
        \right\}
        &
        100 \times
    \end{array} \\
    \begin{array}{l}
        \left.
        \begin{array}{l}
            p(b).  \\
            q(b).  \\
        \end{array}
        \right.
    \end{array}  \\
\end{array}
$
}%
\parbox{8cm}{
\pstree{\TR{}}
       {\pstree[nodesepA=4pt]
               {\TR{$\neg p(a)$}}
               {\TR{$p(a)$}}
        \pstree[linestyle=dotted,nodesepA=4pt]
               {\TR{$\neg q(a)$}}
               {\TR{\mbox{\wasy\char18}}}}
}

\end{center}

\caption{A simple example for the necessity of Antilemmata: Basic ME will
         compute the solution $X=a$,
         detect a fail and backtrack for 100 times, before the
         solution $X=b$ can be 
         computed.}  
\label{fig:anl1}
\end{figure}

%%%%%%%%%%%%%%%%%%%%%%%%%%


%The search mechanism of basic ME is based on enumerating all
%possible solutions for the involved subgoals. Since in general
%variables are shared, found solutions have influence on the solvability
%of the still unsolved subgoals. So the solution of an early subgoal
%may cause a fail in a later subgoal. 
%
%If after backtracking the same solution is computed again, this must
%of course again lead to a fail. If the recomputation of a solution
%appears very often, e.g., 100 times, this leads to a considerable
%amount of redundance. See Figure~\ref{fig:anl1} for a simple example. 
%\begin{figure}[htb]

\psset{levelsep=8ex,linewidth=0.5pt,treefit=loose,nodesepB=4pt}

\begin{center}

\parbox{8cm}{
$
\begin{array}{l}
    \begin{array}{l}
        \left.
        \begin{array}{l}
            \longleftarrow p(X), q(X).  \\
        \end{array}
        \right.
    \end{array}  \\
    \begin{array}{ll}
        \left.
        \begin{array}{c}
            p(a).  \\
            \vdots \\
            p(a).  \\
        \end{array}
        \right\}
        &
        100 \times
    \end{array} \\
    \begin{array}{l}
        \left.
        \begin{array}{l}
            p(b).  \\
            q(b).  \\
        \end{array}
        \right.
    \end{array}  \\
\end{array}
$
}%
\parbox{8cm}{
\pstree{\TR{}}
       {\pstree[nodesepA=4pt]
               {\TR{$\neg p(a)$}}
               {\TR{$p(a)$}}
        \pstree[linestyle=dotted,nodesepA=4pt]
               {\TR{$\neg q(a)$}}
               {\TR{\mbox{\wasy\char18}}}}
}

\end{center}

\caption{A simple example for the necessity of Antilemmata: Basic ME will
         compute the solution $X=a$,
         detect a fail and backtrack for 100 times, before the
         solution $X=b$ can be 
         computed.}  
\label{fig:anl1}
\end{figure}

%
%The above refinements can not prevent the prover from computing the
%same solution for several times. So {\em Antilemmata\/} had to be
%invoked which avoid the repetition of solutions: 
%During backtracking all backtracked instantiations are converted into
%Antilemmata. These forbid to compute the same instantiation again.
%The repetition-check is performed every time when a further
%Extension or Reduction step is tried (see Figure~\ref{fig:anl2}). 
%\begin{figure}[htb]

\psset{levelsep=8ex,linewidth=0.5pt,treefit=loose,nodesepB=4pt}

\begin{center}

\parbox{7cm}{
\pstree{\TR{}}
       {\pstree[linestyle=dotted,nodesepA=4pt]
               {\TR{$\neg p(X), [X \neq a]$}}
               {\TR{}}
        \pstree[linestyle=dotted,nodesepA=4pt]
               {\TR{$\neg q(X)$}}
               {\TR{}}}
}%
\parbox{7cm}{
\pstree{\TR{}}
       {\pstree[nodesepA=4pt]
               {\TR{$\neg p(b), [b \neq a]$}}
               {\TR{$p(b)$}}
        \pstree[nodesepA=4pt]
               {\TR{$\neg q(b)$}}
               {\TR{q(b)}}}
}

\end{center}

\caption{The use of Antilemmata leads to the generation of
         an Antilemma-Constraint $X\neq a$ after the first fail. So
         after backtracking recomputation of the solution $X=a$ is
         avoided. Instead immediately the the solution $X=b$ is
         obtained.}  
\label{fig:anl2}
\end{figure}


The refinements of the Model Elimination Calculus described in the
previous sections cannot prevent the prover from computing the
same solution over and over again. Thus, the concept of
{\em Antilemmata\/} had been developed and implemented to
avoid the repetition of solutions: 
During backtracking all instantiations of variables which are undone
are converted into
Antilemmata. These forbid to compute the same instantiation again.
The repetition-check is performed every time when a further
Extension or Reduction step is tried (see Figure~\ref{fig:anl2}). 

%%%%%%%%%%%%%%%%%%%%%%%%%
%\begin{figure}[htb]

\psset{levelsep=8ex,linewidth=0.5pt,treefit=loose,nodesepB=4pt}

\begin{center}

\parbox{7cm}{
\pstree{\TR{}}
       {\pstree[linestyle=dotted,nodesepA=4pt]
               {\TR{$\neg p(X), [X \neq a]$}}
               {\TR{}}
        \pstree[linestyle=dotted,nodesepA=4pt]
               {\TR{$\neg q(X)$}}
               {\TR{}}}
}%
\parbox{7cm}{
\pstree{\TR{}}
       {\pstree[nodesepA=4pt]
               {\TR{$\neg p(b), [b \neq a]$}}
               {\TR{$p(b)$}}
        \pstree[nodesepA=4pt]
               {\TR{$\neg q(b)$}}
               {\TR{q(b)}}}
}

\end{center}

\caption{The use of Antilemmata leads to the generation of
         an Antilemma-Constraint $X\neq a$ after the first fail. So
         after backtracking recomputation of the solution $X=a$ is
         avoided. Instead immediately the the solution $X=b$ is
         obtained.}  
\label{fig:anl2}
\end{figure}

%%%%%%%%%%%%%%%%%%%%%%%%%

Antilemmata can be invoked with the commands
\begin{center}
\begin{verbatim}
inwasm MSC006-1
sam -dr -anl MSC006-1
\end{verbatim}
\end{center}
The performance gain by Antilemmata is shown in
Table~\ref{tab:tut2:results.anl}. 

\begin{table}[htb]
\begin{center}
\begin{tabular}{|l|r|r|r||r|r|r|r|r|r|}
\hline
Method & $t_{\SAM}$ & $n_i$ & $n_f$ & 
	$n_i^3$ & $n_i^4$ & $n_i^5$ & $n_i^6$ & $n_i^7$ \\
\hline\hline
basic & 2.34 & 160606 & 25252 &
	25 & 145 & 2525 & 26262 & 282828 \\
\hline
{\tt -anl} & 15.12 & 117727 & 195984 &
	30 & 127 & 1625 & 15812 & 100124 \\
\hline
{\tt -cons} & 1.17 & 5350 & 6389 &
	30 & 100 & 359 & 1321 & 3531 \\
\hline
{\tt -cons -linksubs} & 1.17 & 3879 & 3996 &
	27 & 72 & 213 & 804 & 2754 \\
\hline\hline
\end{tabular}
\end{center}
\caption{Search Space for Basic ME, and for ME with Antilemmata}
\label{tab:tut2:results.anl}
\end{table}



\section{Additional Inference Rules: Folding Up and Folding Down}

A major source of redundancy in the Model Elimination Calculus is
the necessity to solve the same subgoals over and over again.
Not only during the search, the same proofs for subgoals occur
repeatedly, but also in the final tableau, the same subgoal can show
up more than once. 
When performing a proof manually, usually, the human defines a {\em lemma\/},
proofs it and then use it for each occurrence of the subgoal.

A similar approach can be made for the Model Elimination Calculus as well
(``Lemmaizing'', cf.\ \cite{METEOR}). As soon as a subgoal gets solved,
the solution is stored in a so-called lemma store from where it can be used
when the same or similar subgoals have to be solved later on.

This approach\footnote{
	The basic functionality for generating and using lemmata has
	been built into SETHEO by means of built-in predicates. 
	These are described in Chapter~\ref{chap:logprg}, but they
	have to be added manually to the input formula.},
however, carries several severe problems: 
during the search a huge amount of lemmas are generated and must be stored
in such a way that they can be retrieved easily. An unrestricted use of
all these lemmas usually lead to an explosion of the search space.
Therefore, effective methods for selecting ``good'' lemmas would be
necessary.

In order to avoid these problems, we have implemented a restricted
version of lemmas by the means of additional inference rules.
In contrast to ``real lemmas'' our restriction means that
\begin{itemize}
\item
	no lemma survives backtracking over subgoal which caused its
	generation,
\item
	variables within a lemma can be bound once only.
\end{itemize}
Nevertheless our restricted version is correct and sound and requires
only a minimal overhead within the \SAM\  (actually, only a few pointers
need to be adjusted).

Figure~\ref{fig:tut2:foldup} shows with the help of a small example,
how this mechanism for ``folding up'' works: let us assume, subgoal
$\neg p(a,b)$ just has been solved (The triangle below that subgoal
indicates a closed tableau), $p(a,b)$ now can be used as a lemma
which could be used, if similar subgoals (in our example $\neg p(X,b)$)
have to be solved later on. 
Instead of copying $p(a,b)$ into an extra piece of memory, we just
add $p(a,b)$ as a new node to the branch from the root of the
tableau to the current nodes (right hand side of the Figure). This
can be accomplished by adjusting a few pointers. Then, when a new subgoal
(in our case $\neg p(X,b)$) has to be solved, the branch can be simply
closed by one ME reduction step (dotted line). 

{\tt
In the Non-Horn case, 

figure

position of folded up literal

folding down
}

\begin{table}[htb]
\begin{center}
\begin{tabular}{|l|r|r|r||r|r|r|r|r|r|}
\hline
Method & $t_{\SAM}$ & $n_i$ & $n_f$ & 
	$n_i^3$ & $n_i^4$ & $n_i^5$ & $n_i^6$ & $n_i^7$ \\
\hline\hline
basic & 2.34 & 160606 & 25252 &
	25 & 145 & 2525 & 26262 & 282828 \\
\hline
{\tt -cons} & 1.17 & 5350 & 6389 &
	30 & 100 & 359 & 1321 & 3531 \\
\hline
{\tt -foldup -cons} & 1.33 & 7474 & 8797 &
	30 & 100 & 365 & 1465 & 5505 \\
\hline
{\tt -foldupx -cons} & 1.10 & 5324 & 6197 &
	30 & 100 & 365 & 1318 & 3502 \\
\hline
{\tt -folddown -cons} & 1.60 & 8402 & 9837 &
	38 & 126 & 446 & 1739 & 6042 \\
\hline\hline
\end{tabular}
\end{center}
\caption{Search Space for Basic ME, ME with refinements, and refinements
combined with the folding-down and folding-up inference rule. The proof with
folding up contains 1 folding step, whereas in the experiment with folding down,the additional inference rule could not be used at all.}
\label{tab:tut2:results.fold}
\end{table}

\section{Reordering}

The basic ME Calculus is based on enumerating solutions for subgoals
until a consistent solution for all literals of the query is found. In
general this leads to a huge search space. To reduce the search space
short clauses should be evaluated first\footnote{This is a good
	heuristic since a short clause increases the number of open
	subgoals less than a long clause. Nevertheless the subproofs
	caused by the chosen short clause might give rise to a larger
	search space than the subproofs that would be caused by a
	rejected long clause.} 
and branches which can not be closed should not be evaluated at all.
 
In order to detect as early as possible if a branch can not be closed
the {\em first-fail principle\/} has been investigated, i.e., the
alternative which will probably lead to a fail is tried first and the
subgoal which is most difficult to prove should be examined first.


\subsection{Static Clause and Subgoal Reordering}

During the preprocessing phase a {\em weak unification graph\/} is
computed, i.e., for each subgoal that clauses are determined who's
heads unify with the subgoal. The unification graph is called {\em
weak\/} because during search for a proof subgoals become
instantiated and then some of these unifications might not be
performable any longer.

While generating the weak unification graph a {\em Static Clause
Reordering\/} is performed, i.e., the order of clauses for an
Extension step is determined by several criterions, including 
\begin{itemize}
\item the length of the clause (short clauses are preferred),
\item the degree of instantiation (ground clauses are preferred
	according to the first-fail principle),
\item the complexity of contained terms (clauses with complex
      terms are preferred according to the first-fail principle).  
\end{itemize}
Because Reduction steps do not increase the number of open subgoals,
but even close the current branch, they are always tried before the
Extension steps\footnote{This is an arbitrary decision: Since also
Extension steps with facts do not increase the number of open subgoals
and close the current branch, Reduction steps could also be tried
after Extension steps with facts.}

The Static Clause Reordering is performed by default. To suppress the
Static Clause Reordering, call
\begin{center}
\begin{verbatim}
inwasm -noclreord MSC006-1
\end{verbatim}
\end{center}

Also a {\em Static Subgoal Reordering\/} is performed during the
preprocessing phase. Here the order of subgoals is computed with
respect to  
\begin{itemize}
\item the degree of instantiation (ground subgoals are preferred
	according to the first-fail principle),
\item the complexity of contained terms (subgoals with complex
	terms are preferred according to the first-fail principle).  
\end{itemize}

During the Static Subgoal Reordering declarative and procedurale
clauses are distinguished. By default, the subgoals of declarative
clauses are reordered and the subgoals of procedurale clauses are not
reordered. To enforce the Static Subgoal Reordering of procedurale
clauses, call
\begin{center}
\begin{verbatim}
inwasm -sgreord MSC006-1
\end{verbatim}
\end{center}
To suppress the Static Subgoal Reordering of procedurale clauses, call
\begin{center}
\begin{verbatim}
inwasm -nosgreord MSC006-1
\end{verbatim}
\end{center}

Table~\ref{tab:tut2:results.static-reord} shows the results for basic
ME, ME with Static Clause Reordering, ME with Static Subgoal
Reordering and ME with Static Clause and Subgoal Reordering.

\begin{table}[htb]
\begin{center}
\begin{tabular}{|l|r|r|r||r||r|r|r|r|r|}
\hline
Method & $t_{\SAM}$ & $n_i$ & $n_f$ & 
	$n_i^3$ & $n_i^4$ & $n_i^5$ & $n_i^6$ & $n_i^7$ \\
\hline\hline
basic & 2.34 & 160606 & 25252 &
	25 & 145 & 2525 & 26262 & 282828 \\
\hline
{\tt -nosgreord -noclreord} & $0.70^\dagger$ & 5848 & 8128 &
	30 & 99 & 409 & 3467 & 1804 \\
\hline
-"- {\tt -cons} & $0.80^\ddagger$ & 3225 & 4335 &
	30 & 94 & 379 & 2070 & 643 \\
\hline
{\tt -nosgreord -cons} & $2.17$ & 12857 & 16513 &
	30 & 94 & 379 & 2070 & 10275 \\
\hline
{\tt -noclreord -cons} & $0.73^{\dagger\dagger}$ & 2961 & 3627 &
	30 & 100 & 359 & 1321 & 1142 \\
\hline
Cl.\ Reord.\ & 2.34 & 160606 & 25252 &
	25 & 145 & 2525 & 26262 & 282828 \\
\hline
Sg.\ Reord.\ & 2.34 & 160606 & 25252 &
	25 & 145 & 2525 & 26262 & 282828 \\
\hline
Sg.\ and Cl.\ Reord.\ & 2.34 & 160606 & 25252 &
	25 & 145 & 2525 & 26262 & 282828 \\
\hline\hline
\end{tabular}
\end{center}
\caption{Search Space for Basic ME, ME with Static Clause Reordering,
         ME with Static Subgoal Reordering and ME with Static Clause
         and Subgoal Reordering.
	Proof $^\dagger$ is different from the others and 
	contains 46 inferences ($^\ddagger$ with 34 inferences,
	$^{\dagger\dagger}$ with 27 inferences)
	in contrast to the ``standard proof'' with 21 inferences.} 
\label{tab:tut2:results.static-reord}
\end{table}


\subsection{Dynamic Subgoal Reordering}

Since during search subgoals get instantiated the initial order
might not be optimal any longer. So a {\em Dynamic Subgoal
Reordering\/} has been investigated. The Dynamic Subgoal
Reordering is realized by means of a {\em Subgoal Selection
Funktion\/} which computes the best subgoal to choose before each call
of a new subgoal.

Again the criterions are 
\begin{itemize}
\item the degree of instantiation (ground subgoals are preferred
	according to the first-fail principle),
\item the complexity of contained terms (subgoals with complex
      terms are preferred according to the first-fail principle).  
\end{itemize}
These criterions are combined to different selection modi rsp.\
different modi of Dynamic Subgoal Reordering. See
Section~\ref{sec:sam} for details. 

The default modus of Dynamic Subgoal Reordering can be invoked by
\begin{center}
\begin{verbatim}
sam -dynsgreord MSC006-1
\end{verbatim}
\end{center}

Table~\ref{tab:tut2:results.dynsgreord} shows the results for basic
ME and for ME with Dynamic Subgoal Reordering.

\begin{table}[htb]
\begin{center}
\begin{tabular}{|l|r|r|r||r|r|r|r|r|r|}
\hline
Method & $t_{\SAM}$ & $n_i$ & $n_f$ & 
	$n_i^3$ & $n_i^4$ & $n_i^5$ & $n_i^6$ & $n_i^7$ \\
\hline\hline
basic & 2.34 & 160606 & 25252 &
	25 & 145 & 2525 & 26262 & 282828 \\
\hline
{\tt -cons} & 1.17 & 5350 & 6389 &
	30 & 100 & 359 & 1321 & 3531 \\
\hline
{\tt -dynsgreord -cons} & 1.38 & 6479 & 8050 &
	30 & 100 & 362 & 1450 & 4528 \\
\hline\hline
\end{tabular}
\end{center}
\caption{Search Space for Basic ME and ME with Dynamic Subgoal Reordering} 
\label{tab:tut2:results.dynsgreord}
\end{table}


\subsection{Subgoal Reordering during Backtracking}

An even more refined version of Subgoal Reordering can be performed
during backtracking. The idea of {\em Subgoal Reordering during
Backtracking\/} is to check after each fail which subgoal is the most
suitable to process and eventually switch from a once chosen subgoal
to another one.

The most important selection criterion is 
\begin{itemize}
\item the length\footnote{Reduction steps are considered as of length
		zero.} 
	of the shortest alternative (subgoals with short alternatives
	are preferred) 
\end{itemize}
In addition,
\begin{itemize}
\item the degree of instantiation,
\item the complexity of contained terms,
\item the predicate symbol,
\item the still available tableau-depth (see
	also Section~\ref{sec:tut2:bounds})
\end{itemize}
can influence the choice. See Section~\ref{sec:sam} for the different
possibilities to combine the selection criterions.

This approach conflicts with the first-fail principle, because the
exhaustion of all the alternatives of a chosen subgoals is given
up. But the approach provides some important advantages, as we will
see. 

Firstly, short proofs are priorized, i.e., if there is a short proof
it will be found earlier by using Subgoal Reordering during
Backtracking. Consider for example a long clause with, e.g., 30
subgoals\footnote{This is not a pathological example since such long
	clauses are the general case in equality handling by
STE-modification \cite{}.}. 
Here, it is very profitable to find a short proof for each subgoal.

Secondly, this method is a step towards breadth-first search. This
provides the possibility of computing some {\em inference look-ahead
value\/} which helps to prune the search space in case of additional
bounds (read Section~\ref{sec:tut2:bounds} about bounds). The
inference look-ahead value is computed by means of the shortest
alternative for each subgoal:
\begin{displaymath}
\mbox{inference look-ahead value} = \sum_{\mbox{open subgoals}} \mbox{length of shortest alternative}
\end{displaymath}
Of course the inference look-ahead value can also be computed in
combination with basic ME or Static/Dynamic Subgoal Reordering. But in
this case the shortest alternative is of length zero for
most\footnote{exactly: for all subgoals that have a Reduction step or
	an Extension step with a fact as shortest alternative} 
subgoals and consequently the inference look-ahead value will not
provide as much information as in combination with Subgoal Reordering
during Backtracking. 

To use SETHEO with Subgoal Reordering during Backtracking, call for
example 
\begin{center}
\begin{verbatim}
sam -singledelay MSC006-1
\end{verbatim}
\end{center}
or 
\begin{center}
\begin{verbatim}
sam -multidelay MSC006-1
\end{verbatim}
\end{center}
See Section~\ref{sec:sam} for details on the different options
concerning Subgoal Reordering during Backtracking.

Table~\ref{tab:tut2:results.delay} shows the results for basic
ME and for ME with Subgoal Reordering during Backtracking.

\begin{table}[htb]
\begin{center}
\begin{tabular}{|l|r|r|r||r|r|r|r|r|r|}
\hline
Method & $t_{\SAM}$ & $n_i$ & $n_f$ & 
	$n_i^3$ & $n_i^4$ & $n_i^5$ & $n_i^6$ & $n_i^7$ \\
\hline\hline
basic & 2.34 & 160606 & 25252 &
	25 & 145 & 2525 & 26262 & 282828 \\
\hline
{\tt -singledelay} & 2.34 & 160606 & 25252 &
	25 & 145 & 2525 & 26262 & 282828 \\
\hline
{\tt -multidelay -cons} & 2.73 & 13158 & 11364 &
	35 & 138 & 634 & 3372 & 8970 \\
\hline\hline
\end{tabular}
\end{center}
\caption{Search Space for Basic ME and ME with Subgoal Reordering
	during Backtracking}  
\label{tab:tut2:results.delay}
\end{table}


\section{Bounds}
\label{sec:tut2:bounds}

Until now, we only have considered the depth-bound (A-literal depth)
and iterative deepening over that bound as a means to obtain completeness.
SETHEO, however, features a large variety of bounds.
In general, the bounds in SETHEO follow the rules:
\begin{itemize}
\item
Each bound can be used separatedly or combined with other bounds.
\item
Iterative deepening is performed with one bound at each time only.
\item
Several bounds are not complete when used alone (e.g., term complexity,
or maximal number of open subgoals).
\item
Bounds can be obtained and set using specific LOP built-in predicates.
These predicates are described in Chapter~\ref{chap:logprg}.
\end{itemize}

For those bounds which allow for iterative deepening, a fixed start-value
is used. The increment can be set using the command-line options.
The following table~\ref{tab:tut2:bounds:list} shows a short
defintion of all bounds available in the current version and their main
features.

\begin{table}[htb]
\begin{center}
\small
\begin{tabular}{|l|l|c|c|l|c|}
\hline
Name & option & complete? & iterate? & iterate with & type \\
\hline\hline
depth & {\tt -d} & Y & Y & {\tt -dr} & L \\
inference & {\tt -i} & Y & Y & {\tt -ir} & G \\
local inference & {\tt -li} & Y & Y & {\tt -lir} & L \\
weighted depth & {\tt -wd} & Y & Y & {\tt -wdr} & L \\
\hline
term complexity & {\tt -tc \#} & N & N & - & - \\
depth dependent term complexity & {\tt -tcd[12]} & N & N & - & - \\
free variables & {\tt -fvars \#} & N & N & - & - \\
open subgoals & {\tt -sgs \#} & N & N & - & - \\
\hline
\end{tabular}
\end{center}
\caption{Search Bounds for SETHEO}
\label{tab:tut2:bounds:list}
\end{table}

Most commonly used are the depth bound, the inference bound, and the
weighted depth bound which has been designed to combine the advantages
of the depth and inference bound. The choice of a bound dramatically
influences the beaviour of SETHEO, as is depicted in
Table~\ref{tab:tut2:bounds.results}. However, there does not seem
to be a ``universally good'' iteration bound for all possible problems.

\begin{table}[htb]
\begin{center}
\footnotesize
\begin{tabular}{|l|r|r|r||r|r|r|r|r|r|r|}
\hline
& $t_{\SAM}$ & $n_i$ & $n_f$ & 
	$b_i:n_i$ &
	$b_i:n_i$ &
	$b_i:n_i$ &
	$b_i:n_i$ &
	$b_i:n_i$ &
	$b_i:n_i$ \\
\hline\hline
{\tt -dr} & 1.17 & 5350 & 6389 &
	3:30 & 4:100 & 5:359 & 6:1321 & 7:3531 & --\\
\hline
{\tt -ir} & $68.08^\dagger$ & 296043 & 365191 &
	5:69 & 8:629 & 11:5369 & 14:29458 & 17:139963 & 20:120555  \\
\hline
{\tt -locir} & 1.20 & 5560 & 6570 &
	5:71 & 6:182 & 7:342 & 8:789 &
	9:1733 & 10:2400 \\
\hline
{\tt -wdr} & 4.52 & 18093 & 22926 &
	3:30 & 4:94 & 5:316 & 6:949 &
	7:3735 & 8:12960 \\
\hline\hline
\end{tabular}
\end{center}
\caption{Search Space for different completeness bounds. $b_i:n_i$ gives the 
	total number of inferences (unifications) performed with bound $b_i$.
	$\dagger$ is the shortest proof with 20 inferences}
\label{tab:tut2:bounds.results}
\end{table}

Therefore, an appropriate completeness bound must be selected upon information
about similar proof tasks within the same domain, and possible additional
information about the proof tasks.

Another approach to deal with this problem is to explore several different
completeness bounds in parallel in a competitive way (as e.g., as SiCoTHEO does):
on each processor, SETHEO tries to solve the problem, using a different
bound (or a combination of bounds). The processor which finds a solution
first, wins and stops the other processors.  
For details on this approach see \cite{Sch96ppai}.
