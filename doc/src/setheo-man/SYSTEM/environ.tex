%%%%%%%%%%%%%%%%%%%%%%%%%%%%%%%%%%%%%%%%%%%%%%%%%%%
%   SETHEO MANUAL
%	(c) J. Schumann, O. Ibens
%	TU Muenchen 1995
%
%	%W% %G%
%%%%%%%%%%%%%%%%%%%%%%%%%%%%%%%%%%%%%%%%%%%%%%%%%%%
\section{The Environment}\label{sec:environ}

The shell scripts mentioned in Section~\ref{sec:overwiew} refer to the
environment variable {\tt SETHEOHOME} which must be defined in the
environment settings. {\tt SETHEOHOME} is the directory where the
binaries of the basic programs \plop, \inw, \wasm, \sam\ and \xp\
reside, e.g.\ {\tt /home/setheo/bin}. This variable is set accordingly,
when the SETHEO system is installed (see Section~\ref{sec:inst-bin}).
You can define the environment variable either by typing 
\begin{verbatim}
setenv SETHEOHOME /home/setheo/bin
\end{verbatim}
or by adding the line
\begin{verbatim}
setenv SETHEOHOME /home/setheo/bin
\end{verbatim}
to your .cshrc/.tcshrc file or by adding the lines
\begin{verbatim}
SETHEOHOME = /home/setheo/bin
export SETHEOHOME
\end{verbatim}
to your .login or bourne-/ ? /... shell startup.
