%%%%%%%%%%%%%%%%%%%%%%%%%%%%%%%%%%%%%%%%%%%%%%%%%%%
%   SETHEO MANUAL
%	(c) J. Schumann, O. Ibens
%	TU Muenchen 1995
%
%	%W% %G%
%%%%%%%%%%%%%%%%%%%%%%%%%%%%%%%%%%%%%%%%%%%%%%%%%%%
\section{Filename Conventions}

All files used within the SETHEO system carry specific file name
extensions. These extensions are mandatory and need {\em not} be given
when a SETHEO command (see Chapter~\ref{chap:2}, \ref{chap:3})
is issued.
In the following we list all possible file name extensions and describe
their contents. Formal definition of the syntax is given in 
Chapter~\ref{chap:5}.

\begin{description}
\item[{\em file}.pl1]
Files with this extension contain formulae in first order predicate
logic in mathematical syntax. Formulae can contain the standard
operators (like $\vee$, $\wedge$, $\rightarrow$, $\leftrightarrow$) and
existential and universal quantifiers. An ASCII representation
of the quantors is given, e.g. {\tt forall} for $\forall$.
A detailed description and formal definition is given in
Section~\ref{sec:pl1-syntax}.

%In the current version of SETHEO, this notation is not supported.
%Note, however, that this syntax is very similar to the input syntax
%of OTTER and can be converted easily (see Section~\ref{sec:plop2otter}.

\item[{\em file}.lop]
Files with this extension contain formulae and/or logic programs
written in LOP syntax. This syntax comprises the default for the
SETHEO system.
A detailed description and the syntax definition is given in
Section~\ref{sec:lop-syntax}. Built-in predicates which can be used
for logic programming are listed in Chapter~\ref{chap:6}.
Files with the extension {\bf .lop} comprise the input of the compiler
{\bf inwasm}.

\item[{\em file}\_pp.lop]
Files with this extension are produced by the compiler {\bf inwasm},
when it is invoked with the command-line option {\tt -lop}.
This file contains the formula after preprocessing. Clauses are fanned
into contra-positives, constraints are added to the clauses
(if selected), and the weak-unification graph is given.
Such a file can be read again by the compiler {\bf inwasm}.
Thus, manual (or automatic) modifications of the formula between
preprocessing and the proof attempt can be accomplished.

\item[{\em file}.out.lop]
Files with this extension are produced by the {\sc Delta} iterator,
when given {\em file\/} as input.
This file contains (in LOP syntax) the original formula (as found
in {\em file\/}{\bf .lop}) with the newly generated unit clauses
appended to it.

\item[{\em file}.s]
Files with this extension contain \SAM\ assembler statements and
are the input of the assembler {\bf wasm}.
Assembler files are generally generated by the compiler {\bf inwasm},
when invoked with the option {\tt -scode}.
Nevertheless, assembler files are (to some extend) human-readable
and can be modified (for prototypical implementation issues).

\item[{\em file}.hex]
Such files are generated by the compiler {\bf inwasm} (per default)
or the assembler {\bf wasm} and contain the
binary (i.e. hexadecimal) representation of \SAM-instructions and
\SAM's symbol table. {\bf .hex} files are directly loaded by the
SETHEO Abstract Machine \SAM\ (command {\tt sam}).

\item[{\em file}.tree]
The prover {\tt sam} generates files with that extension,
if a proof has been found or upon request. Such files contain
one or more Model Elimination Tableaux which then can be displayed
graphically by the proof tree viewer {\bf xptree}.
The syntax of each tableau corresponds to a PROLOG term. Therefore,
tableaux can be read in by a PROLOG system or (after adding
predicate symbols to it) by the compiler {\bf inwasm}.
The syntax of the tree-files are described in Section~\ref{sec:tree-syntax}.

\item[{\em file}.log]
This file is generated by the prover {\tt sam}. It contains useful
information about the current run and comprises
a copy of the data displayed on stdout.
Besides command-line switches and a log of the iterative deepening, this
file contains statistical information which is printed when the
\SAM\ stops (successfully or not).
\end{description}


Two further types of files are used within the SETHEO-system which
do not carry specific extensions.
The \SAM-output file is opened by the built-in predicate {\tt \$tell/1}
and contains all subsequent output as produced by other built-ins
(e.g. {\tt \$write/1} and {\tt \$dumplemma/0}).
For details see Chapter~\ref{chap:6}.

The {\bf xptree} operator file contains  a translation table
of all operators together with their binding power. Its syntax is
defined in Section~\ref{sec:file-formats:opdefs}.

