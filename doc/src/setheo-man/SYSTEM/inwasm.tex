%%%%%%%%%%%%%%%%%%%%%%%%%%%%%%%%%%%%%%%%%%%%%%%%%%
%   SETHEO MANUAL
%	(c) J. Schumann, O. Ibens
%	TU Muenchen 1995
%
%	%W% %G%
%%%%%%%%%%%%%%%%%%%%%%%%%%%%%%%%%%%%%%%%%%%%%%%%%%%
\subsection{inwasm}\label{sec:inwasm}

\Inw\ ({\bf in}put for {\bf w}arren {\bf as}se{\bf m}bler) takes
clauses in \LOP\ 
notation and compiles them into \SAM\ assembler code.  The
output  may  then  be  processed  by \wasm. For  the syntax of
the input language \LOP, see the 
description in Section~\ref{sec:lop-syntax}. \Inw\ distinguishes
between  declarative clauses (axioms) and procedural clauses
(rules). In the Non--Horn case, all declarative clauses are
fanned into contrapositives. Also code to perform reduction
steps is added in this case. 

The Synopsis of \inw\ is the following:

\begin{tabular}{lp{14cm}}
      {\tt inwasm}  &  {\tt [-[no]purity] [-[no]reduct] [-[no]fan] 
                            [-[no|x]sgreord] [-[no]clreord] [-randreord]
                            [-all]
                            [-taut] [-reg] [-eqpred]
                            [-[r]subs] [-overlap[x] [number]] 
                            [-cons] [-ocons[x]] 
                            [-[r]linksubs]
                            [-foldup[x]] [-folddown[x]] 
                            [-notree] [-searchtree] [-interactive]
                            [-verbose [number]]
                            [-lop]}  \\
      {\tt file}    &  
\end{tabular}

The inputfile is the only parameter which has to be given. It must be
a file with extension {\bf .lop}. The output of \inw\ is either a file
with extension {\bf \_pp.lop} (see option {\bf --lop}) or with
extension {\bf .s} (default). {\bf file\_pp.lop} is again an input
file for \inw\ and {\bf file.s} is an input file for \wasm\ (see
Section~\ref{sec:wasm}). 

All optional parameters are listed below:

In the \inw\ some default settings are integrated. For example a
simple reordering  of  input  clauses (short before long) and
their subgoals (according to the first fail principle) is  set
by default.  Switching off (or enforcing) the default settings
can be done with the following options:  

\begin{description}
      \item[--\lb no\rb purity]
           {Suppress or enforce {\it purity reduction\/}. Purity
            reduction means that 
            clauses are deleted, if they contain at least one subgoal
            which is not part of another clause. Purity is performed
            by default.} 
      \item[--\lb no\rb reduct]
           {Suppress or enforce code  generation  for {\it
            reduction\/} steps.  Reduction  steps  guarantee
            completeness in the non--horn case.} 
      \item[--\lb no\rb fan]
           {Suppress or enforce {\it fanning\/} of declarative
            clauses, if the 
            formula contains at least one non--horn clause.}
      \item[--nosgreord]
           {Suppress   {\it subgoal   reordering\/}    of declarative 
            clauses.}  
      \item[--sgreord]
           {Enforce subgoal reordering of procedural clauses.}
      \item[--\lb no\rb clreord]
           {Suppress or enforce {\it clause reordering\/}. Procedural
            clauses and declarative clauses are not distinguished.} 
\end{description}

The default settings of \inw\ are {\bf --purity --reduct --fan
--sgreord  --clreord}. 
Constraints can be computed in order to avoid redundant
substitutions. These are the options for enabling
constraints\footnote{For tautology, subsumption and regularity
constraints see also Section~\ref{sec:sam}.}: 

\begin{description}
      \item[--taut]
           {Instructions to add {\it tautology constraints\/} are added.}
      \item[--reg]
           {Instructions for full {\it regularity\/} check via  {\it
            constraints\/} are inserted.}
      \item[--eqpred]
           {Instructions to detect {\it identical--predecessor--fails\/}
            are added. {\bf --eqpred} implements a restricted form of
            {\bf --reg}.}
      \item[--subs]
           {Instructions to activate {\it unit subsumption constraints\/}
            are inserted.}
      \item[--rsubs]
           {Instructions to activate {\it unit subsumption constraints\/}
            are inserted, but facts are ignored as subsuming clauses.}
      \item[--overlap \lb number\rb]
           {Common initial segments of contrapositives of  declarative
            clauses  are  detected  and extracted. If possible,
            {\it overlap  constraints\/}  are  generated. {\it number\/}
            specifies  the  increasing  complexity [0..10] of these
            overlaps, the default is {\it number\/}~=~2.}
      \item[--overlapx \lb number\rb]
           {Like {\bf --overlap}, but in this case, overlaps are also
            computed for  the procedural clauses. Again, the default is
            {\it number\/}~=~2.}  
      \item[--cons]
           {Regularity, tautology and unit subsumption constraints are
            computed and inserted into the code file. Facts are
            ignored as subsuming clauses. {\bf --cons} is the same as
            {\bf --reg --taut --rsubs}.}  
      \item[--ocons \lb number\rb]
           {Regularity, tautology, unit subsumption and overlap
            constraints are computed and inserted into the code
            file. Facts are 
            ignored as subsuming clauses. Overlaps beetween
            procedural clauses are not computed. {\it number\/}
            specifies  the  increasing complexity [0..10] of these
            overlaps, the default is {\it number\/}~=~2. {\bf --ocons
            \lb number\rb} is the same as {\bf --reg --taut --rsubs
            --overlap \lb number\rb}.} 
      \item[--oconsx \lb number\rb]
           {Regularity, tautology, unit subsumption and overlap
            constraints are computed and inserted into the code
            file. Facts are  ignored as subsuming clauses. Overlaps
            beetween procedural clauses are computed. {\it number\/}
            specifies  the  increasing complexity [0..10] of these
            overlaps, the default is {\it number\/}~=~2. {\bf --oconsx
            \lb number\rb} is the same as {\bf --reg --taut --rsubs
            --overlapx \lb number\rb}.} 
\end{description} 

As an internal representation of the input formula a
weak--unification connection graph  is generated. This
connection graph can be further preprocessed to delete
unnecessary links. The options for link deletion are the
following: 

\begin{description}
      \item[--linksubs \lb number\rb]
           {Tautological and subsumable links are deleted. At  most
            {\it number\/}~*~1000 links are tested. Select the number from
            [1..5]. The default is {\it number\/}~=~6.}
      \item[--rlinksubs \lb number\rb]
           {Tautological and  subsumable  links  are  deleted,  but
            links  into  facts are ignored. Again, at most 
            {\it number\/}~*~1000 links are tested. Select the number from
            [1..5]. The default is {\it number\/}~=~6.} 
\end{description} 

 
The following are additional inference rules for the \SAM. They
can be enabled by setting the concerning \inw\ options:

\begin{description}
      \item[--foldup]
           {Insert code to factorize subgoals with previous ones.}
      \item[--foldupx]
           {Insert code to use previously solved subgoals for pruning
            purposes in an extended regularity check.}
      \item[--folddown]
           {Insert code to factorize subgoals with later ones.}
      \item[--folddownx]
           {Insert code to use later subgoals for pruning  purposes
            in an extended regularity check.} 
\end{description} 

\begin{remark}
Folding up and folding down are  not  compatible  with  each
other.   The  extended regularity checks are not complete in
combination with antilemmata\footnote{See Section~\ref{sec:sam}.}.
\end{remark}

Miscellaneous options:

\begin{description}
      \item[--randreord]
           {Enables random reordering of or--branches (see also \rc\
            in Section~\ref{sec:overwiew}).}
      \item[--all]
           {With this option all possible proofs within the current
            bounds  are enumerated. For each proof \SE\ fails and
            adds the new proof--tree to the {\bf .tree} file. Before
            using 
            \xp\  the  file  must  be  splitted. The option is
            especially useful in combination with  write--statements
            to display answer substituitions of query variables.}
      \item[--notree]
           {Writing the proof tree into a file is disabled.}
      \item[--searchtree]
           {Instructions are inserted to write parts of the  search
            tree into {\bf file.ortree}.}
      \item[--interactive]
           {Instructions are inserted to run the \SAM\ in an  interactive
            mode.}
      \item[--verbose \lb number\rb]
           {Achieves varying  verbosity  specified  with {\it number\/}.
            According to {\it number\/}, you get some intermediate
            preprocessing output. The various possibilities  are
            printed onto the screen when you ommit the {\it number\/}. } 
      \item[--lop]
           {Writes preprocessing output into {\bf file\_pp.lop}. No
            file with extension {\bf .s} is generated. Since {\bf
            file\_pp.lop} itself is a \LOP\ file it can be used as an
            input file for \inw. To avoid multiple fanning all clauses
            in {\bf file\_pp.lop} are procedural clauses.}
\end{description}
