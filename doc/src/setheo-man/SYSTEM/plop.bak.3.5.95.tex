%%%%%%%%%%%%%%%%%%%%%%%%%%%%%%%%%%%%%%%%%%%%%%%%%%%
%   SETHEO MANUAL
%	(c) J. Schumann, O. Ibens
%	TU Muenchen 1995
%
%	%W% %G%
%%%%%%%%%%%%%%%%%%%%%%%%%%%%%%%%%%%%%%%%%%%%%%%%%%%

\subsection{plop}
\label{sec:plop}

{\bf Plop}\footnote{This program is provided on an ``as is'' basis, and is
not maintained any more.} is a translation program which
takes formulae of first order predicate logic in standard notation
and converts them into LOP clausal sets preserving the property
of unsatisfiability. Obvious tautological clauses and redundant
literals are removed. 
Both, the first-order-language syntax and the LOP syntax is described in
a later section. Plop can be invoked by the following command line or via
the Xtheo interface:
\begin{center}
{\bf plop} [ {\bf --necho} ][ {\bf --nopti} ][
{\bf --sko} ][ {\bf --neg} ][ {\bf --baum]} ][
{\bf --debu} ][ {\bf --defgro} ] {\it file\/}[{\bf .pl1}]
\end{center}
The only parameter which has to be given is the name of the input file.
It must have the extension {\bf .pl1}. The generated output file
has the extension {\bf .lop} which can be handled by preproc or inwasm.

The other parameters are optional and have the following meaning:
\begin{description}
\item[--necho]
suppresses the screen output. 
Otherwise the input and output formula is displayed.
\item[--nopti]
suppresses the removal of tautologies and redundancies. 
\item[--sko]
yields output clauses not in LOP form but in skolemized
normal form, i.e. clauses containing disjunction
and negation symbols but neither conditional nor conjunction
symbols.
\item[--neg]
negates the input formula before doing the transformation.
(Some examples or programs may be formulated affirmatively.)
\item[--baum]
displays statistics of the internal formula tree.
\item[--debu]
turns on the verbose debugging mode of plop.
\item[--defgro]
displays constraints concerning the size of some elements
of the input formula. If a given example exceeds them, define-statements
in the source code of PLOP are to be modified.
\end{description}

