\begin{thebibliography}{}

\bibitem[\protect\citeauthoryear{Goller \bgroup \em et al.\egroup
  }{1994}]{GLMS94}
Chr. Goller, R.~Letz, K.~Mayr, and J.~Schumann.
\newblock {SETHEO V3.2: Recent Developments (System Abstract) }.
\newblock In {\em Proc. CADE 12}, pages 778--782, June 1994.

\bibitem[\protect\citeauthoryear{Letz \bgroup \em et al.\egroup
  }{1992}]{LSBB89}
R.~Letz, J.~Schumann, S.~Bayerl, and W.~Bibel.
\newblock {SETHEO}: {A} {H}igh-{P}erformance {T}heorem {P}rover.
\newblock {\em Journal of Automated Reasoning}, 8(2):183--212, 1992.

\bibitem[\protect\citeauthoryear{Letz \bgroup \em et al.\egroup }{1994}]{LMG94}
R.~Letz, K.~Mayr, and C.~Goller.
\newblock {C}ontrolled {I}ntegration of the {C}ut {R}ule into {C}onnection {T}
  ableau {C}alculi.
\newblock {\em Journal Automated Reasoning (JAR)}, (13):297--337, 1994.

\bibitem[\protect\citeauthoryear{Schumann and Letz}{1990}]{SL90}
J.~Schumann and R.~Letz.
\newblock {PARTHEO}: a {H}igh {P}erformance {P}arallel {T}heorem {P}rover.
\newblock In M.~E. Stickel, editor, {\em CADE10}, Lecture Notes in Artificial
  Intelligence, pages 40 -- 56. Springer, 1990.
\begin{quotation}
{\small PARTHEO}, a sound and complete or-parallel theorem prover for first order
logic is presented. The proof calculus is model elimination.
{\small PARTHEO} consists of a uniform network of
sequential theorem provers communicating via message
passing. Each sequential prover is implemented as an
extension of Warren's abstract machine.
{\small PARTHEO} is written in parallel C and running on a network of 16
transputers.
The paper comprises the system architecture, the theoretical background,
details of the implementation,
and results of performance measurements.
\end{quotation}

\bibitem[\protect\citeauthoryear{Schumann \bgroup \em et al.\egroup
  }{1990}]{Sch89}
J.~Schumann, N.~Trapp, and {M.\ van der} Koelen.
\newblock {SETHEO/PARTHEO}: {U}ser's {M}anual.
\newblock Technical Report TUM-I\ 9010. SFB342/7/90 A, Technische
  {Universit\"at M\"unchen}, SFB 342, 1990.

\bibitem[\protect\citeauthoryear{Schumann}{1991}]{Sch91}
J.~Schumann.
\newblock {\em Efficient Theorem Provers based on an Abstract Machine}.
\newblock PhD thesis, Technische Universit\"at M\"unchen, 1991.

\bibitem[\protect\citeauthoryear{Schumann}{1994}]{Sch94a}
J.~Schumann.
\newblock {DELTA --- A Bottom-up Preprocessor for Top-Down Theorem Provers,
  System Abstract}.
\newblock In {\em CADE~12}, 1994.


\bibitem[\protect\citeauthoryear{Schumann}{1995}]{Sch95focus}
J.~Schumann.
\newblock {Using SETHEO for Verifying the Development of a Communication Pro
  tocol in {\sc Focus} -- A Case Study --}.
\newblock In P.~Baumgartner, R.~H{\"a}hnle, and J.~Posegga, editors, {\em
  Proc.~of Workshop Analytic Tableaux and Related Methods, Koblenz}, volume 918
  of {\em LNAI}, pages 338--352. Springer, 1995.
\begin{quotation}
This paper describes experiments with the automated theorem prover
{\small SETHEO}.
The prover is applied to proof tasks which arise during
formal design and specification in {\sc Focus}.

These proof tasks originate from the formal development of a
communication protocol (Stenning protocol).
Its development and verification in {\sc Focus} is described
in ``C. Dendorfer, R. Weber: {\em Development and Implementation of a
Communication Protocol -- An Exercise in {\sc Focus}\/}'' [DW92].
%
A number of propositions of that paper deal with
safety and liveness properties of the Stenning protocol on the level of
traces.
All given propositions and lemmata could be proven
automatically using the theorem prover {\small SETHEO}.

This paper gives a short introduction into
the proof tasks as provided in [DW92].
All steps which were necessary to apply {\small SETHEO} to the
given proof tasks (transformation of syntax, axiomatization) will be described
in detail.
The surprisingly good results obtained by {\small SETHEO} will be presented,
and advantages and problems using an automated theorem prover
for simple, but  frequently occurring proof tasks during a formal development
in {\sc Focus}, as well as possibly ways for improvements
for using {\small SETHEO} as a ``back-end'' for
{\sc Focus} will be discussed.
\end{quotation}

\bibitem[\protect\citeauthoryear{Sutcliffe \bgroup \em et al.\egroup
  }{1994}]{SSY94}
G.~Sutcliffe, C.B. Suttner, and T.~Yemenis.
\newblock {The TPTP Problem Library}.
\newblock In {\em Proceedings of the 12.\ International Conference on Automated
  Deduction (CADE)}, pages 252--266. Springer LNAI 814, 1994.

\end{thebibliography}

